\appendix
\section{Appendix: Example Primer \\ \vspace{2 mm} {\small }}
\label{sec:examplePrimer}
In this Appendix, a set of examples are reported. In order to be as general as possible, the \textit{Model} type ``ExternalModel'' has been used.
%%%% EXAMPLE 1 
\subsection{Example 1.}
\label{subsec:ex1}
This simple example is about the construction of a ``Lorentz attractor'', sampling the relative input space. The parameters that are sampled represent the initial coordinate (x0,y0,z0) of the attractor origin. 

\begin{lstlisting}[style=XML,morekeywords={debug,re,seeding,class,subType,limit}]
<?xml version="1.0" encoding="UTF-8"?>
<Simulation debug="True">
<!-- RUNINFO -->
<RunInfo>
    <WorkingDir>externalModel</WorkingDir>
    <Files>lorentzAttractor.py</Files>
    <Sequence>FirstMRun</Sequence>
    <batchSize>3</batchSize>
</RunInfo>
<!-- STEPS -->
<Steps>
    <MultiRun name='FirstMRun'  re-seeding='25061978'>
        <Input   class='Files'     type=''               >lorentzAttractor.py</Input>
        <Model   class='Models'    type='ExternalModel'  >PythonModule</Model>
        <Sampler class='Samplers'  type='MonteCarlo'     >MC_external</Sampler>
        <Output  class='Datas'     type='Histories'      >testPrintHistories</Output>
        <Output  class='DataBases' type='HDF5'           >test_external_db</Output>
        <Output  class='OutStreamManager' type='Print'   >testPrintHistories_dump</Output>
    </MultiRun >
</Steps>
<!-- MODELS -->
<Models>
    <ExternalModel name='PythonModule' subType='' ModuleToLoad='externalModel/lorentzAttractor'>  
       <variable>sigma</variable>
       <variable>rho</variable>
       <variable>beta</variable>
       <variable>x</variable>
       <variable>y</variable>
       <variable>z</variable>
       <variable>time</variable>
       <variable>x0</variable>
       <variable>y0</variable>
       <variable>z0</variable>
    </ExternalModel>
</Models>
<!-- DISTRIBUTIONS -->
<Distributions>
    <Normal name='x0_distrib'>
        <mean>4</mean>
        <sigma>1</sigma>
    </Normal>
    <Normal name='y0_distrib'>
        <mean>4</mean>
        <sigma>1</sigma>
    </Normal>
    <Normal name='z0_distrib'>
        <mean>4</mean>
        <sigma>1</sigma>
    </Normal>
</Distributions>
<!-- SAMPLERS -->
<Samplers>
    <MonteCarlo name='MC_external'  limit='3' >
      <variable name='x0' >
        <distribution  >x0_distrib</distribution>
      </variable>
      <variable name='y0' >
        <distribution  >y0_distrib</distribution>
      </variable>
      <variable name='z0' >
        <distribution  >z0_distrib</distribution>
      </variable>
    </MonteCarlo>
</Samplers>
<!-- DATABASES -->
<DataBases>
  <HDF5 name="test_external_db"/>
</DataBases>
<!-- OUTSTREAMS -->
<OutStreamManager>
  <Print name='testPrintHistories_dump'>
    <type>csv</type>
    <source>testPrintHistories</source>
  </Print>
</OutStreamManager>
<!-- DATA OBJECTS -->
<Datas>
    <Histories name='testPrintHistories'>
        <Input>x0,y0,z0</Input>
        <Output>time,x,y,z</Output>
   </Histories>
</Datas>
</Simulation>
\end{lstlisting}
The Python \textit{ExternalModel} is reported below:
\begin{lstlisting}[language=python]
import numpy as np

def run(self,Input):
  max_time = 0.03
  t_step = 0.01

  numberTimeSteps = int(max_time/t_step)

  self.x = np.zeros(numberTimeSteps)
  self.y = np.zeros(numberTimeSteps)
  self.z = np.zeros(numberTimeSteps)
  self.time = np.zeros(numberTimeSteps)

  self.x0 = Input['x0']
  self.y0 = Input['y0']
  self.z0 = Input['z0']

  self.x[0] = Input['x0']
  self.y[0] = Input['y0']
  self.z[0] = Input['z0']
  self.time[0]= 0

  for t in range (numberTimeSteps-1):
    self.time[t+1] = self.time[t] + t_step
    self.x[t+1]    = self.x[t] +  self.sigma*
                      (self.y[t]-self.x[t]) * t_step
    self.y[t+1]    = self.y[t] + (self.x[t]*
                      (self.rho-self.z[t])-self.y[t]) * t_step
    self.z[t+1]    = self.z[t] + (self.x[t]*
                          self.y[t]-self.beta*self.z[t]) * t_step
\end{lstlisting}
%%%% EXAMPLE 2 
\subsection{Example 2.}
\label{subsec:ex1}
This example shows a slightly more complicated example, that employs the usage of:
\begin{itemize}
    \item \textit{Samplers:} Grid and Adaptive;
    \item \textit{Models:} External, Reduce Order Models and Post-Processors;
    \item \textit{OutStreams:} Prints and Plots;
    \item \textit{Data Objects:} TimePointSets;
    \item \textit{Functions:} ExternalFunctions.
\end{itemize}
The goal of this input is to compute the ``SafestPoint''.
It provides the coordinates of the farthest
point from the limit surface that is given as an input.
%
The safest point coordinates are expected values of the coordinates of the
farthest points from the limit surface in the space of the ``controllable''
variables based on the probability distributions of the ``non-controllable''
variables.

The term ``controllable'' identifies those variables that are under control
during the system operation, while the ``non-controllable'' variables are
stochastic parameters affecting the system behaviour randomly.

The ``SafestPoint'' post-processor requires the set of points belonging to the
limit surface, which must be given as an input.

\begin{lstlisting}[style=XML,morekeywords={debug,re,seeding,class,subType,limit}]
<Simulation debug="True">

<!-- RUNINFO -->
<RunInfo>
  <WorkingDir>SafestPointPP</WorkingDir>
  <Sequence>pth1,pth2,pth3,pth4</Sequence>
  <batchSize>50</batchSize>
</RunInfo>

<!-- STEPS -->
<Steps>  
  <MultiRun name = 'pth1' pauseAtEnd = 'False'>
    <Sampler  class = 'Samplers'  type = 'Grid'           >grd_vl_ql_smp_dpt</Sampler>
    <Input    class = 'Datas'     type = 'TimePointSet'   >grd_vl_ql_smp_dpt_dt</Input>
    <Model    class = 'Models'    type = 'ExternalModel'  >xtr_mdl</Model>
    <Output   class = 'Datas'     type = 'TimePointSet'   >nt_phy_dpt_dt</Output>    
  </MultiRun >
  
  <MultiRun name = 'pth2' pauseAtEnd = 'True'>
    <Sampler          class = 'Samplers'  type = 'Adaptive'      >dpt_smp</Sampler>
    <Input            class = 'Datas'     type = 'TimePointSet'  >bln_smp_dt</Input>   
    <Model            class = 'Models'    type = 'ExternalModel' >xtr_mdl</Model>
    <Output           class = 'Datas'     type = 'TimePointSet'  >nt_phy_dpt_dt</Output>            
    <SolutionExport   class = 'Datas'     type = 'TimePointSet'  >lmt_srf_dt</SolutionExport>
  </MultiRun>
  
  <PostProcess name='pth3' pauseAtEnd = 'False'>
    <Input    class = 'Datas'          type = 'TimePointSet'       >lmt_srf_dt</Input>
    <Model    class = 'Models'         type = 'PostProcessor'  >SP</Model>
    <Output   class = 'Datas'          type = 'TimePointSet'     >sfs_pnt_dt</Output>
  </PostProcess>  
  
  <OutStreamStep name = 'pth4' pauseAtEnd = 'True'>
  	<Input  class = 'Datas'            type = 'TimePointSet'  >lmt_srf_dt</Input>
  	<Output class = 'OutStreamManager' type = 'Print'         >lmt_srf_dmp</Output>
    <Input  class = 'Datas'            type = 'TimePointSet'  >sfs_pnt_dt</Input>
  	<Output class = 'OutStreamManager' type = 'Print'         >sfs_pnt_dmp</Output>
  </OutStreamStep>
</Steps>

<!-- DATA OBJECTS -->
<Datas> 
  <TimePointSet name = 'grd_vl_ql_smp_dpt_dt'>
    <Input>x1,x2,gammay</Input>
    <Output>OutputPlaceHolder</Output>
  </TimePointSet>
  
  <TimePointSet name = 'nt_phy_dpt_dt'>
    <Input>x1,x2,gammay</Input>
    <Output>g</Output>
  </TimePointSet>
    
  <TimePointSet name = 'bln_smp_dt'>
    <Input>x1,x2,gammay</Input>
    <Output>OutputPlaceHolder</Output>
  </TimePointSet>
    
  <TimePointSet name = 'lmt_srf_dt'>
    <Input>x1,x2,gammay</Input>
    <Output>g_zr</Output>
  </TimePointSet>
  
  <TimePointSet name = 'sfs_pnt_dt'>
    <Input>x1,x2,gammay</Input>
    <Output>p</Output>
  </TimePointSet>
</Datas>

<!-- DISTRIBUTIONS -->
<Distributions>
  <Normal name = 'x1_dst'>
    <upperBound>10</upperBound>
    <lowerBound>-10</lowerBound>
  	<mean>0.5</mean>
    <sigma>0.1</sigma>
  </Normal>
  
  <Normal name = 'x2_dst'>
    <upperBound>10</upperBound>
    <lowerBound>-10</lowerBound>
    <mean>-0.15</mean>
    <sigma>0.05</sigma>
  </Normal>
  
  <Normal name = 'gammay_dst'>
    <upperBound>20</upperBound>
    <lowerBound>-20</lowerBound>
    <mean>0</mean>
    <sigma>15</sigma>
  </Normal>
</Distributions>

<!-- SAMPLERS -->
<Samplers>  
  <Grid name = 'grd_vl_ql_smp_dpt'>
    <variable name = 'x1' >
      <distribution>x1_dst</distribution>
      <grid type = 'value' construction = 'equal' steps = '10' upperBound = '10'>2</grid>
    </variable>  
    <variable name='x2' >
      <distribution>x2_dst</distribution>
      <grid type = 'value' construction = 'equal' steps = '10' upperBound = '10'>2</grid>
    </variable>
    <variable name='gammay' >
      <distribution>gammay_dst</distribution>
      <grid type = 'value' construction = 'equal' steps = '10' lowerBound = '-20'>4</grid>
    </variable>
  </Grid>
  
  <Adaptive name = 'dpt_smp' debug='True'>
    <ROM              class = 'Models'    type = 'ROM'           >accelerated_ROM</ROM>
    <Function         class = 'Functions' type = 'External'      >g_zr</Function>
    <TargetEvaluation class = 'Datas'     type = 'TimePointSet'  >nt_phy_dpt_dt</TargetEvaluation>
    <Convergence limit = '3000' forceIteration = 'False' weight = 'none' persistence = '5'>1e-2</Convergence>
      <variable name = 'x1'>
        <distribution>x1_dst</distribution>
      </variable>
      <variable name = 'x2'>
        <distribution>x2_dst</distribution>
      </variable>
      <variable name = 'gammay'>
        <distribution>gammay_dst</distribution>
      </variable>
  </Adaptive>
</Samplers>

<!-- MODELS -->
<Models>  
  <ExternalModel name = 'xtr_mdl' subType = '' ModuleToLoad = 'SafestPointPP/safest_point_test_xtr_mdl'>
    <variable>x1</variable>
    <variable>x2</variable>
    <variable>gammay</variable>
    <variable>g</variable>
  </ExternalModel>
  
  <ROM name = 'accelerated_ROM' subType = 'SciKitLearn'>
    <Features>x1,x2,gammay</Features>
    <Target>g_zr</Target>
    <SKLtype>svm|SVC</SKLtype>
    <kernel>rbf</kernel>
    <gamma>10</gamma>
    <tol>1e-5</tol>
    <C>50</C>
  </ROM>

  <PostProcessor name='SP' subType='SafestPoint'>
    <!-- List of Objects (external with respect to this PP) needed by this post-processor -->
    <Distribution     class = 'Distributions'  type = 'Normal'>x1_dst</Distribution>
    <Distribution     class = 'Distributions'  type = 'Normal'>x2_dst</Distribution>
    <Distribution     class = 'Distributions'  type = 'Normal'>gammay_dst</Distribution>
    <!- end of the list -->
    <controllable>
    	<variable name = 'x1'>
    		<distribution>x1_dst</distribution>
    		<grid type = 'value' steps = '20'>1</grid>   		
    	</variable>
    	<variable name = 'x2'>
    		<distribution>x2_dst</distribution>
    		<grid type = 'value' steps = '20'>1</grid>   		
    	</variable>
    </controllable>
    <non-controllable>
    	<variable name = 'gammay'>
    		<distribution>gammay_dst</distribution>
    		<grid type = 'value' steps = '20'>2</grid>
    	</variable> 	
    </non-controllable>
  </PostProcessor>
</Models>

<!-- FUNCTIONS -->
<Functions>
  <External name='g_zr' file='SafestPointPP/safest_point_test_g_zr.py'>
    <variable>g</variable>
  </External>
</Functions>

<!-- OUT-STREAMS -->
<OutStreamManager> 
  <Print name = 'lmt_srf_dmp'>
  	<type>csv</type>
  	<source>lmt_srf_dt</source>
  </Print>
  
  <Print name = 'sfs_pnt_dmp'>
  	<type>csv</type>
  	<source>sfs_pnt_dt</source>
  </Print>
</OutStreamManager>

</Simulation>
\end{lstlisting}
The Python \textit{ExternalModel} is reported below:
\begin{lstlisting}[language=python]
def run(self,Input): 
  self.g = self.x1+4*self.x2-self.gammay
\end{lstlisting}
The ``Goal Function'',the function that defines the transitions with respect the input space coordinates, is as follows:
\begin{lstlisting}[language=python]
def __residuumSign(self):
  if self.g<0 : return  1
  else        : return -1
\end{lstlisting}

%%%%% EXAMPLE 3
%\subsection{Example3}
%\label{subsec:ex1}
%example 3
