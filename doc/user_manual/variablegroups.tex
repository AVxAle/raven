\section{VariableGroups}
\label{sec:VariableGroups}

The \xmlNode{VariableGroups} block is an optional input for the convenience of the user.  It allows the
possibility of creating a collection of variables instead of re-listing all the variables in places throughout
the input file, such as DataObjects, ROMs, and ExternalModels.
%
Each entry in the \xmlNode{VariableGroups} block has a distinct name and list of each constituent variable in
the group.
%
Alternatively, set operations can be used to construct variable groups from other variable groups.  In this case,
the dependent groups and the base group on which operations should be performed must be listed.  The following types of
set operations are included in RAVEN:
\begin{itemize}
  \item \texttt{+}, Union, the combination of all variables in the \xmlString{base} set and listed set,
  \item \texttt{-}, Complement, the relative complement of the listed set in the \xmlString{base} set,
  \item \texttt{\^}, Intersection, the variables common to both the \xmlString{base} and listed set,
  \item \texttt{\%}, Symmetric Difference, the variables in only either the \xmlString{base} or listed set,
    but not both.
\end{itemize}
Multiple operations can be performed by separating them with commas in the text of the group node.  In the
event the listed set is a single variable, it will be treated like a set with a single entry.

When using the variable groups in a node, they can be listed alone or as part of a comma-separated list.  The
variable group name will only be substituted in the text of nodes, not attributes or tags.

Each \xmlNode{Group} node has the following attributes:
\vspace{-5mm}
\begin{itemize}
  \itemsep0em
  \item \xmlAttr{name}, \xmlDesc{required string attribute}, user-defined name
  of the group. This is the identifier that will be used elsewhere in the RAVEN input.
  %
  \item \xmlAttr{dependencies}, \xmlDesc{optional comma-separated string attribute}, the other variable groups
    on which this group is dependent for construction.  If listed, all entries in the text of this node must
    be proceeded by one of the set operators above.  Defaults to an empty string.
  \item \xmlAttr{base}, \xmlDesc{optional string attribute}, the starting set for constructing a dependent
    variable group.  This attribute is required if any dependencies are listed.  Set operations are performed
    by performing the chosen operation on the variable group listed in this attribute along with the listed
    variable group.  No default.
\end{itemize}
\vspace{-5mm}

An example of constructing and using variable groups is listed here.  The variable groups \xmlString{x\_odd},
\xmlString{x\_even}, \xmlString{x\_first},  and  \xmlString{y\_group} are constructed independently, and the
remainder are examples of other operations.
\begin{lstlisting}[style=XML,morekeywords={name,file}] %moreemph={name,file}]
<Simulation>
  ...
  <VariableGroups>
    <Group name="x_odd"  >x1,x3,x5</Group>
    <Group name="x_even" >x2,x4,x6</Group>
    <Group name="x_first">x1,x2,x3</Group>
    <Group name="y_group">y1,y2</Group>
    <Group name="add_remove" dependencies="x_first"         base="x_first">-x1,+ x4,+x5</Group>
    <Group name="union"      dependencies="x_odd,x_even"    base="x_odd">+x_even</Group>
    <Group name="complement" dependencies="x_odd,x_first"   base="x_odd">-x_first</Group>
    <Group name="intersect"  dependencies="x_even,x_first"  base="x_even">^x_first</Group>
    <Group name="sym_diff"   dependencies="x_odd,x_first"   base="x_odd">% x_first</Group>
  </VariableGroups>
  ...
  <DataObjects>
    <PointSet name="dataset">
      <Input>union</Input>
      <Output>y_group</Output>
    </PointSet>
  </DataObjects>
  ...
</Simulation>
\end{lstlisting}
