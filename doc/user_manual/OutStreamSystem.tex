\section{OutStream system \\ \vspace{2 mm} }
\label{sec:outstream}
The PRA and UQ framework provides the capabilities to visualize and dump out the
data that are generated, imported (from a system code) and post-processed during
the analysis.
%
These capabilities are contained in the "OutStream" system.
%
Actually, two different OutStream types are available:
\vspace{-5mm}
\begin{itemize}
  \itemsep0em
  \item \textbf{Print}, module that lets the user dump the data contained in the
  internal objects;
  \item \textbf{Plot}, module, based on MatPlotLib~\cite{MatPlotLib}, aimed to
  provide advanced plotting capabilities.
  %
\end{itemize}
\vspace{-5mm}
Both the types listed above only accept as ``input'' a \textit{Data} object
type.
%
This choice has been taken since the ``\textit{Datas}'' system (see
section~\ref{sec:Datas}) has the main advantages, among the others, of ensuring
a standardized approach for exchanging the data/meta-data among the different
framework entities.
%
Every module can project its outcomes into a \textit{Data} object.
%
This provides, to the user, the capability to visualize/dump all the modules'
results.
%
As already mentioned [put reference to the xml input section], the RAVEN
framework input is based on the \textbf{E}xtensible \textbf{M}arkup
\textbf{L}anguage (\textbf{XML}) format.
%
Thus, in order to activate the ``\textit{OutStream}'' system, the input needs to
contain a block identified by the ``\textbf{\xmlNode{OutStreamManager}}'' tag (as
shown below).

\begin{lstlisting}[style=XML]
-----------------------------------------------------------
<OutStreamManager>
    <!-- "OutStream" objects that need to be created-->
</OutStreamManager>
-----------------------------------------------------------
\end{lstlisting}
In the ``OutStreamManager'' block an unlimited number of ``Plot'' and ``Print''
sub-blocks can be inputted.
%
The input specifications and the main capabilities for both types are reported
in the following sections.
%
%%%%%%%%%
% PRINTING SYSTEM
%
%%%%%%%%%
\subsection{Printing system \label{sec:printing}}
The Printing system has been created in order to let the user dump the data,
contained in the internal data objects (see [reference to Data(s) section]), out
at anytime during the calculation.
%
Currently, the only available output is a \textbf{C}omma \textbf{S}eparated
\textbf{V}alue (\textbf{CSV}) file.
%
In the near future, an XML formatted file option will be available.
%
This will facilitate the exchanging of results and provide the possibility to
dump the solution of an analysis and "restart" another one constructing a
\textit{Data} from scratch.
%
The XML code, that is reported below, shows different ways to request a
\textit{Print} OutStream.
%
The user needs to provide a name for each sub-block (XML attribute).
%
These names are then used in the \textit{Steps'} blocks in order to activate the
Printing options at anytime.
%
As shown in the examples below, every \textit{Print} block must contain, at
least, the two required tags:
\vspace{-5mm}
\begin{itemize}
  \itemsep0em
  \item \xmlNode{type}, the output file type (csv or xml).
  %
  \textit{Note, only \textbf{csv} is currently available}
  \item \xmlNode{source}, the \textit{Data} name (one of the \textit{Data} defined in
  the ``\textit{Datas}'' block)
\end{itemize}
\vspace{-5mm}
If only these two tags are provided (as in the ``first-example'' below), the
output file will be filled with the whole content of the ``d-name''
\textit{Data}.
%
\begin{lstlisting}[style=XML]
-----------------------------------------------------------
<OutStreamManager>
  <Print name='first_example'>
    <type>csv</type>
    <source>d-name</source>
  </Print>
  <Print name='second-example'>
    <type>csv</type>
    <source>d-name</source>
    <variables>Output</variables>
  </Print>
  <Print name='third-example'>
    <type>csv</type>
    <source>d-name</source>
    <variables>Input</variables>
  </Print>
  <Print name='forth-example'>
    <type>csv</type>
    <source>d-name</source>
    <variables>Input|var-name-in,Output|var-name-out</variables>
  </Print>
</OutStreamManager>
-----------------------------------------------------------
\end{lstlisting}
If just few parts of the \xmlNode{source} are important for a particular analysis, the
additional XML tag \xmlNode{variables} can be provided.
%
In this block, the variables that need to be dumped must be inputted, in a comma
separated format.
%
The available options, for the \xmlNode{variables} sub-block, are listed below:
\vspace{-5mm}
\begin{itemize}
  \itemsep0em
  \item \textbf{Output}, the output space will be dumped out (see
  ``second-example'')
  \item \textbf{Input}, the input space will be dumped out (see
  ``third-example'')
  \item \textbf{Input|var-name-in/Output|var-name-out}, only the particular
  variables ``var-name-in'' and ``var-name-out'' will be reported in the output
  file (see ``forth-example'')
\end{itemize}
\vspace{-5mm}
Note that all the XML tags are case-sensitive but not their content.
%
%%%%%%%%%
% PLOTTING SYSTEM
%
%%%%%%%%%
\subsection{Plotting system \label{sec:plotting}}
The Plotting system provides all the capabilities to visualize the analysis
outcomes, in real-time or at the post-processing stage.
%
The system is based on the Python library MatPlotLib~\cite{MatPlotLib}.
%
MatPlotLib is a 2D/3D plotting library which produces publication quality
figures in a variety of hardcopy formats and interactive environments across
platforms.
%
This external tool has been wrapped in the RAVEN framework, and is usable by the
user.
%
Since it was unfeasible to support, in the source code, all the interfaces for
all the available plot types, the RAVEN Plotting system directly provide a
formatted input structure for 11 different plot types (2D/3D).
%
The user may request a plot not present among the supported ones, since the
RAVEN Plotting system has the capability to construct on the fly the interface
for a Plot, based on XML instructions.
%
This capability will be discussed in the
sub-section~\ref{sec:Interpretedplotting}.
%
%%%%%%%%%%%%%
% Plot Input Strucutre sub-sub-section
%%%%%%%%%%%%%
\subsubsection{Plot input structure \label{sec:PlotInputStructure}}
In order to create a plot, the user needs to add, within the
\xmlNode{OutStreamManager} block, a \xmlNode{Plot} sub-block.
%
As for the \textit{Print} OutStream, the user needs to specify a name as
attribute of the plot.
%
This name will then be used to request the plot in the \textit{Steps'} block.
%
In addition, the Plot block may need the following attributes:
\vspace{-5mm}
\begin{itemize}
  \itemsep0em
  \item \textbf{dim}, \textit{required integer attribute}, define the
  dimensionality of the plot: 2 (2D) or 3 (3D)
  \item \textbf{interactive}, \textit{optional bool attribute (default=False)'},
  specify if the Plot needs to be interactively created (real-time screen
  visualization)
  \item \textbf{overwrite}, \textit{optional bool attribute (default=False)'},
  if the plot needs to be dumped into picture file/s, does the code need to
  overwrite them every time a new plot (with the same name) is requested?
\end{itemize}
\vspace{-5mm}

As shown, in the XML input example below, the body of the Plot XML input
contains two main sub-nodes:
\vspace{-5mm}
\begin{itemize}
  \itemsep0em
  \item\xmlNode{actions}, where general control options for the figure layout are
  defined (see [])
  \item \xmlNode{plot\_settings}, where the actual plot options are provided
  \vspace{-5mm}
\end{itemize}
These two main sub-block are discussed in the following paragraphs.
%
%%%%%%%%%%%%%
% Actions' block sub-sub-sub section
%%%%%%%%%%%%%
\paragraph{``Actions'' input block \label{sec:actionsBlock}}
The input in the \xmlNode{actions} sub-node is common to all the Plot types, since, in
it, the user specifies all the controls that need to be applied to the figure
style.
%
This block must be unique in the definition of the \xmlNode{Plot} main block.
%
In the following list, all the predefined ``actions'' are reported:
\vspace{-5mm}
\begin{itemize}
  \itemsep0em
  \item \xmlNode{how}, comma separated list of output types:
     \begin{itemize}
    \item \textit{screen}, show the figure on the screen in interactive mode
    \item \textit{pdf}, save the figure as a Portable Document Format file (PDF)
    \item \textit{png}, save the figure as a Portable Network Graphics file
    (PNG)
    \item \textit{eps}, save the figure as a Encapsulated Postscript file (EPS)
    \item \textit{pgf}, save the figure as a LaTeX PGF Figure file (PGF)
    \item \textit{ps}, save the figure as a Postscript file (PS)
    \item \textit{gif}, save the figure as a Graphics Interchange Format (GIF)
    \item \textit{svg}, save the figure as a Scalable Vector Graphics file (SVG)
    \item \textit{jpeg}, save the figure as a jpeg file (JPEG)
    \item \textit{raw}, save the figure as a Raw RGBA bitmap file (RAW)
    \item \textit{bmp}, save the figure as a Windows bitmap file (BMP)
    \item \textit{tiff}, save the figure as a Tagged Image Format file (TIFF)
    \item \textit{svgz}, save the figure as a Scalable Vector Graphics file
    (SVGZ)
      \end{itemize}
  \item \xmlNode{title}, as the name suggests , within this block the user can specify
  the title of the figure.
  %
  In the body, few other keywords (required and not) are present:

% TITLE
 \begin{itemize}
    \item \textit{\xmlNode{text}}, string type, title of the figure
    \item \textit{\xmlNode{kwargs}}, within this block the user can specify optional
    parameters with the following format:
        \begin{lstlisting}[style=XML]
        --------------------------
         <kwargs>
           <param1>value1</param1>
           <param2>value2</param2>
         </kwargs>
        -------------------------
       \end{lstlisting}
    The kwargs block is able to convert whatever string into a python type (for
    example \xmlNode{param1>{'1stKeyword':45}</param1} will be converted into a
    dictionary, \xmlNode{param2>[56,67]</param2} into a list, etc.).
    %
    For reference regarding the available kwargs, see
    ``matplotlib.pyplot.title'' method in~\cite{MatPlotLib}.
    %
      \end{itemize}
% LABEL FORMAT
  \item \xmlNode{label\_format}, within this block the default scale formating can be
  modified.
  %
  In the body, few keywords can be specified (all optional):
 \begin{itemize}
    \item \textit{\xmlNode{style}}, string, the style of the number notation, 'sci' or
    'scientific' for scientific, 'plain' for plain notation.
    %
    Default = scientific
    \item \textit{\xmlNode{scilimits}}, tuple, (m, n), pair of integers; if style is
    ‘sci’, scientific notation will be used for numbers outside the range
    10`m`:sup: to 10`n`:sup:.
    %
    Use (0,0) to include all numbers.
    %
    NB.
    %
    The value for this keyword, needs to be inputted between brackets [for
    example, (5,6)].
    %
    Default = (0,0)
    \item \textit{\xmlNode{useOffset}}, bool or double, if True, the offset will be
    calculated as needed; if False, no offset will be used; if a numeric offset
    is specified, it will be used.
    %
    Default = False
    \item \textit{\xmlNode{axis}}, string, the axis where to apply the defined format,
    'x','y' or 'both'.
    %
    Default = 'both'.
    %
    NB.
    %
    If this action will be used in a 3-D plot, the user can input 'z' as well
    and 'both' will apply this format to all three axis.
    %
      \end{itemize}
% FIGURE PROPERTIES
  \item \xmlNode{figure\_properties}, within this block the user specifies how to
  customize the figure style/quality.
  %
  Thus, through this ``action'' the user has got full control on the quality of
  the figure, its dimensions, etc.
  %
  This control is performed by the following keywords:
 \begin{itemize}
    \item \textit{\xmlNode{figsize}}, tuple (optional), (width, hight), in inches
    \item \textit{\xmlNode{dpi}}, integer, dots per inch
    \item \textit{\xmlNode{facecolor}}, string, set the figure background color
    (please refer to ``matplotlib.figure.Figure'' in~\cite{MatPlotLib} for a
    list of all the colors available)
    \item \textit{\xmlNode{edgecolor}}, string, the figure edge background color
    (please refer to ``matplotlib.figure.Figure'' in~\cite{MatPlotLib} for a
    list of all the colors available)
    \item \textit{\xmlNode{linewidth}}, self explainable keyword
    \item \textit{\xmlNode{frameon}}, bool, if False, suppress drawing the figure
    frame
      \end{itemize}
% RANGE
  \item \xmlNode{range}, the range ``action'' allows to specify the ranges of all the
  axis.
  %
  All the keywords in the body of this block are optional:
     \begin{itemize}
    \item \textit{\xmlNode{ymin}}, double (optional), lower boundary for y axis
    \item \textit{\xmlNode{ymax}}, double (optional), upper boundary for y axis
    \item \textit{\xmlNode{xmin}}, double (optional), lower boundary for x axis
    \item \textit{\xmlNode{xmax}}, double (optional), upper boundary for x axis
    \item \textit{\xmlNode{zmin}}, double (optional), lower boundary for z axis.
    %
    NB.
    %
    Obviously, this keyword is effective in 3-D plots only
    \item \textit{\xmlNode{zmax}}, double (optional), upper boundary for z axis.
    %
    NB.
    %
    Obviously, this keyword is effective in 3-D plots only
      \end{itemize}
% CAMERA
  \item \xmlNode{camera}, the camera item is available in 3-D plots only.
  %
  Through this ``action'', it is possible to orientate the plot as wished.
  %
  The controls are:
     \begin{itemize}
    \item \textit{\xmlNode{elevation}}, double (optional), stores the elevation angle
    in the z plane
    \item \textit{\xmlNode{azimuth}}, double (optional), stores the azimuth angle in
    the x,y plane
      \end{itemize}
% SCALE
  \item \xmlNode{scale}, the scale block allows the specification of the axis scales:
     \begin{itemize}
    \item \textit{\xmlNode{xscale}}, string (optional), scale of the x axis.
    %
    Three options are available: ``linear'',``log'',``symlog''.
    %
    Default = linear
    \item \textit{\xmlNode{yscale}}, string (optional), scale of the y axis.
    %
    Three options are available: ``linear'',``log'',``symlog''.
    %
    Default = linear
    \item \textit{\xmlNode{zscale}}, string (optional), scale of the z axis.
    %
    Three options are available: ``linear'',``log'',``symlog''.
    %
    Default = linear.
    %
    NB.
    %
    Obviously, this keyword is effective in 3-D plots only
      \end{itemize}
% ADD_TEXT
  \item \xmlNode{add\_text}, same as title
% AUTOSCALE
  \item \xmlNode{autoscale}, the autoscale block is a convenience method for simple
  axis view autoscaling.
  %
  It turns autoscaling on or off, and then, if autoscaling for either axis is
  on, it performs the autoscaling on the specified axis or axes.
  %
  The following keywords are available:
     \begin{itemize}
    \item \textit{\xmlNode{enable}}, bool (optional), True turns autoscaling on, False
    turns it off.
    %
    None leaves the autoscaling state unchanged.
    %
    Default = True
    \item \textit{\xmlNode{axis}}, string (optional), string, the axis where to apply
    the defined format, 'x','y' or 'both'.
    %
    Default = 'both'.
    %
    NB.
    %
    If this action will be used in a 3-D plot, the user can input 'z' as well
    and 'both' will apply this format to all three axis.
    %
    \item \textit{\xmlNode{tight}}, bool (optional), if True, set view limits to data
    limits; if False, let the locator and margins expand the view limits; if
    None, use tight scaling if the only artist is an image, otherwise treat
    tight as False.
    %
      \end{itemize}
%HORIZONTAL_LINE
  \item \xmlNode{horizontal\_line}, this ``action'' provides the ability to draw a
  horizontal line in the current figure.
  %
  This capability might be useful, for example, if the user wants to highlight a
  trigger, function of a variable.
  %
  The following keywords are settable:
    \begin{itemize}
    \item \textit{\xmlNode{y}}, double (optional), y coordinate.
    %
    Default = 0
    \item \textit{\xmlNode{xmin}}, double (optional), starting coordinate on the x
    axis.
    %
    Default = 0
    \item \textit{\xmlNode{xmax}}, double (optional), ending coordinate on the x axis.
    %
    Default = 1
    \item \textit{\xmlNode{kwargs}}, within this block the user can specify optional
    parameters with the following format:
        \begin{lstlisting}[style=XML]
        --------------------------
         <kwargs>
           <param1>value1</param1>
           <param2>value2</param2>
         </kwargs>
        -------------------------
       \end{lstlisting}
    The kwargs block is able to convert whatever string into a python type (for
    example \xmlNode{param1>{'1stKeyword':45}</param1} will be converted into a
    dictionary, \xmlNode{param2>[56,67]</param2} into a list, etc.).
    %
    For reference regarding the available kwargs, see
    ``matplotlib.pyplot.axhline'' method in~\cite{MatPlotLib}.
    %
      \end{itemize}
  NB.
  %
  This capability is not available for 3-D plots.
  %
%VERTICAL_LINE
  \item \xmlNode{vertical\_line}, similarly to the ``horizontal\_line'' action, this
  block provides the ability to draw a vertical line in the current figure.
  %
  This capability might be useful, for example, if the user wants to highlight a
  trigger, function of a variable.
  %
  The following keywords are settable:
    \begin{itemize}
    \item \textit{\xmlNode{x}}, double (optional), x coordinate.
    %
    Default = 0
    \item \textit{\xmlNode{ymin}}, double (optional), starting coordinate on the y
    axis.
    %
    Default = 0
    \item \textit{\xmlNode{ymax}}, double (optional), ending coordinate on the y axis.
    %
    Default = 1
    \item \textit{\xmlNode{kwargs}}, within this block the user can specify optional
    parameters with the following format:
        \begin{lstlisting}[style=XML]
        --------------------------
         <kwargs>
           <param1>value1</param1>
           <param2>value2</param2>
         </kwargs>
        -------------------------
       \end{lstlisting}
    The kwargs block is able to convert whatever string into a python type (for
    example \xmlNode{param1>{'1stKeyword':45}</param1} will be converted into a
    dictionary, \xmlNode{param2>[56,67]</param2} into a list, etc.).
    %
    For reference regarding the available kwargs, see
    ``matplotlib.pyplot.axvline'' method in~\cite{MatPlotLib}.
    %
      \end{itemize}
  NB.
  %
  This capability is not available for 3-D plots.
  %
%HORIZONTAL_RECTANGLE
  \item \xmlNode{horizontal\_rectangle}, this ``action'' provides the possibility to
  draw, in the current figure, a horizontally orientated rectangle .
  %
  This capability might be useful, for example, if the user wants to highlight a
  zone in the plot.
  %
  The following keywords are settable:
    \begin{itemize}
    \item \textit{\xmlNode{ymin}}, double (required), starting coordinate on the y
    axis
    \item \textit{\xmlNode{ymax}}, double (required), ending coordinate on the y axis
    \item \textit{\xmlNode{xmin}}, double (optional), starting coordinate on the x
    axis.
    %
    Default = 0
    \item \textit{\xmlNode{xmax}}, double (optional), ending coordinate on the x axis.
    %
    Default = 1
    \item \textit{\xmlNode{kwargs}}, within this block the user can specify optional
    parameters with the following format:
        \begin{lstlisting}[style=XML]
        --------------------------
         <kwargs>
           <param1>value1</param1>
           <param2>value2</param2>
         </kwargs>
        -------------------------
       \end{lstlisting}
    The kwargs block is able to convert whatever string into a python type (for
    example \xmlNode{param1>{'1stKeyword':45}</param1} will be converted into a
    dictionary, \xmlNode{param2>[56,67]</param2} into a list, etc.).
    %
    For reference regarding the available kwargs, see
    ``matplotlib.pyplot.axhspan'' method in~\cite{MatPlotLib}.
    %
      \end{itemize}
  NB.
  %
  This capability is not available for 3-D plots.
  %
%VERTICAL_RECTANGLE
  \item \xmlNode{vertical\_rectangle}, this ``action'' provides the possibility to
  draw, in the current figure, a vertically orientated rectangle .
  %
  This capability might be useful, for example, if the user wants to highlight a
  zone in the plot.
  %
  The following keywords are settable:
    \begin{itemize}
    \item \textit{\xmlNode{xmin}}, double (required), starting coordinate on the x
    axis
    \item \textit{\xmlNode{xmax}}, double (required), ending coordinate on the x axis
    \item \textit{\xmlNode{ymin}}, double (optional), starting coordinate on the y
    axis.
    %
    Default = 0
    \item \textit{\xmlNode{ymax}}, double (optional), ending coordinate on the y axis.
    %
    Default = 1
    \item \textit{\xmlNode{kwargs}}, within this block the user can specify optional
    parameters with the following format:
        \begin{lstlisting}[style=XML]
        --------------------------
         <kwargs>
           <param1>value1</param1>
           <param2>value2</param2>
         </kwargs>
        -------------------------
       \end{lstlisting}
    The kwargs block is able to convert whatever string into a python type (for
    example \xmlNode{param1>{'1stKeyword':45}</param1} will be converted into a
    dictionary, \xmlNode{param2>[56,67]</param2} into a list, etc.).
    %
    For reference regarding the available kwargs, see
    ``matplotlib.pyplot.axvspan'' method in~\cite{MatPlotLib}.
    %
      \end{itemize}
  NB.
  %
  This capability is not available for 3-D plots.
  %
%AXES_BOX
  \item \xmlNode{axes\_box}, this keyword controls the axes' box.
  %
  No body and its value can be 'on' or 'off'.
  %
  \item \xmlNode{axis\_properties}, this block is used to set axis properties.
  %
  There are not fixed keywords.
  %
  If only a single property needs to be set, it can be specified as body of this
  block, otherwise a dictionary-like string needs to be provided.
  %
  For reference regarding the available keys, refer to
  ``matplotlib.pyplot.axis'' method in~\cite{MatPlotLib}.
  %
  \item \xmlNode{grid}, this block is used to define a grid that needs to be added in
  the plot.
  %
  The following keywords can be inputted:
    \begin{itemize}
    \item \textit{\xmlNode{b}}, double (required), starting coordinate on the x axis
    \item \textit{\xmlNode{which}}, double (required), ending coordinate on the x axis
    \item \textit{\xmlNode{axis}}, double (optional), starting coordinate on the y
    axis.
    %
    Default = 0
    \item \textit{\xmlNode{kwargs}}, within this block the user can specify optional
    parameters with the following format:
        \begin{lstlisting}[style=XML]
        --------------------------
         <kwargs>
           <param1>value1</param1>
           <param2>value2</param2>
         </kwargs>
        -------------------------
       \end{lstlisting}
    The kwargs block is able to convert whatever string into a python type (for
    example \xmlNode{param1>{'1stKeyword':45}</param1} will be converted into a
    dictionary, \xmlNode{param2>[56,67]</param2} into a list, etc.).
    %
    For reference regarding the available kwargs, see ``matplotlib.pyplot.grid''
    method in~\cite{MatPlotLib}.
    %
      \end{itemize}
  \vspace{-5mm}
\end{itemize}
%%%%%%%%%%%%%
%Plot block sub-sub-sub section
%%%%%%%%%%%%%
\paragraph{``plot\_settings'' input block \label{sec:plotSettings}}
The sub-block identified by the keyword \xmlNode{plot\_settings} is used to define the
plot characteristics.Within this sub-section at least a \xmlNode{plot} block must be
present.
%
the \xmlNode{plot} sub-section may not be unique within the \xmlNode{plot\_settings}
definition; the number of \xmlNode{plot} sub-blocks is equal to the number of plots
that need to be placed in the same figure.
%
For example, in the following XML cut, a ``line'' and a ``scatter'' type are
combined in the same figure.
%
\begin{lstlisting}[style=XML]
-----------------------------------------------------------
<OutStreamManager>
  <Plot name='example2PlotsCombined' dim='2'>
    <actions>
      <!-- Actions -->
    </actions>
    <plot\_settings>
       <plot>
        <type>line</type>
        <x>d-type|Output|x1</x>
        <y>d-type|Output|y1</y>
      </plot>
       <plot>
        <type>scatter</type>
        <x>d-type|Output|x2</x>
        <y>d-type|Output|y2</y>
      </plot>
      <xlabel>label X</xlabel>
      <ylabel>label Y</ylabel>
    </plot\_settings>
  </Plot>
</OutStreamManager>
-----------------------------------------------------------
\end{lstlisting}
As already mentioned, within the \xmlNode{plot\_settings} block, at least a \xmlNode{plot}
sub-block needs to be inputted.
%
Independently from the plot type, some keywords are mandatory:
\begin{itemize}
  \item \textit{\xmlNode{type}}, string, required parameter, the plot type (for
  example, line, scatter, wireframe, etc.);
  \item \textit{\xmlNode{x}}, string, required parameter, the parameter needs to be
  considered as x coordinate;
  \item \textit{\xmlNode{y}}, string, required parameter, the parameter needs to be
  considered as y coordinate;
  \item \textit{\xmlNode{z}}, string required parameter (3D plots only), the parameter
  needs to be considered as z coordinate.
  %
\end{itemize}
In addition other plot-dependent keywords, reported in
section~\ref{sec:23Dplotting}, can be provided.
%
\\Under the \xmlNode{plot\_settings} block other keywords, common to all the plots the
user decided to combine in the figure, can be inputted, such as:
\begin{itemize}
  \item \textit{\xmlNode{xlabel}}, string, optional parameter, x axis label;
  \item \textit{\xmlNode{ylabel}}, string, optional parameter, y axis laber;
  \item \textit{\xmlNode{zlabel}}, string, optional parameter (3D plots only), z axis
  label;
  \item \textit{\xmlNode{colorMap}}, string, the parameter needs to be used to define
  a color map.
  %
\end{itemize}
As already mentioned, the Plot system accepts as parameter (i.e., x, y, z,
colorMap) the \textbf{Datas} object only.
%
Considering the structure of ``Datas'', the parameters are inputted as follow:
\\ $DataObjectName|Input(or)Output|variableName$.
%
\\ If the ``variableName'' contains the symbol $|$, it must be surrounded by
brackets:
\\ $DataObjectName|Input(or)Output|(whatever|variableName)$.

%%%%%%%%%%%%%
%Predefined Plotting System block sub-sub-sub section
%%%%%%%%%%%%%
\paragraph{Predefined Plotting System: 2D/3D \label{sec:23Dplotting}}
As already mentioned above, the Plotting system provides specialized input
structure for several different kind of plots:
 \begin{itemize}
  \item \textit{2 Dimensional plots}:
       \begin{itemize}
    \item \textit{scatter}.
    %
    2 dimensional scatter plot.
    %
    It used to create a scatter plot of x vs y, where x and y are sequences of
    numbers of the same length;
    \item \textit{line}.
    %
    2 dimensional line plot.
    %
    It used to create a line plot of x vs y, where x and y are sequences of
    numbers of the same length;
    \item \textit{histogram}.
    %
    2 dimensional histogram plot.
    %
    Compute and draw the histogram of x.
    %
    It must be noticed that this plot accepts only the XML node \xmlNode{x} even if it
    is considered as 2D plot type;
    \item \textit{stem}.
    %
    2 dimensional stem plot.
    %
    A stem plot plots vertical lines at each x location from the baseline to y,
    and places a marker there;
    \item \textit{step}.
    %
    2 dimensional step plot;
    \item \textit{pseudocolor}.
    %
    2 dimensional scatter plot.
    %
    It creates a pseudocolor plot of a two dimensional array.
    %
    The two dimensional array is built creating a mesh from \xmlNode{x> and <y} data,
    in conjunction with the data inputted in \xmlNode{colorMap};
    \item \textit{contour}.
    %
    2 dimensional contour plot.
    %
    It plots the contour lines.
    %
    Contour plot is built creating a plot from \xmlNode{x> and <y} data, in
    conjunction with the data inputted in \xmlNode{colorMap};
    \item \textit{filledContour}.
    %
    2 dimensional contour plot.
    %
    It plots the filled contour leveles.
    %
    Filled contour plot is built creating a plot from \xmlNode{x> and <y} data, in
    conjunction with the data inputted in \xmlNode{colorMap};
      \end{itemize}
  \item \textit{3 Dimensional plots}:
       \begin{itemize}
    \item \textit{scatter}.
    %
    3 dimensional scatter plot.
    %
    It is used to create a scatter plot of (x,y) vs z, where x, y, z are
    sequences of numbers of the same length;
    \item \textit{line}.
    %
    3 dimensional line plot.
    %
    It used to create a line plot of (x,y) vs z, where x, y, z are sequences of
    numbers of the same length;
    \item \textit{stem}.
    %
    3 dimensional stem plot.
    %
    It creates a 3 Dimensional stem plot of (x,y) vs z;
    \item \textit{surface}.
    %
    3 dimensional surface plot.
    %
    Create a surface plot of (x,y) vs z.
    %
    By default it will be colored in shades of a solid color, but it also
    supports color mapping;
    \item \textit{wireframe}.
    %
    3 dimensional wire-frame plot.
    %
    No color mapping is supported;
    \item \textit{tri-surface}.
    %
    3 dimensional tri-surface plot.
    %
    It is a surface plot with automatic triangulation;
    \item \textit{contour3D}.
    %
    3 dimensional contour plot.
    %
    It plots the contour lines.
    %
    Contour plot is built creating a plot from \xmlNode{x>, <y> and <z} data, in
    conjunction with the data inputted in \xmlNode{colorMap}
    \item \textit{filledContour3D}.
    %
    3 dimensional filled contour plot.
    %
    It plots the filled contour leveles.
    %
    Filled contour plot is built creating a plot from \xmlNode{x>, <y> and <z} data,
    in conjunction with the data inputted in \xmlNode{colorMap};
    \item \textit{histogram}.
    %
    3 dimensional histogram plot.Compute and draw the histogram of x and y.
    %
    It must be noticed that this plot accepts only the XML nodes \xmlNode{x> and <y}
    even if it is considered as 3D plot type
       \end{itemize}
\end{itemize}
As already mentioned, the settings for each plot type are inputted within the
XML block called \xmlNode{plot}.
%
The sub-nodes that can be inputted depends on the plot type: each plot type has
its own set of parameters that can be specified.
%
\\In the following sub-sections all the options for the plot types listed above
are reported.

\subsubsection{2D \& 3D Scatter plot.}
In order to create a ``scatter'' plot, either 2D or 3D, the user needs to write
in the \xmlNode{type} body the keyword ``scatter''.
%
In order to customize the plot, the user can define the following XML sub-nodes:
  \begin{itemize}
  \item \xmlNode{s}\textbf{\textit{, integer, optional field.}}.
  %
  Size in points\^2.
  %
  \default{20};
  \item \xmlNode{c}\textbf{\textit{, string, optional field.}}.
  %
  color or sequence of color.
  %
  \xmlNode{c} can be a single color format string, or a sequence of color
  specifications of length N, or a sequence of N numbers to be mapped to colors
  using the cmap and norm specified via kwargs.
  %
  Note that \xmlNode{c} should not be a single numeric RGB or RGBA sequence because
  that is indistinguishable from an array of values to be colormapped.
  %
  \xmlNode{c} can be a 2-D array in which the rows are RGB or RGBA;
  \item \xmlNode{marker}\textbf{\textit{, string, optional field.}}.
  %
  Marker type.
  %
  \default{o};
  \item \xmlNode{alpha}\textbf{\textit{, string, optional field.}}.
  %
  The alpha blending value, between 0 (transparent) and 1 (opaque).
  %
  \default{None} ;
  \item \xmlNode{linewidths}\textbf{\textit{, string, optional field.}}.
  %
  Widths of lines.
  %
  Note that this is a tuple, and if you set the linewidths argument you must set
  it as a sequence of floats.
  %
  \default{None};
  \item \textit{\xmlNode{kwargs}}, within this block the user can specify optional
  parameters with the following format:
        \begin{lstlisting}[style=XML]
        --------------------------
         <kwargs>
           <param1>value1</param1>
           <param2>value2</param2>
         </kwargs>
        -------------------------
       \end{lstlisting}
  The kwargs block is able to convert whatever string into a python type (for
  example \xmlNode{param1>{'1stKeyword':45}</param1} will be converted into a
  dictionary, \xmlNode{param2>[56,67]</param2} into a list, etc.).
  %
  For reference regarding the available kwargs, see
  ``matplotlib.pyplot.scatter'' method in~\cite{MatPlotLib}.
  %
    \end{itemize}

\subsubsection{2D \& 3D Line plot.}
In order to create a ``line'' plot, either 2D or 3D, the user needs to write in
the \xmlNode{type} body the keyword ``line''.
%
In order to customize the plot, the user can define the following XML sub-nodes:
  \begin{itemize}
  \item \xmlNode{interpolationType}\textbf{\textit{, string, optional field.}}.
  %
  Type of interpolation algorithm to use for the data.
  %
  Available are ``nearest'', ``linear'', ``cubic'', ``multiquadric'',
  ``inverse'', ``gaussian'', ``Rbflinear'', ``Rbfcubic'', ``quintic'',
  ``thin\_plate''.
  %
  \default{linear};
  \item \xmlNode{interpPointsX}\textbf{\textit{, integer, optional field.}}.
  %
  Number of points need to be used for interpolation of x axis;
  \item \xmlNode{interpPointsY}\textbf{\textit{, integer, optional field.}}.
  %
  Number of points need to be used for interpolation of y axis (only 3D line
  plot);
    \end{itemize}

\subsubsection{2D \& 3D Histogram plot.}
In order to create a ``histogram'' plot, either 2D or 3D, the user needs to
write in the \xmlNode{type} body the keyword ``histogram''.
%
In order to customize the plot, the user can define the following XML sub-nodes:
  \begin{itemize}
  \item \xmlNode{bins}\textbf{\textit{, integer or array\_like, optional field.}}.
  %
  Number of bins if an integer is inputted or a sequence of edges if a python
  list is defined.
  %
  \default{10};
  \item \xmlNode{normed}\textbf{\textit{, boolean, optional field.}}.
  %
  if True normalize the histogram to 1.
  %
  \default{False};
  \item \xmlNode{weights}\textbf{\textit{, sequence, optional field.}}.
  %
  An array of weights, of the same shape as x.
  %
  Each value in x only contributes its associated weight towards the bin count
  (instead of 1).
  %
  If normed is True, the weights are normalized, so that the integral of the
  density over the range remains 1.
  %
  \default{None};
  \item \xmlNode{cumulative}\textbf{\textit{, boolean, optional field.}}.
  %
  If True, then a histogram is computed where each bin gives the counts in that
  bin plus all bins for smaller values.
  %
  The last bin gives the total number of datapoints.
  %
  If normed is also True then the histogram is normalized such that the last bin
  equals 1.
  %
  If cumulative evaluates to less than 0 (e.g., -1), the direction of
  accumulation is reversed.
  %
  In this case, if normed is also True, then the histogram is normalized such
  that the first bin equals 1.
  %
  \default{False} ;
  \item \xmlNode{histtype}\textbf{\textit{, string, optional field.}}.
  %
  The type of histogram to draw:
         \begin{itemize}
    \item \textbf{bar} is a traditional bar-type histogram.
    %
    If multiple data are given the bars are aranged side by side;
    \item \textbf{barstacked} is a bar-type histogram where multiple data are
    stacked on top of each other;
    \item \textbf{step} generates a lineplot that is by default unfilled;
    \item \textbf{stepfilled} generates a lineplot that is by default filled;
         \end{itemize}
  \default{bar};
  \item \xmlNode{align}\textbf{\textit{, string, optional field.}}.
  %
  Controls how the histogram is plotted.
  %
         \begin{itemize}
    \item \textbf{left} bars are centered on the left bin edge;
    \item \textbf{mid} bars are centered between the bin edges;
    \item \textbf{right} bars are centered on the right bin edges;
         \end{itemize}
  \default{mid};
  \item \xmlNode{orientation}\textbf{\textit{, string, optional field.}}.
  %
  Orientation of the histogram:
         \begin{itemize}
    \item \textbf{horizontal};
    \item \textbf{vertical};
         \end{itemize}
  \default{vertical};
  \item \xmlNode{rwidth}\textbf{\textit{, float, optional field.}}.
  %
  The relative width of the bars as a fraction of the bin width.
  %
  \default{None};
  \item \xmlNode{log}\textbf{\textit{, boolean, optional field.}}.
  %
  Set a log scale.
  %
  \default{False};
  \item \xmlNode{color}\textbf{\textit{, string, optional field.}}.
  %
  Color of the histogram.
  %
  \default{blue};
  \item \xmlNode{stacked}\textbf{\textit{, boolean, optional field.}}.
  %
  If True, multiple data are stacked on top of each other If False multiple data
  are aranged side by side if histtype is ‘bar’ or on top of each other if
  histtype is ‘step’.
  %
  \default{False};
  \item \textit{\xmlNode{kwargs}}, within this block the user can specify optional
  parameters with the following format:
        \begin{lstlisting}[style=XML]
        --------------------------
         <kwargs>
           <param1>value1</param1>
           <param2>value2</param2>
         </kwargs>
        -------------------------
       \end{lstlisting}
  The kwargs block is able to convert whatever string into a python type (for
  example \xmlNode{param1>{'1stKeyword':45}</param1} will be converted into a
  dictionary, \xmlNode{param2>[56,67]</param2} into a list, etc.).
  %
  For reference regarding the available kwargs, see ``matplotlib.pyplot.hist''
  method in~\cite{MatPlotLib}.
  %
    \end{itemize}

\subsubsection{2D \& 3D Stem plot.}
In order to create a ``stem'' plot, either 2D or 3D, the user needs to write in
the \xmlNode{type} body the keyword ``stem''.
%
In order to customize the plot, the user can define the following XML sub-nodes:
  \begin{itemize}
  \item \xmlNode{linefmt}\textbf{\textit{, string, optional field.}}.
  %
  Line style.
  %
  \default{b-};
  \item \xmlNode{markerfmt}\textbf{\textit{, string, optional field.}}.
  %
  Marker format.
  %
  \default{bo};
  \item \xmlNode{basefmt}\textbf{\textit{, string, optional field.}}.
  %
  Base format.
  %
  \default{r-};
  \item \textit{\xmlNode{kwargs}}, within this block the user can specify optional
  parameters with the following format:
        \begin{lstlisting}[style=XML]
        --------------------------
         <kwargs>
           <param1>value1</param1>
           <param2>value2</param2>
         </kwargs>
        -------------------------
       \end{lstlisting}
  The kwargs block is able to convert whatever string into a python type (for
  example \xmlNode{param1>{'1stKeyword':45}</param1} will be converted into a
  dictionary, \xmlNode{param2>[56,67]</param2} into a list, etc.).
  %
  For reference regarding the available kwargs, see ``matplotlib.pyplot.stem''
  method in~\cite{MatPlotLib}.
  %
    \end{itemize}

\subsubsection{2D Step plot}
In order to create a 2D ``step'' plot, the user needs to write in the \xmlNode{type}
body the keyword ``step''.
%
In order to customize the plot, the user can define the following XML sub-nodes:
  \begin{itemize}
  \item \xmlNode{where}\textbf{\textit{, string, optional field.}}.
  %
  Positioning:
     \begin{itemize}
    \item \textbf{pre}, the interval from x[i] to x[i+1] has level y[i+1];
    \item \textbf{post}, that interval has level y[i];
    \item \textbf{mid}, the jumps in y occur half-way between the x-values;
     \end{itemize}
  \default{mid};
  \item \textit{\xmlNode{kwargs}}, within this block the user can specify optional
  parameters with the following format:
        \begin{lstlisting}[style=XML]
        --------------------------
         <kwargs>
           <param1>value1</param1>
           <param2>value2</param2>
         </kwargs>
        -------------------------
       \end{lstlisting}
  The kwargs block is able to convert whatever string into a python type (for
  example \xmlNode{param1>{'1stKeyword':45}</param1} will be converted into a
  dictionary, \xmlNode{param2>[56,67]</param2} into a list, etc.).
  %
  For reference regarding the available kwargs, see ``matplotlib.pyplot.step''
  method in~\cite{MatPlotLib}.
  %
    \end{itemize}

\subsubsection{2D Pseudocolor plot}
In order to create a 2D ``pseudocolor'' plot, the user needs to write in the
\xmlNode{type} body the keyword ``pseudocolor''.
%
In order to customize the plot, the user can define the following XML sub-nodes:
  \begin{itemize}
  \item \xmlNode{interpolationType}\textbf{\textit{, string, optional field.}}.
  %
  Type of interpolation algorithm to use for the data.
  %
  Available are ``nearest'', ``linear'', ``cubic'', ``multiquadric'',
  ``inverse'', ``gaussian'', ``Rbflinear'', ``Rbfcubic'', ``quintic'',
  ``thin\_plate''.
  %
  \default{linear};
  \item \xmlNode{interpPointsX}\textbf{\textit{, integer, optional field.}}.
  %
  Number of points need to be used for interpolation of x axis;
  \item \textit{\xmlNode{kwargs}}, within this block the user can specify optional
  parameters with the following format:
        \begin{lstlisting}[style=XML]
        --------------------------
         <kwargs>
           <param1>value1</param1>
           <param2>value2</param2>
         </kwargs>
        -------------------------
       \end{lstlisting}
  The kwargs block is able to convert whatever string into a python type (for
  example \xmlNode{param1>{'1stKeyword':45}</param1} will be converted into a
  dictionary, \xmlNode{param2>[56,67]</param2} into a list, etc.).
  %
  For reference regarding the available kwargs, see ``matplotlib.pyplot.pcolor''
  method in~\cite{MatPlotLib}.
  %
    \end{itemize}

\subsubsection{2D Contour or filledContour plots.}
In order to create a 2D ``Contour'' or ``filledContour'' plot, the user needs to
write in the \xmlNode{type} body the keyword ``contour'' or ``filledContour'',
respectively.
%
In order to customize the plot, the user can define the following XML sub-nodes:
  \begin{itemize}
  \item \xmlNode{number\_bins}\textbf{\textit{, integer, optional field.}}.
  %
  Number of bins.
  %
  \default{5};
  \item \xmlNode{interpolationType}\textbf{\textit{, string, optional field.}}.
  %
  Type of interpolation algorithm to use for the data.
  %
  Available are ``nearest'', ``linear'', ``cubic'', ``multiquadric'',
  ``inverse'', ``gaussian'', ``Rbflinear'', ``Rbfcubic'', ``quintic'',
  ``thin\_plate''.
  %
  \default{linear};
  \item \xmlNode{interpPointsX}\textbf{\textit{, integer, optional field.}}.
  %
  Number of points need to be used for interpolation of x axis;
  \item \textit{\xmlNode{kwargs}}, within this block the user can specify optional
  parameters with the following format:
        \begin{lstlisting}[style=XML]
        --------------------------
         <kwargs>
           <param1>value1</param1>
           <param2>value2</param2>
         </kwargs>
        -------------------------
       \end{lstlisting}
  The kwargs block is able to convert whatever string into a python type (for
  example \xmlNode{param1>{'1stKeyword':45}</param1} will be converted into a
  dictionary, \xmlNode{param2>[56,67]</param2} into a list, etc.).
  %
  For reference regarding the available kwargs, see
  ``matplotlib.pyplot.contour'' method in~\cite{MatPlotLib}.
  %
    \end{itemize}

\subsubsection{3D Surface Plot.}
In order to create a 3D ``Surface'' plot, the user needs to write in the
\xmlNode{type} body the keyword ``surface''.
%
In order to customize the plot, the user can define the following XML sub-nodes:
  \begin{itemize}
  \item \xmlNode{rstride}\textbf{\textit{, integer, optional field.}}.
  %
  Array row stride (step size).
  %
  \default{1};
  \item \xmlNode{cstride}\textbf{\textit{, integer, optional field.}}.
  %
  Array column stride (step size).
  %
  \default{1};
  \item \xmlNode{cmap}\textbf{\textit{, string, optional field.}}.
  %
  Color map.
  %
  \default{jet};
  \item \xmlNode{antialiased}\textbf{\textit{, boolean, optional field.}}.
  %
  Antialiased rendering.
  %
  \default{False};
  \item \xmlNode{linewidth}\textbf{\textit{, integer, optional field.}}.
  %
  Widths of lines.
  %
  Note that this is a tuple, and if you set the linewidths argument you must set
  it as a sequence of floats.
  %
  \default{0};
  \item \xmlNode{interpolationType}\textbf{\textit{, string, optional field.}}.
  %
  Type of interpolation algorithm to use for the data.
  %
  Available are ``nearest'', ``linear'', ``cubic'', ``multiquadric'',
  ``inverse'', ``gaussian'', ``Rbflinear'', ``Rbfcubic'', ``quintic'',
  ``thin\_plate''.
  %
  \default{linear};
  \item \xmlNode{interpPointsX}\textbf{\textit{, integer, optional field.}}.
  %
  Number of points need to be used for interpolation of x axis;
  \item \xmlNode{interpPointsY}\textbf{\textit{, integer, optional field.}}.
  %
  Number of points need to be used for interpolation of y axis;
  \item \textit{\xmlNode{kwargs}}, within this block the user can specify optional
  parameters with the following format:
        \begin{lstlisting}[style=XML]
        --------------------------
         <kwargs>
           <param1>value1</param1>
           <param2>value2</param2>
         </kwargs>
        -------------------------
       \end{lstlisting}
  The kwargs block is able to convert whatever string into a python type (for
  example \xmlNode{param1>{'1stKeyword':45}</param1} will be converted into a
  dictionary, \xmlNode{param2>[56,67]</param2} into a list, etc.).
  %
  For reference regarding the available kwargs, see
  ``matplotlib.pyplot.contour'' method in~\cite{MatPlotLib}.
  %
    \end{itemize}

\subsubsection{3D Wireframe Plot.}
In order to create a 3D ``Wireframe'' plot, the user needs to write in the
\xmlNode{type} body the keyword ``wireframe''.
%
In order to customize the plot, the user can define the following XML sub-nodes:
  \begin{itemize}
  \item \xmlNode{rstride}\textbf{\textit{, integer, optional field.}}.
  %
  Array row stride (step size).
  %
  \default{1};
  \item \xmlNode{cstride}\textbf{\textit{, integer, optional field.}}.
  %
  Array column stride (step size).
  %
  \default{1};
  \item \xmlNode{cmap}\textbf{\textit{, string, optional field.}}.
  %
  Color map.
  %
  \default{jet};
  \item \xmlNode{interpolationType}\textbf{\textit{, string, optional field.}}.
  %
  Type of interpolation algorithm to use for the data.
  %
  Available are ``nearest'', ``linear'', ``cubic'', ``multiquadric'',
  ``inverse'', ``gaussian'', ``Rbflinear'', ``Rbfcubic'', ``quintic'',
  ``thin\_plate''.
  %
  \default{linear};
  \item \xmlNode{interpPointsX}\textbf{\textit{, integer, optional field.}}.
  %
  Number of points need to be used for interpolation of x axis;
  \item \xmlNode{interpPointsY}\textbf{\textit{, integer, optional field.}}.
  %
  Number of points need to be used for interpolation of y axis;
  \item \textit{\xmlNode{kwargs}}, within this block the user can specify optional
  parameters with the following format:
        \begin{lstlisting}[style=XML]
        --------------------------
         <kwargs>
           <param1>value1</param1>
           <param2>value2</param2>
         </kwargs>
        -------------------------
       \end{lstlisting}
  The kwargs block is able to convert whatever string into a python type (for
  example \xmlNode{param1>{'1stKeyword':45}</param1} will be converted into a
  dictionary, \xmlNode{param2>[56,67]</param2} into a list, etc.).
  %
  For reference regarding the available kwargs, see
  ``matplotlib.pyplot.contour'' method in~\cite{MatPlotLib}.
  %
    \end{itemize}

\subsubsection{3D Tri-surface Plot.}
In order to create a 3D ``Tri-surface'' plot, the user needs to write in the
\xmlNode{type} body the keyword ``tri-surface''.
%
In order to customize the plot, the user can define the following XML sub-nodes:
  \begin{itemize}
  \item \xmlNode{color}\textbf{\textit{, string, optional field.}}.
  %
  Color of the surface patches.
  %
  \default{b};
  \item \xmlNode{shade}\textbf{\textit{, boolean, optional field.}}.
  %
  Apply shading.
  %
  \default{False};
  \item \xmlNode{cmap}\textbf{\textit{, string, optional field.}}.
  %
  Color map.
  %
  \default{jet};
  \item \textit{\xmlNode{kwargs}}, within this block the user can specify optional
  parameters with the following format:
        \begin{lstlisting}[style=XML]
        --------------------------
         <kwargs>
           <param1>value1</param1>
           <param2>value2</param2>
         </kwargs>
        -------------------------
       \end{lstlisting}
  The kwargs block is able to convert whatever string into a python type (for
  example \xmlNode{param1>{'1stKeyword':45}</param1} will be converted into a
  dictionary, \xmlNode{param2>[56,67]</param2} into a list, etc.).
  %
  For reference regarding the available kwargs, see
  ``matplotlib.pyplot.contour'' method in~\cite{MatPlotLib}.
  %
    \end{itemize}

\subsubsection{3D Contour or filledContour plots.}
In order to create a 3D ``Contour'' or ``filledContour'' plot, the user needs to
write in the \xmlNode{type} body the keyword ``contour3D'' or ``filledContour3D'',
respectively.
%
In order to customize the plot, the user can define the following XML sub-nodes:
  \begin{itemize}
  \item \xmlNode{number\_bins}\textbf{\textit{, integer, optional field.}}.
  %
  Number of bins.
  %
  \default{5};
  \item \xmlNode{interpolationType}\textbf{\textit{, string, optional field.}}.
  %
  Type of interpolation algorithm to use for the data.
  %
  Available are ``nearest'', ``linear'', ``cubic'', ``multiquadric'',
  ``inverse'', ``gaussian'', ``Rbflinear'', ``Rbfcubic'', ``quintic'',
  ``thin\_plate''.
  %
  \default{linear};
  \item \xmlNode{interpPointsX}\textbf{\textit{, integer, optional field.}}.
  %
  Number of points need to be used for interpolation of x axis;
  \item \xmlNode{interpPointsY}\textbf{\textit{, integer, optional field.}}.
  %
  Number of points need to be used for interpolation of y axis;
  \item \textit{\xmlNode{kwargs}}, within this block the user can specify optional
  parameters with the following format:
        \begin{lstlisting}[style=XML]
        --------------------------
         <kwargs>
           <param1>value1</param1>
           <param2>value2</param2>
         </kwargs>
        -------------------------
       \end{lstlisting}
  The kwargs block is able to convert whatever string into a python type (for
  example \xmlNode{param1}\{'1stKeyword':45\}\xmlNode{/param1} will be converted into a
  dictionary, \xmlNode{param2}\[56,67\]\xmlNode{/param2} into a list, etc.).
  %
  For reference regarding the available kwargs, see
  ``matplotlib.pyplot.contour'' method in~\cite{MatPlotLib}.
  %
    \end{itemize}

%\subsubsection{Interpreted Plotting instruction \label{sec:Interpretedplotting}}

\subsubsection{Example XML input.}
-----------------------------------------------------------
\begin{lstlisting}[style=XML]
<OutStreamManager>
  <Plot name='2DHistoryPlot' dim='2' interactive='False' overwrite='False'>
    <actions>
      <how>pdf,png,eps</how>
      <title>
        <text>***</text>
      </title>
    </actions>
    <plot_settings>
       <plot>
        <type>line</type>
        <x>stories|Output|time</x>
        <y>stories|Output|pipe1_Hw</y>
        <kwargs>
         <color>green</color>
         <label>pipe1-Hw</label>
        </kwargs>
      </plot>
       <plot>
        <type>line</type>
        <x>stories|Output|time</x>
        <y>stories|Output|pipe1_aw</y>
        <kwargs>
         <color>blue</color>
         <label>pipe1-aw</label>
        </kwargs>
      </plot>
      <xlabel>time [s]</xlabel>
      <ylabel>evolution</ylabel>
    </plot_settings>
  </Plot>
</OutStreamManager>
\end{lstlisting}
-----------------------------------------------------------

