\section{Installation}
The installation of the RAVEN code is a straightforward procedure; depending on the usage purpose
and machine architecture, the installation process slightly differs. \\In the following sections, all the different
installation procedures are reported.

\subsection{Framework Source Install}

\subsubsection{Unpacking The Source}

Untar the source install (if there is more than one version of the
source tarball, the full filename will need to be used instead of *):

\begin{lstlisting}[language=bash]
tar -xvzf raven_framework_*_source.tar.gz
\end{lstlisting}

\subsubsection{Dependencies}
\label{raven_dependencies}
RAVEN has a number of dependencies that must be installed before it can be used,
the recommended minimum versions are:

\begin{enumerate}
\item numpy-1.7.0
\item hdf5-1.8.12
\item Cython-0.18
\item h5py-2.2.1
\item scipy-0.12.0
\item scikit-learn-0.14.1
\item matplotlib-1.4.0
\end{enumerate}

Older versions may work, but it is highly recommended to use these
minimum versions.  For newer version of Fedora and Ubuntu, these (or
newer) are available in the distribution repositories.  For OSX
Yosemite, there is a package that can be used called
raven\_libs\_version.dmg For OSX Yosemite and OSX Mavericks there is a
package available called raven\_miniconda.dmg
%
You may install these dependencies yourself, or by running the
\texttt{raven\_libs\_script.sh} script provided within the RAVEN distribution:

%\maljdan{Do we still have the trunk directory?} \alfoa{No}
\begin{lstlisting}[language=bash]
#Only use raven_libs_script.sh if other methods don't work
cd full_path_to_raven_distribution/raven/
./raven_libs_script.sh
cd ..
\end{lstlisting}

\subsubsection{Compilation}

First, the RAVEN modules must be compiled. This can be done either
with the makefile or python setup.

Using the Makefile:
\begin{lstlisting}[language=bash]
cd raven/
make framework_modules
\end{lstlisting}

Using the setup.py files:
\begin{lstlisting}[language=bash]
cd raven/crow/
python setup.py build_ext build install \
--install-platlib=`pwd`/install
cd ..
python setup.py build_ext build  install \
--install-platlib=`pwd`/src/contrib
\end{lstlisting}

\subsubsection{Running the Test Suite}

Next, the tests should be run:

\begin{lstlisting}[language=bash]
#cd into the raven directory if needed
./run_framework_tests
\end{lstlisting}

The output should describe why any tests failed.
%
At the end, there should be a line that looks similar to the output below:
\begin{lstlisting}[language=bash]
8 passed, 19 skipped, 0 pending, 0 failed
\end{lstlisting}

Normally there are skipped tests because either some of the codes are
not available, or some of the test are not currently working.  The
output will explain why each is skipped.

\subsection{Ubuntu Framework Install}

Ubuntu versions before 14.04 do not have new enough libraries, and so
some of them will have to be installed manually, or with the
\verb'raven_libs_script.sh'

\subsubsection{Dependencies}
Install the dependencies using the command:

\begin{lstlisting}[language=bash]
sudo apt-get install libtool git python-dev swig g++ python3-dev \
 python-numpy python-sklearn python-h5py
\end{lstlisting}

%\maljdan{Somebody verify what \TeX packages are needed under Ubuntu
%(These are what I manually installed according to my
%\texttt{/var/lib/apt/extended\_states} file and things work fine).}
% I verified it.  Josh

Optionally, if you want to be able to edit and rebuild the manual, you can
install \TeX~Live and its related packages:
\begin{lstlisting}[language=bash]
sudo apt-get install texlive-latex-base texlive-extra-utils \
  texlive-latex-extra texlive-math-extra
\end{lstlisting}

\subsubsection{Installation}
Untar the binary (if there is more than one version of the
binary, the full filename will need to be used instead of *):

\begin{lstlisting}[language=bash]
tar -xvzf raven_framework_*_ubuntu.tar.gz
\end{lstlisting}

\subsubsection{Running the Test Suite}
Run the tests:

\begin{lstlisting}[language=bash]
cd raven/
./run_framework_tests
\end{lstlisting}

The output should describe why any tests failed.
%
At the end, there should be a line that looks similar to the output below:
\begin{lstlisting}[language=bash]
8 passed, 19 skipped, 0 pending, 0 failed
\end{lstlisting}

Normally there are skipped tests because either some of the codes are
not available, or some of the test are not currently working.  The
output will explain why each is skipped.

\subsection{Fedora Framework Install}

Fedora versions before 21 do not have new enough libraries, and so
some of them will have to be installed manually, or with the
\verb'raven_libs_script.sh'

\subsubsection{Dependencies}
Install the dependencies:

\begin{lstlisting}[language=bash]
yum install swig libtool gcc-c++ python-devel python3-devel \
 numpy h5py scipy python-scikit-learn python-matplotlib-qt4
\end{lstlisting}

Optionally, if you want to be able to edit and rebuild the manual, you can
install \TeX~Live and its related packages:
\begin{lstlisting}[language=bash]
yum install texlive texlive-subfigure texlive-stmaryrd \
  texlive-titlesec texlive-preprint
\end{lstlisting}

\subsubsection{Installation}
Untar the binary (if there is more than one version of the
binary, the full filename will need to be used instead of *):

\begin{lstlisting}[language=bash]
tar -xvzf raven_framework_*_fedora.tar.gz
\end{lstlisting}

\subsubsection{Running The Test Suite}
Run the tests:

\begin{lstlisting}[language=bash]
cd raven/
./run_framework_tests
\end{lstlisting}

The output should describe why any tests failed.
%
At the end, there should be a line that looks similar to the output below:
\begin{lstlisting}[language=bash]
8 passed, 19 skipped, 0 pending, 0 failed
\end{lstlisting}

Normally there are skipped tests because either some of the codes are
not available, or some of the test are not currently working.  The
output will explain why each is skipped.

\subsection{OSX Framework Install}

Open up the file \texttt{raven\_framework\_complete\_version.dmg}.
%
Next, Open up the \texttt{raven\_libs.pkg} inside, and install it.  If
you get an error that the package is not signed, then Control click
the package, and choose ``Open With'' and then Installer.
%
The files will be installed into \texttt{/opt/raven\_libs}.
%
Your \texttt{.bash\_profile} will be modified to source the
\texttt{/opt/raven\_libs/environments/raven\_libs\_profile} file.
%
This file sets up the environment variables \texttt{PYTHONPATH} and
\texttt{PATH} so that the \texttt{raven\_framework} command can be used.

\subsection{MOOSE and RAVEN Source Install}

\subsubsection{MOOSE, RELAP-7, and Other Dependencies}
First, MOOSE should be installed.  Follow the instructions for MOOSE listed
here: \url{http://mooseframework.org/getting-started/}.

Next, if the portion of RAVEN in \texttt{C++} is desired, RELAP-7 needs to be installed.
%
Follow the RELAP-7 instructions available at \url{https://hpcgitlab.inl.gov/idaholab/relap-7}

Ensure that MOOSE and RELAP-7 are installed in the same directory level.

Once any proxy variables are setup (such as \texttt{http\_proxy} and
\texttt{https\_proxy}) and the moose dependencies are setup, the
commands are roughly:

\begin{lstlisting}[language=bash]
git clone https://github.com/idaholab/moose.git
cd  moose/scripts
./update_and_rebuild_libmesh.sh
cd ../../
git clone git@hpcgitlab.inl.gov:idaholab/relap-7.git
cd relap-7
git submodule update --init contrib/iapws
cd ..
\end{lstlisting}

Compiling MOOSE is optional.  If only the MOOSE source code is needed for
RAVEN, then the following can be used instead:

\begin{lstlisting}[language=bash]
git clone https://github.com/idaholab/moose.git
cd moose
git submodule init libmesh
git submodule update libmesh
cd ..
\end{lstlisting}

\subsubsection{Obtaining CROW and RAVEN}
Then clone CROW and RAVEN:

\begin{lstlisting}[language=bash]
git clone git@hpcgitlab.inl.gov:idaholab/crow.git
git clone git@hpcgitlab.inl.gov:idaholab/raven.git
\end{lstlisting}

Install the RAVEN dependencies via one of the methods mentioned for
the RAVEN framework (see Section \ref{raven_dependencies}).  Some of
the moose framework setup methods will already include the raven
dependencies.

\subsubsection{Compilation}
%\maljdan{Do they need to build CROW first?} \alfoa{No.Crow is compiled and distributed in \texttt{raven\_libs\_framework\_and\_crow.dmg}} \cogljj{No, crow is compiled when raven is}

Then compile RAVEN:

\begin{lstlisting}[language=bash]
cd raven
make
\end{lstlisting}

Note that if there are multiple processors available, \texttt{make} can accept a \texttt{-j} option that specifies the number of processors to use, so if there are eight processors available the following will run faster:

\begin{lstlisting}[language=bash]
make -j8
\end{lstlisting}


\subsubsection{Running The Test Suite}
Then run the tests:

\begin{lstlisting}[language=bash]
./run_tests
\end{lstlisting}

The output should describe why any tests failed.
%
At the end, there should be a line that looks similar to the output below:
\begin{lstlisting}[language=bash]
8 passed, 19 skipped, 0 pending, 0 failed
\end{lstlisting}

Normally there are skipped tests because either some of the codes are
not available, or some of the test are not currently working.  The
output will explain why each is skipped.

\subsection{Troubleshooting the Installation}

Often the problems result from one or more of the libraries being
incorrect or missing.  In the raven directory, the command:

\begin{lstlisting}[language=bash]
./run_tests --library_report
\end{lstlisting}
can be used to check if all the libraries are available, and which
ones are being used.  If amsc, distribution1D or interpolationND are
missing, then the RAVEN modules need to be compiled or recompiled.
Otherwise, the RAVEN dependencies need to be fixed.
