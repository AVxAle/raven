\section{Functions}
\label{sec:functions}

\maljdan{Can we rename this?}

\alfoa{Why shoudl we?}

\maljdan{Because a Function will have functions defined in it, and that is confusing.}

The RAVEN code provides support for the usage of user-defined external
functions.
%
These functions are python modules, with a format that is automatically
interpretable by the RAVEN framework.
%
For example, users can define their own method to perform a particular
post-processing activity and the code will be embedded and use the function as
though it were an active part of the code itself.
%
In this section, the XML input syntax and the format of the accepted functions
are fully specified.

The specifications of an external function must be defined within the XML block
\xmlNode{External}.
%
This XML node requires the following attributes:
\vspace{-5mm}
\begin{itemize}
  \itemsep0em
  \item \xmlAttr{name}, \xmlDesc{required string attribute}, user-defined name
  of this function.
  %
  \nb As with other objects, this name can be used to refer to this
  specific entity from other input blocks in the XML.
  %
  \item \xmlAttr{file}, \xmlDesc{required string attribute}, absolute or
  relative path specifying the code associated to this function.
  %
  \nb If a relative path is specified, it must be relative with respect
  to where the user is running the instance of RAVEN.
  %
\end{itemize}
\vspace{-5mm}
In order to make the RAVEN code aware of the variables the user is going to
manipulate/use in her/his own python function, the variables need to be
specified in the \xmlNode{External} input block.
%
The user needs to input, within this block, only the variables directly used by
the external code and not the local variables that the user does not want,
for example, those stored in a RAVEN internal object.
%
These variables are named within consecutive
\xmlNode{variable} XML nodes:
\begin{itemize}
  \item \xmlNode{variable}, \xmlDesc{string, required parameter}, in the body of
  this XML node, the user needs to specify the name of the variable.
  %
  This variable needs to match a variable used/defined in the external python
  function.
  %
\end{itemize}
When the external function variables are defined, at runtime, RAVEN initializes
them and keeps track of their values during the simulation.
%
Each variable defined in the \xmlNode{External} block is available in the
function as a python \texttt{self.} member. In the following, an example of a
user-defined external function is reported (a python module and its related XML
input specifications).

Example Python Function:
\begin{lstlisting}[language=python]
import numpy as np
def residuumSign(self):
  if self.var1 < self.var2 :
    return  1
  else:
    return -1
\end{lstlisting}

Example XML Input:
\begin{lstlisting}[style=XML,morekeywords={name,file}] %moreemph={name,file}]
<Simulation>
  ...
  <Functions>
    ...
    <External name='whatever' file='path_to_python_file'>
     ...
     <variable>var1</variable>
     <variable>var2</variable>
     ...
    </External>
    ...
  </Functions>
   ...
</Simulation>
\end{lstlisting}
