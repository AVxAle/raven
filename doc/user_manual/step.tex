\section{Steps  \\ \vspace{2 mm} {\small }}
\label{sec:steps}
The core of the RAVEN calculation flow is employed in the so called \textbf{Step} system. The \textbf{Step} is in charge to assemble the different ``entities'' in RAVEN (e.g. Samplers, Models, DataBases, etc.) in order to perform a task that is defined by the kind of Step is under usage. A sequence of different \textbf{Steps} represents the calculation flow and core of the analysis the user wants to perform. 
\\Before analyzing each \textbf{Step} type, it is worth to briefly explain how a general Step entity is organized and the concept of ``role'' in its definition.
\\In the following example, a general example of a Step is reported:
\begin{lstlisting}[style=XML]
--------------------------------------------
<Simulation>
  ...
  <Steps>
    ...
    <WhatEverStepType name='***'>
        <Role1 class='***'     type='***'   >***</Role1>
        <Role2 class='***'     type='***'   >***</Role2>
        <Role3 class='***'     type='***'   >***</Role3>
        <Role4 class='***'     type='***'   >***</Role4>
    </WhatEverStepType>
    ...
  </Steps>
  ...
</Simulation>
--------------------------------------------
\end{lstlisting}
As it can be noticed above, independently of the type, each textbf{Step}  is basically constituted by a list of entities organized in ``Roles''.  Each role represents a ``behavior'' the entity (object) is going to take during the evolution of the Step.
In RAVEN several different ``roles'' are available:
\begin{itemize}
\item \textbf{Input.} As the name suggests, it represents the input of the Step. The objects usable as Inputs depends on the type of \textbf{Model} in the Step;
\item \textbf{Output.} The Output represents the container of an action performed by the \textbf{Model}. It can generally be of type \textbf{Datas}, \textbf{DataBases}, \textbf{OutStreamManager};
\item \textbf{Model.} The Model is the actual entity that represents a physical or mathematical representation of a system or behavior. The object that is used in this role, defines the compatibility of the different Inputs and Outputs listed in this step;
\item \textbf{Sampler.} It is the role that defines the Sampling strategy that needs to be performed. 
\item \textbf{Function.} The Function role is extremely important, for example, when performing Adaptive Sampling to represent the metric of the transition regions. This role is the role used, for example, to collapse information coming from a Model.
\item \textbf{ROM.} Acceleration Reduced Order Model;
\item \textbf{SolutionExport,} It represents the container of the eventual output of a Sampler.
\end{itemize}
As understandable, depending on the \textbf{Step} type, different combinations of these roles are allowed to be used.
For this reason, it is important to analyze each \textbf{Step} type in details.
%__interFaceDict['MultiRun'         ] = MultiRun
%__interFaceDict['IOStep'           ] = IOStep
%__interFaceDict['IODataBase'       ] = IOStep
%__interFaceDict['RomTrainer'       ] = RomTrainer
%__interFaceDict['PostProcess'      ] = SingleRun
%__interFaceDict['OutStreamStep'    ] = IOStep
%%%%%%%%%%%%%%%%%%%%
 %%%%% SINGLERUN %%%%%
%%%%%%%%%%%%%%%%%%%%
\subsection{SingleRun}
\label{subsec:stepSingleRun}
The  \textbf{SingleRun} is the simplest step the user can use to assemble a calculation flow. It is aimed to perform a single  action, employed by a \textbf{Model}. For example, it can be used to run a single job (Code Model) and collect the outcomes in a ``Datas'' of type ``TimePoint'' or ``History''.
\\ The specifications of this Step must be defined within the xml block $<SingleRun>$. This xml-node needs to contain the attributes:
\vspace{-5mm}
\begin{itemize}
\itemsep0em
\item \textbf{name}, \textit{required string attribute}, user-defined name of this Step. N.B. As for the other objects, this is the name that can be used to refer to this specific entity in the \textit{RunInfo} block, under the xml-node $<Sequence>$;
\item \textbf{pauseAtEnd}, \textit{optional boolean/string attribute}, if True (True values = True, yes, y, t, si, dajie), in case one, or more, of the Outputs is/are of type ``OutStreamManager'', the code is going to pause at the end of the step, waiting for an user signal to continue. For example, it can be used when a OutStreamManager of type Plot is supposed to be output on the screen, in order to be able to interact with the Plot itself (e.g. rotate the figure, change the scale, etc.).  \textit{Default = False};
\end{itemize}
\vspace{-5mm}
In the \textbf{SingleRun} input block, the user needs to specify the objects that need to be used for the different allowable roles. This step accepts the following roles:
\begin{itemize}
\item $<Input>$, string, required paramete . Name of the ``entity'' that is going to be used as input for the model specified in this step. This xml node needs to contain the following attributes:
\begin{itemize}
  \item \textbf{class}, \textit{required string attribute}, main object class type. The string, required here, corresponds to the tag of the main objects type used in the input. For example, ``Files'', ``Datas'', ``DataBases'', etc;
  \item \textbf{type}, \textit{required string attribute}, the actual entity type. This attribute needs to specify the object type within the main object class. For example, if the  \textbf{class} attribute is ``Datas'', the \textbf{type} attribute might be ``TimePointSet''. NB. The class ``Files'' has no type (i.e. \textbf{type = ``''}).
\end{itemize}
NB. The \textbf{class} and, consequentially,  the \textbf{type} usable for this role depends on the particular $<Model>$ is going to be used. In addition, the user can specify as many $<Input>$ as needed by the model;
\item $<Model>$, string, required parameter. Name of the ``entity'' that is going to be used as Model. This xml node needs to contain the following attributes:
\begin{itemize}
  \item \textbf{class}, \textit{required string attribute}, main object class type. The string, required here, corresponds to the tag of the main objects type used in the input. For this role, only ``Models'' can be used;
  \item \textbf{type}, \textit{required string attribute}, the actual entity type. This attribute needs to specify the object type within the ``Models'' object class. For example, the \textbf{type} attribute might be ``Code'', ``ROM'', etc.
\end{itemize}
\item $<Output>$, string, required parameter. Name of the ``entity'' that is going to be used as Output for the Model. This xml node needs to contain the following attributes:
\begin{itemize}
  \item \textbf{class}, \textit{required string attribute}, main object class type. The string, required here, corresponds to the tag of the main objects type used in the input. For this role, only ``Datas'', ``DataBases'' and ``OutStreamManager'' can be used;
  \item \textbf{type}, \textit{required string attribute}, the actual entity type. This attribute needs to specify the object type within the main object class. For example, if the  \textbf{class} attribute is ``Datas'', the \textbf{type} attribute might be ``TimePointSet''.
\end{itemize}
NB. The number of $<Output>$ nodes is unlimited.
\end{itemize}

\begin{lstlisting}[style=XML]
---------------------------------------------------------
Example:
---------------------------------------------------------
<Steps>
  ...
  <SingleRun name='StepName' pauseAtEnd='false'> 
        <Input   class='Files'     type=''>anInputFile.i</Input>
        <Input   class='Files'     type=''>anotherFileNeededByTheCode</Input>
        <Model   class='Models'    type='Code'      >aCode</Model>
        <Output  class='DataBases' type='HDF5' >aDataBase</Output>
        <Output  class='Datas'     type='History' >aData</Output>
  </SingleRun>
  ...
</Steps>
---------------------------------------------------------
\end{lstlisting}
%%%%%%%%%%%%%%%%%%%
 %%%%% MULTIRUN %%%%%
%%%%%%%%%%%%%%%%%%%
\subsection{SingleRun}
\label{subsec:stepSingleRun}
The  \textbf{SingleRun} is the simplest step the user can use to assemble a calculation flow. It is aimed to perform a single  action, employed by a \textbf{Model}. For example, it can be used to run a single job (Code Model) and collect the outcomes in a ``Datas'' of type ``TimePoint'' or ``History''.
\\ The specifications of this Step must be defined within the xml block $<SingleRun>$. This xml-node needs to contain the attributes:
\vspace{-5mm}
\begin{itemize}
\itemsep0em
\item \textbf{name}, \textit{required string attribute}, user-defined name of this Step. N.B. As for the other objects, this is the name that can be used to refer to this specific entity in the \textit{RunInfo} block, under the xml-node $<Sequence>$;
\item \textbf{pauseAtEnd}, \textit{optional boolean/string attribute}, if True (True values = True, yes, y, t, si, dajie), in case one, or more, of the Outputs is/are of type ``OutStreamManager'', the code is going to pause at the end of the step, waiting for an user signal to continue. For example, it can be used when a OutStreamManager of type Plot is supposed to be output on the screen, in order to be able to interact with the Plot itself (e.g. rotate the figure, change the scale, etc.).  \textit{Default = False};
\end{itemize}
\vspace{-5mm}
In the \textbf{SingleRun} input block, the user needs to specify the objects that need to be used for the different allowable roles. This step accepts the following roles:
\begin{itemize}
\item $<Input>$, string, required paramete . Name of the ``entity'' that is going to be used as input for the model specified in this step. This xml node needs to contain the following attributes:
\begin{itemize}
  \item \textbf{class}, \textit{required string attribute}, main object class type. The string, required here, corresponds to the tag of the main objects type used in the input. For example, ``Files'', ``Datas'', ``DataBases'', etc;
  \item \textbf{type}, \textit{required string attribute}, the actual entity type. This attribute needs to specify the object type within the main object class. For example, if the  \textbf{class} attribute is ``Datas'', the \textbf{type} attribute might be ``TimePointSet''. NB. The class ``Files'' has no type (i.e. \textbf{type = ``''}).
\end{itemize}
NB. The \textbf{class} and, consequentially,  the \textbf{type} usable for this role depends on the particular $<Model>$ is going to be used. In addition, the user can specify as many $<Input>$ as needed by the model;
\item $<Model>$, string, required parameter. Name of the ``entity'' that is going to be used as Model. This xml node needs to contain the following attributes:
\begin{itemize}
  \item \textbf{class}, \textit{required string attribute}, main object class type. The string, required here, corresponds to the tag of the main objects type used in the input. For this role, only ``Models'' can be used;
  \item \textbf{type}, \textit{required string attribute}, the actual entity type. This attribute needs to specify the object type within the ``Models'' object class. For example, the \textbf{type} attribute might be ``Code'', ``ROM'', etc.
\end{itemize}
\item $<Output>$, string, required parameter. Name of the ``entity'' that is going to be used as Output for the Model. This xml node needs to contain the following attributes:
\begin{itemize}
  \item \textbf{class}, \textit{required string attribute}, main object class type. The string, required here, corresponds to the tag of the main objects type used in the input. For this role, only ``Datas'', ``DataBases'' and ``OutStreamManager'' can be used;
  \item \textbf{type}, \textit{required string attribute}, the actual entity type. This attribute needs to specify the object type within the main object class. For example, if the  \textbf{class} attribute is ``Datas'', the \textbf{type} attribute might be ``TimePointSet''.
\end{itemize}
NB. The number of $<Output>$ nodes is unlimited.
\end{itemize}

\begin{lstlisting}[style=XML]
---------------------------------------------------------
Example:
---------------------------------------------------------
<Steps>
  ...
  <SingleRun name='StepName' pauseAtEnd='false'> 
        <Input   class='Files'     type=''>anInputFile.i</Input>
        <Input   class='Files'     type=''>anotherFileNeededByTheCode</Input>
        <Model   class='Models'    type='Code'      >aCode</Model>
        <Output  class='DataBases' type='HDF5' >aDataBase</Output>
        <Output  class='Datas'     type='History' >aData</Output>
  </SingleRun>
  ...
</Steps>
---------------------------------------------------------
\end{lstlisting}



\begin{itemize}
%%%%%%%%%%%%%%%%%%%%%%%%%%%%%%%%%%%%%%%%%%%%%%%%%%%%%%
\item MultiRun: This class implements one step of the simulation pattern where several runs are, needed without being adaptive.  
\begin{itemize}
\item
	\begin{itemize}
	\item name = name of the step (sequence) defined in the RunInfo block, under the Sequence card
	\end{itemize}
\item pauseAtEnd= if True the code will not go to next step until plots are closed manually by the user
\end{itemize}
\begin{itemize}
\item Sampler: 
\begin {itemize}
\item class=Samplers 
\item type: the type of sampler used in the Samplers block 
\item name of the sampler
\end{itemize}
\item Model: 
\begin{itemize}
\item class= (Models) 
\item type=(dummy, ROM, External Model, Code, Projector, PostProcessor) 
\item name of the model
\end{itemize}
\item Input:  
\begin{itemize}
\item class=(Data e Files)
\item type:
\begin{itemize}
\item if class = files ----$>$ none
\item if class = Data  ----$>$ timepoint, timepointset, historie, histories
\end{itemize}
\end{itemize}
\item Output: 
\begin{itemize}
\item class: they are the output destinations
\begin{itemize}
\item Datas 
\item OutStreamManager
\end{itemize} 
\item type:
\begin{itemize}
\item if class=Datas ----$>$(timepoint, timepointset, historie, histories) 
\item if class=OutstreamManager ---$>$ type:print, plot
\end{itemize} 
\item name of the output  
\end{itemize}

\end{itemize}
\begin{lstlisting}[style=XML]
<MultiRun name='***' pauseAtEnd='***'>
 <Sampler class='Samplers' type='***'>***</Sampler>
 <Input class='***' type='***'>***</Input>
 <Model class='Models' type='***'>***</Model>
 <Output class='Datas'*** type='***'</Output>
 <Output class='OutStreamManager' type='***'>***</Output>
</MultiRun>
\end{lstlisting}
%%%%%%%%%%%%%%%%%%%%%%%%%%%%%%%%%%%%%%%%%%%%%%%%%%%%
\item Adaptive: this class implement one step of the simulation pattern where several runs are needed in an adaptive scheme
\begin{itemize}
\item Sampler: class=samplers type=Adaptive>???<
\item Model: class=Models type=(dummy, ROM, External Model, Code, Projector, PostProcessor)
\item Function: it takes in a datas and generate the value of the goal functions, it gives the criteria for which i represent the limit surface (class=Functions)
\item Input:
\item TargetEvaluation: is the output datas that is used for the evaluation of the goal function, it represents the sampling points. It has to be declared among the outputs. 
\item SolutionExport: if declared it is used to export the location of the goal functions = 0, it exports the limit surface
\item ROM: is boolean, it selects values of input (through sampling) to find the limit surface through the values that were given by the function. 
\item Output:
\end{itemize}
\begin{lstlisting}[style=XML]
<Adaptive name='***' pauseAtEnd='***'> 
 <Input class = 'Datas' type = 'TimePointSet'>***</Input>
 <Sampler class = 'Samplers' type = 'Adaptive'>***</Sampler>
 <TargetEvaluation class = 'Datas' type = '***'>***</TargetEvaluation>
\end{lstlisting}
%%%%%%%%%%%%%%%%%%%%%%%%%%%%%%%%%%%%%%%%%%%%%%%%%%%%%%%
\item IODataBase: This step type is used only to extract or push information from/into a DataBase. If in the Databases block the Database is created (no directory or filename) the then the Databse will be an output of this block, otherwise, if it is uploaded then in this block it will be a Input
\begin{itemize}
\item Input: class=(Datas or Databases) type=(if Databases, HDF5, if Datas:timepoint, timepointset, historie, histories) 
\item Output: class=(Datas or Databases) type=(if Databases, HDF5, if Datas:timepoint, timepointset, historie, histories) 
\end{itemize}
%%%%%%%%%%%%%%%%%%%%%%%%%%%%%%%%%%%%%%%%%%%%%%%%%%%%%%%%
\label{subsec:stepTraining}
\item RomTrainer: This step type is used only to train a ROM. 
\begin{lstlisting}[style=XML]
<RomTrainer name='***'>
 <Input   class='Datas' type='TimePointSet'>***</Input>
\end{lstlisting}
%%%%%%%%%%%%%%%%%%%%%%%%%%%%%%%%%%%%%%%%%%%%%%%%%%%%%%%%
\item PostProcess:
\item OutStreamStep:
\end{itemize}











