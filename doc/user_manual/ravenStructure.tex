\section{Raven Input Structure  \\ \vspace{2 mm} {\small }}
The RAVEN code does not have a fixed calculation flow, since all its basic objects can be combined in order to create a user-defined calculation flow. Thus, its input (xml) is organized in different xml blocks, each with a different functionality. 
The main input blocks are as follows:
\begin{itemize}
\item \textbf{Simulation}: The simulation block is the one that has inside the entire input, all the following blocks fit inside the simulation block;
\item \textbf{RunInfo}: Block in which the calculation settings are specified (number of parallel simulations, etc.);
\item \textbf{Distributions}: Distributions' container;
\item \textbf{Samplers}: Exploration of the uncertain domain strategy specification;
\item \textbf{Functions}: External functions container;
\item \textbf{Models}: Models' specifications (e.g. Codes,ROM,etc.);
\item \textbf{Steps}: Place where the single basic objects get combained;
\item \textbf{Datas}: Internal Data object block;
\item \textbf{Databases}: Databases block;
\item \textbf{OutStream system}: Visualization and Printing system block.
\end{itemize} 
Each of these blocks are explained in dedicated sections in the following chapters. 


