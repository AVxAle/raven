\section{Raven Input Structure  \\ \vspace{2 mm} {\small }}
The RAVEN code does not have a fixed calculation flow, since all its basic objects can be combined in order to create a user-defined calculation flow. Thus, its input (XML) is organized in different XML blocks, each with a different functionality. 
The main input blocks are as follows:
\begin{itemize}
\item \textbf{Simulation}: As the name suggests, this XML block represents the container of the Simulation is going to be performed by RAVEN. The simulation block covers the entire input, all the following XML nodes fit inside the Simulation section;
\item \textbf{RunInfo}: This XML block is the place where the user specifies how the calculation needs to be performed (i.e where the calculation settings, such as number of parallel jobs to be spooned, etc,  are specified;
\item \textbf{Distributions}: Distributions' container;
\item \textbf{Samplers}: Exploration of the uncertain domain strategy specification;
\item \textbf{Functions}: External functions container;
\item \textbf{Models}: Models' specifications (e.g. Codes,ROM,etc.);
\item \textbf{Steps}: Place where the single basic objects get combained;
\item \textbf{Datas}: Internal Data object block;
\item \textbf{Databases}: Databases block;
\item \textbf{OutStream system}: Visualization and Printing system block.
\end{itemize} 
Each of these blocks are explained in dedicated sections in the following chapters. 


