\section{Raven Input Structure}
The RAVEN code does not have a fixed calculation flow, since all of its basic
objects can be combined in order to create a user-defined calculation flow.
%
Thus, its input (XML format) is organized in different XML blocks, each with a
different functionality.
%
The main input blocks are as follows:
\begin{itemize}
  \item \xmlNode{Simulation}: The root node containing the
  entire input, all of
  the following blocks fit inside the \emph{Simulation} block.
  %
  \item \xmlNode{RunInfo}: Specifies the calculation
  settings (number of parallel simulations, etc.).
  %
  \item \xmlNode{Files}: Specifies the files to be
  used in the calculation.
  %
  \item \xmlNode{Distributions}: Defines distributions
  needed for describing parameters, etc.
  %
  \item \xmlNode{Samplers}: Sets up the strategies used for
  exploring an uncertain domain.
  %
  \item \xmlNode{DataObjects}: Specifies internal data objects
  used by RAVEN.
  %
  \item \xmlNode{Databases}: Lists the HDF5 databases used
  as input/output to a
  RAVEN run.
  %
  \item \xmlNode{OutStreams}: Visualization and
  Printing system block.
  %
  \item \xmlNode{Models}: Specifies codes, ROMs,
  post-processing analysis, etc.
  %
  \item \xmlNode{Functions}: Details interfaces to external
  user-defined functions and modules.
  %
  the user will be building and/or running.
  %
  \item \xmlNode{Steps}: Combines other blocks to detail a
  step in the RAVEN workflow including I/O and computations to be performed.
  %
\end{itemize}

Each of these blocks are explained in dedicated sections in the following
chapters.
%
\subsection{Comments}
Comments may be included in the RAVEN input using standard XML comments,
using \verb|<!--| and \verb|-->| as shown in the example below.
\begin{lstlisting}[style=XML]
<Simulation>
  ...
  <!-- An Example Comment -->
  <Samplers>
  ...
\end{lstlisting}
Comments may be placed anywhere \emph{except} before the \xmlNode{Simulation}
node or after the \xmlNode{/Simulation} node.
%
Comments outside the root node will cause errors in maintaining input file
compatability.
%
Additionally, comments must completely surround any nodes they comment out.
%
Comments are intended to completely remove blocks of code, or to add readability.
%
For instance, the following is INCORRECT usage:
\begin{lstlisting}[style=XML]
  <!--<Assembler> -->
  <!--</Assembler> -->
\end{lstlisting}
%
and the following is compatible usage for a code block:
%
\begin{lstlisting}[style=XML]
  <!--<Samplers>
    <Monte Carlo name='mc'>
      ...
    </Monte Carlo>
    ...
  </Samplers> -->
\end{lstlisting}


\subsection{Verbosity}
\label{sec:verbosity}
Each block within RAVEN also makes use of a \xmlAttr{verbosity} system,
which allows a user to control the level of output to the user interface.
These settings are declared globally as attributes in the \xmlNode{Simulation} node,
and locally in each block.  The verbosity levels are
\begin{itemize}
\item \xmlString{silent} - Only simulation-breaking errors are displayed.
\item \xmlString{quiet} - Errors as well as warnings are displayed.
\item \xmlString{all} (default) - Errors, warnings, and messages are displayed.
\item \xmlString{debug} - For developers. All errors, warnings, messages, and debug messages are displayed.
\end{itemize}
Examples of verbosity usage are included in many examples throughout this manual.

At the \xmlNode{Simulation} node, the following global variables can be set:
\begin{itemize}
  \item \xmlAttr{verbosity}, optional string, determines the global verbosity level.  Defaults to
    \xmlString{all}.
  \item \xmlAttr{printTimeStamps}, optional boolean, determines whether time stamps will be added to printed
    messages.  Defaults to true.
  \item \xmlAttr{color}, optional boolean, determines whether ANSI color tags will be used in printed
    messages.  Defaults to false.
\end{itemize}


\subsection{External Input Files}
The \xmlNode{ExternalXML} node defines external input file (XML format) that can be used to replace any XML nodes
under \xmlNode{Simulation} in the RAVEN input file. This node allows a user to load any external input file that contains
the required XML nodes into the RAVEN input file. Each \xmlNode{ExternalXML} node has the following attributes:
\begin{itemize}
\item \xmlAttr{node}, \xmlDesc{required string attribute}, user-defined XML node of RAVEN input file.
\item \xmlAttr{xmlToLoad}, \xmlDesc{required string attribute}, file name with its absolute or relative path. Note: if a
relative path is specified, it must be relative with respect to the RAVEN input file.
\end{itemize}
%
For example, if the file \texttt{Models.xml} contain the required RAVEN input XML node \xmlNode{Models},
the RAVEN input file might appear as:
%
\begin{lstlisting}[style=XML,morekeywords={node,xmlToLoad}]
<Simulation>
  ...
  <Steps>
    ...
  </Steps>
  ...
  <ExternalXML node='Models' xmlToLoad='external_input/Models.xml'/>
  ...
</Simulation>
\end{lstlisting}
%
Another example, if the file \texttt{MultiRun.xml} contain the required RAVEN input XML node \xmlNode{MultiRun}
under node \xmlNode{Steps}, the RAVEN input file might appear as:
\begin{lstlisting}[style=XML,morekeywords={node,xmlToLoad}]
<Simulation>
  ...
  <Steps>
    ...
    <ExternalXML node='MultiRun' xmlToLoad='external_input/MultiRun.xml'/>
    ...
  </Steps>
  ...
</Simulation>
\end{lstlisting}
