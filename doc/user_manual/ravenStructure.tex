\section{Raven Input Structure}
The RAVEN code does not have a fixed calculation flow, since all of its basic
objects can be combined in order to create a user-defined calculation flow.
%
Thus, its input (XML format) is organized in different XML blocks, each with a
different functionality.
%
The main input blocks are as follows:
\begin{itemize}
  \item \textbf{\textless Simulation\textgreater}: The root node containing the
  entire input, all of
  the following blocks fit inside the \emph{Simulation} block.
  %
  \item \textbf{\textless RunInfo\textgreater}: Specifies the calculation
  settings (number of parallel simulations, etc.).
  %
  \item \textbf{\textless Distributions\textgreater}: Defines distributions
  needed for describing parameters, etc.
  %
  \item \textbf{\textless Samplers\textgreater}: Sets up the strategies used for
  exploring an uncertain domain.
  %
  \item \textbf{\textless Functions\textgreater}: Details interfaces to external
  user-defined functions and modules.
  %
  \item \textbf{\textless Models\textgreater}: Specifies codes, ROMs,
  post-processing analysis, etc.
  %
  the user will be building and/or running.
  %
  \item \textbf{\textless Steps\textgreater}: Combines other blocks to detail a
  step in the RAVEN workflow including I/O and computations to be performed.
  %
  \item \textbf{\textless Data\textgreater}: Specifies internal data objects
  used by RAVEN.
  %
  \item \textbf{\textless Databases\textgreater}: Lists the HDF5 databases used
  as input/output to a
  RAVEN run.
  %
  \item \textbf{\textless OutStreamManager\textgreater}: Visualization and
  Printing system block.
  %
\end{itemize}

Each of these blocks are explained in dedicated sections in the following
chapters.
%
Each block within RAVEN also makes use of a \xmlAttr{verbosity} system,
which allows a user to control the level of output to the user interface.
These settings are declared globally as attributes in the \xmlNode{Simulation} node,
and locally in each block.  The verbosity levels are
\begin{itemize}
\item \xmlString{silent} - Only simulation-breaking errors are displayed.
\item \xmlString{quiet} - Errors as well as warnings are displayed.
\item \xmlString{all} (default) - Errors, warnings, and messages are displayed.
\item \xmlString{debug} - For developers. All errors, warnings, messages, and debug messages are displayed.
\end{itemize}
Examples of verbosity usage are included in many examples throughout this manual.