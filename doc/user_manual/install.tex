\section{Install}

\subsection{Framework Source Install}

Install the dependencies.  Either they can be installed manually or the script can be used.  Other versions may work.

\begin{enumerate}
\item numpy-1.7.0
\item hdf5-1.8.12
\item Cython-0.18
\item h5py-2.2.1
\item scipy-0.12.0
\item scikit-learn-0.14.1
\item matplotlib-1.4.0
\end{enumerate}

Untar the source install (if there is more than one version of the
source tarball, the full filename will need to be used instead of *):

\begin{lstlisting}[language=bash]
tar -xvzf raven_framework_*_source.tar.gz
\end{lstlisting}

If the dependencies were not installed previously, they can be installed by
running the \verb'raven_libs_script.sh':

\begin{lstlisting}[language=bash]
cd trunk/raven/
./raven_libs_script.sh
cd ../../
\end{lstlisting}

Crow needs to be compiled.  This can be done either with the python
setup or with a makefile.  Using the setup.py file:

\begin{lstlisting}[language=bash]
cd trunk/crow/
python setup.py build_ext build install --user
\end{lstlisting}

Using the makefile:

\begin{lstlisting}[language=bash]
cd trunk/crow/
make -f Makefile.linux
\end{lstlisting}

Next, the tests should be run (you may need to change the \verb'cd'
command if you are not in the crow directory):

\begin{lstlisting}[language=bash]
cd ../raven/
./run_tests --re=framework --skip-config-check
\end{lstlisting}

There should be a line something like:
{\tt 8 passed, 19 skipped, 0 pending, 0 failed}
at the end.  If any failed, look at the output to see why.


\subsection{Ubuntu Framework Install}

Install the dependencies:

\begin{lstlisting}[language=bash]
sudo apt-get install libtool python-dev swig g++ python3-dev \
 python-numpy python-sklearn python-h5py
\end{lstlisting}

Untar the binary install (if there is more than one version of the
binary install, the full filename will need to be used instead of *):

\begin{lstlisting}[language=bash]
tar -xvzf raven_framework_*_ubuntu.tar.gz
\end{lstlisting}

Run the tests:

\begin{lstlisting}[language=bash]
cd trunk/raven/
./run_tests --re=framework --skip-config-check
\end{lstlisting}

There should be a line something like:
{\tt 8 passed, 19 skipped, 0 pending, 0 failed}
at the end.  If any failed, look at the output to see why.

\subsection{Fedora Framework Install}

Install the dependencies:

\begin{lstlisting}[language=bash]
yum install swig libtool gcc-c++ python-devel python3-devel \
 numpy h5py scipy python-scikit-learn python-matplotlib-qt4
\end{lstlisting}

If you want to be able to edit and rebuild the manual, install:
\begin{lstlisting}[language=bash]
yum install texlive texlive-subfigure texlive-stmaryrd
\end{lstlisting}

Untar the binary install (if there is more than one version of the
binary install, the full filename will need to be used instead of *):

\begin{lstlisting}[language=bash]
tar -xvzf raven_framework_*_fedora.tar.gz
\end{lstlisting}

Run the tests:

\begin{lstlisting}[language=bash]
cd trunk/raven/
./run_tests --re=framework --skip-config-check
\end{lstlisting}

There should be a line something like:
{\tt 8 passed, 19 skipped, 0 pending, 0 failed}
at the end.  If any failed, look at the output to see why.

\subsection{OSX Framework Install}

Open up the file \verb'raven_libs_framework_and_crow.dmg' Open up the
\verb'raven_libs.pkg' inside, and install it.  The files will be
installed into \verb'/opt/raven_libs' They will edit your
\verb'.bash_profile' to source the
\verb'/opt/raven_libs/environments/raven_libs_profile' file.  This
file sets up the \verb'PYTHONPATH' and the \verb'PATH' so that the
\verb'raven_framework' command can be used.

\subsection{Moose and Raven Source Install}

First, moose should be installed.  Follow the instructions for moose:
\url{http://mooseframework.org/getting-started/}

Next, if the C++ Raven is desired, RELAP-7 needs to be installed.
Follow the RELAP-7 instructions, but moose needs to be installed
in the same directory level as RELAP-7.

Then clone crow and raven:

\begin{lstlisting}[language=bash]
git clone git@hpcgitlab.inl.gov:idaholab/crow.git
git clone git@hpcgitlab.inl.gov:idaholab/raven.git
\end{lstlisting}

Install the raven dependencies via one of the methods mentioned for
the raven framework.

Then compile raven:

\begin{lstlisting}[language=bash]
cd raven
make
\end{lstlisting}

Then run the tests:

\begin{lstlisting}[language=bash]
./run_tests
\end{lstlisting}

There will be a line telling how many passed.  If any failed, look at
the output to see why.
