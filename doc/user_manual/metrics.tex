\section{Metrics}
\label{sec:Metrics}

The Metrics block allows the user to specify the similarity/dissimilarity metrics to be used in specific clustering algorithms available in RAVEN.
These metrics are used to calculate the distance values among points and histories.
The data mining algorithms in RAVEN which accept the definition of a metric are the following:
\begin{itemize}
  \item DBSCAN (see Section~\ref{subparagraph:DBSCAN})
  \item Affinity Propagation (see Section~\ref{subparagraph:Affinity})
\end{itemize}
Both of these algorithms can take as input an $N \times N$ square matrix $D=[d_{ij}]$ where each element $d_{ij}$ of $D$ is the distance between 
element $i$ $(i=1,\ldots,N)$ and $j$
$(j=1,\ldots,N)$.

Available metrics are as follows:
\begin{itemize}
  \item Minkowski (see Section~\ref{subsection:Minkowski})
\end{itemize}

In the RAVEN input file these metrics are defined as follows:
\begin{lstlisting}[style=XML]
<Simulation>
  ...
  <Metrics>
    ...
    <MetricID name='metricName'>
      ...
     <param1>value</param1>
      ...
    </MetricID>
    ...
  </Metrics>
  ...
</Simulation>
\end{lstlisting}

\subsection{Minkowski}
\label{subsection:Minkowski}
Minkowski distance is the most basic and used metric in any data mining algorithm.
Given two multi-dimensional vectors $X=(x_1,x_2,\ldots,x_n)$ and $Y=(y_1,y_2,\ldots,y_n)$, the Minkowski distance $d_p$ of order $p$ between these 
two vectors is:
\begin{equation}
d_p = \left ( \sum_{i=1}^{n} \left \| x_i-y_i \right \|^p \right )^\frac{1}{p}
\end{equation}
Minkowski distance is typically used with $p$ being 1 or 2. The latter one is the Euclidean distance, while the former is sometimes known as the 
Manhattan distance. 
Note that this metric can be employed for both PointSets and HistorySets.

Note that this metric can be employed directly on HistorySets.
For HistorySets, all histories must contain an equal number of samples. Note that, in this respect, the interfaced Post-Processor (see Section~\ref{Interfaced}) 
HistorySetSampling can be employed to perform such action. In addition, several more interfaced Post-Processors can be used to manipulate HistorySets.

The specifications of a Minkowski distance must be defined within the XML block
\xmlNode{Minkowski}.
%
This XML node needs to contain the attributes:

\begin{itemize}
  \item \xmlNode{p}, \xmlDesc{float, required field}, value for the parameter $p$
  \item \xmlNode{pivotParameter}, \xmlDesc{string, optional field}, the ID of the temporal variable; this is required in case the metric is applied to historysets instead of pointsets
\end{itemize}

An example of Minkowski distance defined in RAVEN is provided below:
\begin{lstlisting}[style=XML]
<Simulation> 
  ...
  <Metrics>
    ...
    <Minkowski name="example" subType="">
      <p>2</p>
      <pivotParameter>time</pivotParameter>
    </Minkowski>
    ...
  </Metrics>
  ...
</Simulation>
\end{lstlisting}


\subsection{Dynamic Time Warping}
\label{subsection:DTW}
The Dynamic Time Warping (DTW) is a distanc metrice that can be employed only for HistorySets (i.e., time dependent data).

The specifications of a DTW distance must be defined within the XML block.
\xmlNode{DTW}.

This XML node needs to contain the attributes:


\begin{itemize}
  \item \xmlNode{order},          \xmlDesc{int, required field},    order of the DTW calculation: $0$ specifices a classical DTW caluclation and $1$ specifies 
                                                                    a derivative DTW calculation 
  \item \xmlNode{pivotParameter}, \xmlDesc{string, optional field}, the ID of the temporal variable
  \item \xmlNode{localDistance},  \xmlDesc{string, optional field}, the ID of the distance function to be employed to determine the local distance 
                                                                    evaluation of two time series. Available options are provided by the sklearn 
                                                                    pairwise\_distances (cityblock, cosine, euclidean, $l1$, $l2$, manhattan,
                                                                    braycurtis, canberra, chebyshev, correlation, dice, hamming, jaccard, 
                                                                    kulsinski, mahalanobis, matching, minkowski, rogerstanimoto, russellrao, 
                                                                    seuclidean, sokalmichener, sokalsneath, sqeuclidean, yule)
\end{itemize}

An example of Minkowski distance defined in RAVEN is provided below:
\begin{lstlisting}[style=XML]
<Simulation> 
  ...
  <Metrics>
    ...
    <DTW name="example" subType="">
      <order>0</order>
      <pivotParameter>time</pivotParameter>
      <localDistance>euclidean</localDistance>
    </DTW>
    ...
  </Metrics>
  ...
</Simulation>
\end{lstlisting}

\subsection{SKL Metrics}
\label{subsection:SKL_metrics}

This Metric class interfaces directly with the metric distances available within scikit-learn.
Note that these distance metrics apply only to PointSets. However, note that this distance metrics can be applied to HistorySets after the HistorySets are converted into PointSets.
In this respect, the HSPS interfaced Post-Processor (see Section~\ref{Interfaced}) can be employed to perform such conversion.

The specifications of a SKL metric must be defined within the XML block.
\xmlNode{SKL}.

Available metrics from SKL are:
\begin{itemize}  
  \item   From scikit-learn pairwise kernels:
    \begin{itemize}
       \item rbf: $k(x, y) = \exp( -\gamma \| x-y \|^2)$, parameters required: gamma  
       \item sigmoid: $k(x, y) = \tanh( \gamma x^\top y + c_0)$, parameters required: gamma and coef0 
       \item polynomial: $k(x, y) = (\gamma x^\top y +c_0)^d$, parameters required: gamma, degree, coef0
       \item laplacian: $k(x, y) = \exp( -\gamma \| x-y \|_1)$, parameters required: gamma 
       \item linear: $k(x, y) = x^\top y$ 
       \item cosine: $k(x, y) = \frac{x y^\top}{\|x\| \|y\|}$
       \item $chi2$: $k(x, y) = \exp \left (-\gamma \sum_i \frac{(x[i] - y[i]) ^ 2}{x[i] + y[i]} \right )$ 
       \item $additive_chi2$: $k(x, y) = -\sum [(x - y)^2 / (x + y)]$
    \end{itemize}
  
  \item From scipy.spatial.distance
    \begin{itemize}
      \item euclidean: ${||u-v||}_2$
      \item cosine: $1 - \frac{u \cdot v}{||u||_2 ||v||_2}$ where $u \cdot v$ is the dot product of $u$ and $v$
      \item cityblock: $\sum_i {\left| u_i - v_i \right|}$
      \item l1: see cityblock
      \item l2: see euclidean
      \item manhattan: see cityblock
     
%       These are distance metrics available with newer version of sklearn (from 0.18.1)
%       \item braycurtis: $\sum{|u_i-v_i|} / \sum{|u_i+v_i|}$ 
%       \item canberra: $ d(u,v) = \sum_i \frac{|u_i-v_i|}{|u_i|+|v_i|}$ 
%       \item chebyshev: $\max_i {|u_i-v_i|}$       
%       \item correlation: $1 - \frac{(u - \bar{u}) \cdot (v - \bar{v})}{{||(u - \bar{u})||}_2 {||(v - \bar{v})||}_2}$ where $\bar{u}$ is the mean of the elements of $u$
%       and $x \cdot y$ is the dot product of $x$ and $y$
%       \item dice: $\frac{c_{TF} + c_{FT}}{2c_{TT} + c_{FT} + c_{TF}}$  where $c_{ij}$ is the number of occurrences of $\mathtt{u[k]} = i$ and $\mathtt{v[k]} = j$ for $k < n$
%       \item hamming: $\frac{c_{01} + c_{10}}{n}$  where $c_{ij}$ is the number of occurrences of $\mathtt{u[k]} = i$ and $\mathtt{v[k]} = j$ for $k < n$
%       \item jaccard: $\frac{c_{TF} + c_{FT}}{c_{TT} + c_{FT} + c_{TF}}$  where $c_{ij}$ is the number of occurrences of $\mathtt{u[k]} = i$ and $\mathtt{v[k]} = j$ for $k < n$
%       \item kulsinski $\frac{c_{TF} + c_{FT} - c_{TT} + n}{c_{FT} + c_{TF} + n}$ where $c_{ij}$ is the number of occurrences of $\mathtt{u[k]} = i$ and $\mathtt{v[k]} = j$ for $k < n$
%       \item mahalanobis: $\sqrt{ (u-v) V^{-1} (u-v)^T }$ where $V$ is the covariance matrix.  Note that the argument $VI$ is the inverse of $V$ 
%       \item minkowski: ${||u-v||}_p = (\sum{|u_i - v_i|^p})^{1/p}$, parameters required: p
%       \item wminkowski: $\left(\sum{(|w_i (u_i - v_i)|^p)}\right)^{1/p}$, paramteres required: p
%       \item rogerstanimoto: $\frac{R}{c_{TT} + c_{FF} + R}$ where $c_{ij}$ is the number of occurrences of $\mathtt{u[k]} = i$ and $\mathtt{v[k]} = j$ 
%       for $k < n$ and $R = 2(c_{TF} + c_{FT})$
%       \item russellrao: $\frac{n - c_{TT}}{n}$  where $c_{ij}$ is the number of occurrences of $\mathtt{u[k]} = i$ and $\mathtt{v[k]} = j$ for $k < n$
%       \item seuclidean: $\sqrt{\sum {(u_i-v_i)^2 / V[x_i]}}$ where $V$ is the variance vector; $V[i]$ is the variance computed over all the i'th components of the points. 
%        If not passed, it is automatically computed.
%       \item sokalmichener: $\frac{R}{S + R}$ where $c_{ij}$ is the number of occurrences of $\mathtt{u[k]} = i$ and $\mathtt{v[k]} = j$ for
%       $k < n$, $R = 2 * (c_{TF} + c_{FT})$ and $S = c_{FF} + c_{TT}$
%       \item sokalsneath: $\frac{R}{c_{TT} + R}$ where $c_{ij}$ is the number of occurrences of $\mathtt{u[k]} = i$ and $\mathtt{v[k]} = j$ 
%       for $k < n$ and $R = 2(c_{TF} + c_{FT})$
%       \item sqeuclidean: ${||u-v||}_2^2$
%       \item yule: $\frac{R}{c_{TT} * c_{FF} + \frac{R}{2}}$ where $c_{ij}$ is the number of occurrences of $\mathtt{u[k]} = i$ and $\mathtt{v[k]} = j$ for
%       $k < n$ and $R = 2.0 * c_{TF} * c_{FT}$
    \end{itemize}
\end{itemize}

