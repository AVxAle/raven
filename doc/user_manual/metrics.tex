\section{Metrics}
\label{sec:Metrics}

The Metrics block allows the user to specify the similarity/dissimilarity metrics to be used in specific clustering algorithms available in RAVEN.
These metrics are used to calculate the distance values among points and histories.
The data mining algorithms in RAVEN which accept the definition of a metric are the following:
\begin{itemize}
  \item DBSCAN (see Section~\ref{subparagraph:DBSCAN})
  \item Affinity Propagation (see Section~\ref{subparagraph:Affinity})
\end{itemize}
Both of these algorithms can take as input an $N \times N$ square matrix $D=[d_{ij}]$ where each element $d_{ij}$ of $D$ is the distance between 
element $i$ $(i=1,\ldots,N)$ and $j$
$(j=1,\ldots,N)$.

Available metrics are as follows:
\begin{itemize}
  \item Minkowski (see Section~\ref{subsection:Minkowski})
\end{itemize}

In the RAVEN input file these metrics are defined as follows:
\begin{lstlisting}[style=XML]
<Simulation>
  ...
  <Metrics>
    ...
    <MetricID name='metricName'>
      ...
     <param1>value</param1>
      ...
    </MetricID>
    ...
  </Metrics>
  ...
</Simulation>
\end{lstlisting}

\subsection{Minkowski}
\label{subsection:Minkowski}
Minkowski distance is the most basic and used metric in any data mining algorithm.
Given two multi-dimensional vectors $X=(x_1,x_2,\ldots,x_n)$ and $Y=(y_1,y_2,\ldots,y_n)$, the Minkowski distance $d_p$ of order $p$ between these 
two vectors is:
\begin{equation}
d_p = \left ( \sum_{i=1}^{n} \left \| x_i-y_i \right \|^p \right )^\frac{1}{p}
\end{equation}
Minkowski distance is typically used with $p$ being 1 or 2. The latter one is the Euclidean distance, while the former is sometimes known as the 
Manhattan distance. 
Note that this metric can be employed for both PointSets and HistorySets.

Note that for HistorySets, all histories must contain an equal number of samples.

The specifications of a Minkowski distance must be defined within the XML block
\xmlNode{Minkowski}.
%
This XML node needs to contain the attributes:

\begin{itemize}
  \item \xmlNode{p}, \xmlDesc{float, required field}, value for the parameter $p$
  \item \xmlNode{pivotParameter}, \xmlDesc{string, optional field}, the ID of the temporal variable; this is required in case the metric is applied to historysets instead of pointsets
\end{itemize}

An example of Minkowski distance defined in RAVEN is provided below:
\begin{lstlisting}[style=XML]
<Simulation> 
  ...
  <Metrics>
    ...
    <Minkowski name="example" subType="">
      <p>2</p>
      <pivotParameter>time</pivotParameter>
    </Minkowski>
    ...
  </Metrics>
  ...
</Simulation>
\end{lstlisting}


\subsection{Dynamic Time Warping}
\label{subsection:DTW}
The Dynamic Time Warping (DTW) is a distanc metrice that can be employed only for HistorySets (i.e., time dependent data).

The specifications of a DTW distance must be defined within the XML block.
\xmlNode{DTW}.

This XML node needs to contain the attributes:


\begin{itemize}
  \item \xmlNode{order},          \xmlDesc{int, required field},    order of the DTW calculation: $0$ specifices a classical DTW caluclation and $1$ specifies 
                                                                    a derivative DTW calculation 
  \item \xmlNode{pivotParameter}, \xmlDesc{string, optional field}, the ID of the temporal variable
  \item \xmlNode{localDistance},  \xmlDesc{string, optional field}, the ID of the distance function to be employed to determine the local distance 
                                                                    evaluation of two time series. Available options are provided by the sklearn 
                                                                    pairwise\_distances (cityblock, cosine, euclidean, $l1$, $l2$, manhattan,
                                                                    braycurtis, canberra, chebyshev, correlation, dice, hamming, jaccard, 
                                                                    kulsinski, mahalanobis, matching, minkowski, rogerstanimoto, russellrao, 
                                                                    seuclidean, sokalmichener, sokalsneath, sqeuclidean, yule)
\end{itemize}

An example of Minkowski distance defined in RAVEN is provided below:
\begin{lstlisting}[style=XML]
<Simulation> 
  ...
  <Metrics>
    ...
    <DTW name="example" subType="">
      <order>0</order>
      <pivotParameter>time</pivotParameter>
      <localDistance>euclidean</localDistance>
    </DTW>
    ...
  </Metrics>
  ...
</Simulation>
\end{lstlisting}
