\section{Distributions \\ \vspace{2 mm} {\small Author: Andrea Alfonsi}}
\label{sec:distributions}
\newcommand{\distname}[1]{\textbf{#1}}
\newcommand{\distattrib}[1]{\textit{#1}}
RAVEN provides support for several probability distributions. Currently, the user can choose among all the most important 1-Dimensional distributions and N-Dimensional ones, either custom or Multi-Variate.  
\\ The user needs to specify the probability distributions, that need to be used during the simulation, within the $<Distributions>$ xml block:
\begin{lstlisting}[style=XML]
<Simulation>
   ...
  <Distributions>
    <!-- here all the distributions, that need to be used, are listed -->
  </Distributions>
  ...
</Simulation>
\end{lstlisting}
In the following two sub-sections, the input requirements for all of them are reported.
%%%%%% 1-Dimensional Probability distributions
\subsection{1-Dimensional Probability Distributions}
\label{subsec:1dDist}

In this sub-section, all the 1-D distributions', currently available in RAVEN, are reported. Firstly, all the probability distributions functions in the code can be truncated by the following keywords:
\begin{lstlisting}[style=XML]
<lowerBound>***</lowerBound>
<upperBound>***</upperBound>
\end{lstlisting}
Obviously, each distribution already defines its validity domain (e.g. Normal distribution, [-inf,+inf]).
\\ RAVEN currently provides support for 12 1-Dimensional distributions.  In the following paragraphs, all the input requirements are reported and commented.

%%%%%% Bernoulli
\subsubsection{Bernoulli Distribution}
\label{subsubsec:Bernoulli}
The \distname{Bernoulli} distribution is a discrete distribution of the outcome of a single trial with only two results, 0 (failure) or 1 (success), with a probability of success \distattrib{p}. It is the simplest building block on which other discrete distributions of sequences of independent Bernoulli trials can be based. Basically, it is the binomial distribution (k = 1, \distattrib{p}) with only one trial.
\\ The specifications of this distribution must be defined within the xml block $<Bernoulli>$. This xml-node needs to contain the attribute:
\vspace{-5mm}
\begin{itemize}
\itemsep0em
\item \textbf{name}, \textit{required integer attribute}, Name of this distribution. As for the other objects, this is the name that can be used to refer to this specific entity in other input blocks (xml).   
\end{itemize}
\vspace{-5mm}
This distribution can be initialized through the following keyword/s:
\begin{itemize}
\item $<p>$, float, required parameter, probability of success.
 \end{itemize}
\begin{lstlisting}[style=XML]
----------------------------
Example:
----------------------------
<Distributions>
  ...
  <Bernoulli name='...'>
    <p>***</p>
  </Bernoulli>
  ...
</Distributions>
----------------------------
\end{lstlisting}

%%%%%% Beta
\subsubsection{Beta Distribution}
\label{subsubsec:Beta}
The \distname{Beta} distribution is a continuous distribution  defined on the interval $[0, 1]$ parametrized by two positive shape parameters, denoted by $\alpha$ and $\beta$, that appear as exponents of the random variable and control the shape of the distribution. The distribution domain  can be changed,specifying new boundaries, to fit the user needs.
\\ The specifications of this distribution must be defined within the xml block $<Beta>$. This xml-node needs to contain the attribute:
\vspace{-5mm}
\begin{itemize}
\itemsep0em
\item \textbf{name}, \textit{required string attribute}, user-defined name of this distribution. N.B. As for the other objects, this is the name that can be used to refer to this specific entity from other input blocks (xml).   
\end{itemize}
\vspace{-5mm}
This distribution can be initialized through the following keyword/s:
\begin{itemize}
\item $<alpha>$, float, required parameter, first shape parameter;
\item $<beta>$, float, required parameter, second shape parameter;
\item $<low>$, float, optional parameter,  lower domain boundary;  
\item $<high>$, float, required parameter, upper domain boundary.
\end{itemize}

\begin{lstlisting}[style=XML]
----------------------------
Example:
----------------------------
<Distributions>
  ...
  <Beta name='...'>
     <low>***</low>
     <high>***</high>
     <alpha>***</alpha>
     <beta>***</beta>
  </Beta>
  ...
</Distributions>
----------------------------
\end{lstlisting}

%%%%%% Binomial
\subsubsection{Binomial Distribution}
\label{subsubsec:Binomial}
The \distname{Binomial} distribution is the discrete probability distribution of the number of successes in a sequence of \distattrib{n} independent yes/no experiments, each of which yields success with probability \distattrib{p}. 
\\ The specifications of this distribution must be defined within the xml block $<Binomial>$. This xml-node needs to contain the attribute:
\vspace{-5mm}
\begin{itemize}
\itemsep0em
\item \textbf{name}, \textit{required string attribute}, user-defined name of this distribution. N.B. As for the other objects, this is the name that can be used to refer to this specific entity from other input blocks (xml).   
\end{itemize}
\vspace{-5mm}
This distribution can be initialized through the following keyword/s:
\begin{itemize}
\item $<p>$, float, required parameter,  probability of success;
\item $<n>$, integer, required parameter, number of experiment.
\end{itemize}

\begin{lstlisting}[style=XML]
----------------------------
Example:
----------------------------
<Distributions>
  ...
  <Binomial name='...'>
    <n>***</n>
    <p>***</p>
  </Binomial>
  ...
</Distributions>
----------------------------
\end{lstlisting}

%%%%%% Binomial
\subsubsection{Exponential Distribution}
\label{subsubsec:Exponential}
The \distname{Exponential} distribution is a continuous distribution that can be used to model the time between independent events that
happen at a constant average time. 
\\ The specifications of this distribution must be defined within the xml block $<Exponential>$. This xml-node needs to contain the attribute:
\vspace{-5mm}
\begin{itemize}
\itemsep0em
\item \textbf{name}, \textit{required string attribute}, user-defined name of this distribution. N.B. As for the other objects, this is the name that can be used to refer to this specific entity from other input blocks (xml).   
\end{itemize}
\vspace{-5mm}
This distribution can be initialized through the following keyword/s:
\begin{itemize}
\item $<lambda>$, float, required parameter,  rate parameter.
\end{itemize}

\begin{lstlisting}[style=XML]
----------------------------
Example:
----------------------------
<Distributions>
  ...
  <Exponential name='...'>
    <lambda>***</lambda>
  </Exponential>
  ...
</Distributions>
----------------------------
\end{lstlisting}



%logistic - BasicLogisticDistribution - continuous
%location
%scale

 The \distname{Gamma} distribution is a continuous distribution.  It
has three parameters, \distattrib{low} for the lowest value,
\distattrib{alpha} is the shape parameter, and \distattrib{beta} which
is 1/scale or the inverse scale parameter.

\begin{lstlisting}[style=XML]
<Gamma name='...'>
<low>***</low>
<alpha>***</alpha>
<beta>***</beta>
\end{lstlisting}


The \distname{Logistic} distribution is a continuous distribution
similar to the normal distribution with a CDF that is an instance of a
logistic function.  It has two parameters, \distattrib{location} with
is the most common value and the center, and \distattrib{scale} which
determines the shape.

\begin{lstlisting}[style=XML]
<Logistic name='...'>
<location>***</location>
<scale>***</scale>
\end{lstlisting}

%log-normal - BasicLogNormalDistribution - continuous
%mean
%sigma

The \distname{LogNormal} distribution is a continuous distribution
with the logarithm of the random variable being normally distributed.
It has two parameters, \distattrib{mean} which is the expected value,
and \distattrib{sigma} which is the standard deviation.

\begin{lstlisting}[style=XML]
<LogNormal name='...'>
<mean>***</mean>
<sigma>***</sigma>
\end{lstlisting}

%normal - BasicNormalDistribution - continuous
%mean
%sigma

The \distname{Normal} distribution (or Gaussian) distribution is a
continuous distribution which because of the central limit theorem,
the mean of many distributions approximates a normal distribution.  It
has two parameters, \distattrib{mean} which is the middle value, and
\distattrib{sigma} which is the standard deviation.

\begin{lstlisting}[style=XML]
<Normal name='...'>
<mean>***</mean>
<sigma>***</sigma>
\end{lstlisting}


%Poisson - BasicPoissonDistribution - discrete
%mu

The \distname{Poisson} distribution is a discrete distribution that
expresses the probability of the number of events occurring in a fixed
period of time.  It has one parameter, \distattrib{mu} the mean rate
of events/time.

\begin{lstlisting}[style=XML]
<Poisson name='...'>
<mu>***</mu>
\end{lstlisting}

%triangular - BasicTriangularDistribution - continuous
%apex
%min
%max

The \distname{Triangular} distribution is a continuous distribution
that has a triangular shape for the PDF.  The peak falls at the
\distattrib{apex} and the values run from \distattrib{min} to
\distattrib{max}.

\begin{lstlisting}[style=XML]
<Triangular name='...'>
<apex>***</apex>
<min>***</min>
<max>***</max>
\end{lstlisting}


%uniform - BasicUniformDistribution - continuous 
%low
%high

The \distname{Uniform} distribution is a continuous distribution with
a rectangular shaped PDF.  It goes from \distattrib{low} to
\distattrib{high}.

\begin{lstlisting}[style=XML]
<Uniform name='...'>
<low>***</low>
<hi>***</hi>
\end{lstlisting}

%weibull - BasicWeibullDistribution - continuous
%k
%lambda

The \distname{Weibull} distribution is a continuous distribution that
can be used is failure analysis.  It takes two parameters,
\distattrib{k} or the shape parameter, and \distattrib{lambda} or the
scale parameter.

\begin{lstlisting}[style=XML]
<Weibull name='...'>
<lambda>***</lambda>
<k>***</k>
\end{lstlisting}

%gamma - BasicGammaDistribution - continuous 
%low
%alpha
%beta



%beta - BasicBetaDistribution - continuous
%low
%high
%alpha
%beta


