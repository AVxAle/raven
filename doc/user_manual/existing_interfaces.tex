\section{Existing Interfaces}
\label{sec:existingInterface}

\subsection{Generic Interface}
The GenericCode interface is meant to handle a wide variety of generic codes 
that take take straightforward input files and produce output CSV files.  There are
some limitations for this interface.  
If a code: \vspace{-20pt}
\begin{itemize}
\item accepts a keyword-based input file with no cross-dependent inputs,
\item has no more than one filetype extension per command line flag,
\item and returns a CSV with the input parameters and output parameters,
\end{itemize}\vspace{-20pt}
the GenericCode interface should cover the code for RAVEN.

If a code contains cross-dependent data, the generic interface is not able to
edit the correct values.  For example, if a geometry-building script specifies
inner\_radius, outer\_radius, and thickness, the generic interface cannot 
calculate the thickness given the outer and inner radius, or vice versa.

 An example of the code interface is shown here.  The input parameters are read 
 from the input files \texttt{gen.one} and \texttt{gen.two} respectively.
 The code is run using \texttt{python}, so that is part of the \xmlNode{prepend} node.
 The command line entry to normally run the code is
\begin{lstlisting}[language=bash]
python poly_inp.py -i gen.one -a gen.two -o myOut
\end{lstlisting}
and produces the output \texttt{myOut.csv}.

Example:
\begin{lstlisting}[style=XML]
    <Code name="poly" subType="GenericCode">
      <executable>GenericInterface/poly_inp.py</executable>
      <inputExtentions>.one,.two</inputExtentions>
      <clargs type='prepend' arg='python'/>
      <clargs type='input'   arg='-i' extension='.one'/>
      <clargs type='input'   arg='-a' extension='.two'/>
      <clargs type='output'  arg='-o'/>
      <prepend>python</prepend>
    </Code>
\end{lstlisting}

If a code doesn't accept necessary Raven-editable auxiliary input files 
or output filenames through the command line, the GenericCode interface
can also edit the input files and insert the filenames there.  For example,
in the previous example, say instead of \texttt{-a gen.two} and \texttt{-o myOut}
in the command line, \texttt{gen.one} has the following lines:
\begin{lstlisting}[language=python]
...
auxfile = gen.two
case = myOut
...
\end{lstlisting}
Then, our example XML for the code would be

Example:
\begin{lstlisting}[style=XML]
    <Code name="poly" subType="GenericCode">
      <executable>GenericInterface/poly_inp.py</executable>
      <inputExtentions>.one,.two</inputExtentions>
      <clargs   type='prepend' arg='python'/>
      <clargs   type='input'   arg='-i'  extension='.one'/>
      <fileargs type='input'   arg='two' extension='.two'/>
      <fileargs type='output'  arg='out'/>
      <prepend>python</prepend>
    </Code>
\end{lstlisting}
and the corresponding template input file lines would be changed to read
\begin{lstlisting}[language=python]
...
auxfile = $RAVEN-two$
case = $RAVEN-out$
...
\end{lstlisting}

\subsection{RELAP5 Interface}

\subsubsection{Files}
In the \xmlNode{Files} section, as specified before, all of the files needed for
the code to run should be specified.
%
In the case of RELAP5, the files typically needed are:
\begin{itemize}
  \item RELAP5 Input file
  \item Table file or files that RELAP needs to run
  \item The Relap5 executable
\end{itemize}
Example:
\begin{lstlisting}[style=XML]
<Files>X10.i,tpfh2o,r5executable.x</Files>
\end{lstlisting}

It is a good practice to put inside the working directory all of these files and
also:
\begin{itemize}
  \item the RAVEN input file
  \item the license for the executable of RELAP5
\end{itemize}
%%%%%%%%%%%%%%%%%%%%%%%%%%%%%%%%%%%%%%%%%%%%%%%%%%%%

\subsubsection{Sequence}
In the \xmlNode{Sequence} section, the names of the steps declared in the
\xmlNode{Steps} block should be specified.
%
As an example, if we called the first multirun ``Grid\_Sampler'' and the second
multirun ``MC\_Sampler'' in the sequence section we should see this:
\begin{lstlisting}[style=XML]
<Sequence>Grid_Sampler,MC_Sampler</Sequence>
\end{lstlisting}
%%%%%%%%%%%%%%%%%%%%%%%%%%%%%%%%%%%%%%%%%%%%%%%%%%%

\subsubsection{batchSize and mode}
For the \xmlNode{batchSize} and \xmlNode{mode} sections please refer to the
\xmlNode{RunInfo} block in the previous chapters.
%
%%%%%%%%%%%%%%%%%%%%%%%%%%%%%%%%%%%%%%%%%%%%%%%%%%%%
\subsubsection{RunInfo}
After all of these blocks are filled out, a standard example RunInfo block may
look like the example below:
\begin{lstlisting}[style=XML]
<RunInfo>
  <WorkingDir>~/working_dir</WorkingDir>
  <Files>inputfilerelap.i,tpfh2o,executablerelap5.x</Files>
  <Sequence>Grid_Sampler,MC_Sampler</Sequence>
  <batchSize>1</batchSize>
  <mode>mpi</mode>
  <expectedTime>1:00:00</expectedTime>
  <ParallelProcNumb>1</ParallelProcNumb>
</RunInfo>
\end{lstlisting}
%%%%%%%%%%%%%%%%%%%%%%%%%%%%%%%%%%%%%%%%%%%%%%%%%%%%%%%%%%%
\subsubsection{Models}
For the \xmlNode{Models} block here is a standard example of how it would look
when using RELAP5 as the external model:
\begin{lstlisting}[style=XML]
<Models>
  <Code name='MyRELAP' subType='Relap5'>
    <executable>~/path_to_the_executable</executable>
  </Code>
</Models>
\end{lstlisting}
%%%%%%%%%%%%%%%%%%%%%%%%%%%%%%%%%%%%%%%%%%%%%%%%%%%%%%%%%
\subsubsection{Distributions}
The \xmlNode{Distribution} block defines the distributions that are going
to be used for the sampling of the variables defined in the \xmlNode{Samplers}
block.
%
For all the possibile distributions and all their possible inputs please see the
chapter about Distributions (see~\ref{sec:distributions}).
%
Here we give a general example of three different distributions:
\begin{lstlisting}[style=XML,morekeywords={name,debug}]
<Distributions verbosity='debug'>
  <Triangular name='BPfailtime'>
    <apex>5.0</apex>
    <min>4.0</min>
  <max>6.0</max>
  </Triangular>
  <LogNormal name='BPrepairtime'>
    <mean>0.75</mean>
    <sigma>0.25</sigma>
  </LogNormal>
  <Uniform name='ScalFactPower'>
    <low>1.0</low>
    <hi>1.2</hi>
  </Uniform>
 </Distributions>
\end{lstlisting}

It is good practice to name the distribution something similar to what kind of
variable is going to be sampled, since there might be many variables with the
same kind of distributions but different input parameters.
%
%%%%%%%%%%%%%%%%%%%%%%%%%%%%%%%%%%%%%%%%%%%%%%%%%%%%%%%%%
\subsubsection{Samplers}
In the \xmlNode{Samplers} block we want to define the variables that are going
to be sampled.
%
\textbf{Example}:
We want to do the sampling of 3 variables:
\begin{itemize}
  \item Battery Fail Time
  \item Battery Repair Time
  \item Scaling Factor Power Rate
\end{itemize}

We are going to sample these 3 variables using two different sampling methods:
grid and MonteCarlo.

In RELAP5, the sampler reads the variable as, given the name, the first number
is the card number and the second number is the word number.
%
In this example we are sampling:
\begin{itemize}
  \item For card 0000588 (trip) the word 6 (battery failure time)
  \item For card 0000575 (trip) the word 6 (battery repari time)
  \item For card 20210000 (reactor power) the word 4 (reactor scaling factor)
\end{itemize}

We proceed to do so for both the Grid sampling and the MonteCarlo sampling.

\begin{lstlisting}[style=XML,morekeywords={name,type,construction,lowerBound,steps,limit,initial_seed}]
<Samplers verbosity='debug'>
  <Grid name='Grid_Sampler' initial_seed ='210491' >
    <variable name='0000588:6'>
      <distribution>BPfailtime</distribution>
      <grid type='value' construction='equal' lowerBound='0.0' steps='10'>2880</grid>
    </variable>
    <variable name='0000575:6'>
      <distribution>BPrepairtime</distribution>
      <grid type='value' construction='equal' lowerBound='0.0' steps='10'>2880</grid>
    </variable>
    <variable name='20210000:4'>
      <distribution>ScalFactPower</distribution>
      <grid type='value' construction='equal' lowerBound='1.0' steps='10'>0.02</grid>
    </variable>
  </Grid>
  <MonteCarlo name='MC_Sampler' limit='1000'>
    <variable name='0000588:6'>
      <distribution>BPfailtime</distribution>
    </variable>
    <variable name='0000575:6'>
      <distribution>BPrepairtime</distribution>
    </variable>
    <variable name='20210000:4'>
      <distribution>ScalFactPower</distribution>
    </variable>
  </MonteCarlo>
</Samplers>
\end{lstlisting}

It can be seen that each variable is connected with a proper distribution
defined in the \\\xmlNode{Distributions} block (from the previous example).
%
The following demonstrates how the input for the first variable is read.

We are sampling a a variable situated in word 6 of the card 0000588 using a Grid
sampling method.
%
The distribution that this variable is following is a Triangular distribution
(see section above).
%
We are sampling this variable beginning from 0.0 in 10 \textit{equal} steps of
2880.
%
%%%%%%%%%%%%%%%%%%%%%%%%%%%%%%%%%%%%%%%%%%%%%%%%%%%%%%%%%%%
\subsubsection{Steps}
For a RELAP interface, the \xmlNode{MultiRun} step type will most likely be
used.
%
First, the step needs to be named: this name will be one of the names used in
the \xmlNode{Sequence} block.
%
In our example, \texttt{Grid\_Sampler} and \texttt{MC\_Sampler}.
%
\begin{lstlisting}[style=XML,morekeywords={name,debug,re-seeding}]
     <MultiRun name='Grid_Sampler' verbosity='debug' re-seeding='210491'>
\end{lstlisting}

With this step, we need to import all the files needed for the simulation:
\begin{itemize}
  \item RELAP input file
  \item Relap 5 Executable
  \item element tables -- tpfh2o
\end{itemize}
\begin{lstlisting}[style=XML,morekeywords={name,class,type}]
    <Input   class='Files' type=''>inputrelap.i</Input>
    <Input   class='Files' type=''>r5executable.x</Input>
    <Input   class='Files' type=''>tpfh2o</Input>
\end{lstlisting}
We then need to define which model will be used:
\begin{lstlisting}[style=XML]
    <Model  class='Models' type='Code'>MyRELAP</Model>
\end{lstlisting}
We then need to specify which Sampler is used, and this can be done as follows:
\begin{lstlisting}[style=XML]
    <Sampler class='Samplers' type='Grid'>Grid_Sampler</Sampler>
\end{lstlisting}
And lastly, we need to specify what kind of output the user wants.
%
For example the user might want to make a database (in RAVEN the database
created is an HDF5 file).
%
Here is a classical example:
\begin{lstlisting}[style=XML,morekeywords={class,type}]
    <Output  class='Databases' type='HDF5'>MC_out</Output>
\end{lstlisting}
Following is the example of two MultiRun steps which use different sampling
methods (grid and Monte Carlo), and creating two different databases for each
one:
\begin{lstlisting}[style=XML]
<Steps verbosity='debug'>
  <MultiRun name='Grid_Sampler' verbosity='debug' re-seeding='210491'>
    <Input   class='Files'     type=''    >X10.i</Input>
    <Input   class='Files'     type=''    >r5executable.x</Input>
    <Input   class='Files'     type=''    >tpfh2o</Input>
    <Model   class='Models'    type='Code'>MyRELAP</Model>
    <Sampler class='Samplers'  type='Grid'>Grid_Sampler</Sampler>
    <Output  class='Databases' type='HDF5'>Grid_out</Output>
  </MultiRun>
  <MultiRun name='MC_Sampler' verbosity='debug' re-seeding='210491'>
    <Input   class='Files'     type=''          >X10.i</Input>
    <Input   class='Files'     type=''          >r5executable.x</Input>
    <Input   class='Files'     type=''          >tpfh2o</Input>
    <Model   class='Models'    type='Code'      >MyRELAP</Model>
    <Sampler class='Samplers'  type='MonteCarlo'>MC_Sampler</Sampler>
    <Output  class='Databases' type='HDF5'      >MC_out</Output>
  </MultiRun>
</Steps>
\end{lstlisting}
%%%%%%%%%%%%%%%%%%%%%%%%%%%%%%%%%%%%%%%%%%%%%%%%%%%%%%
\subsubsection{Databases}
As shown in the \xmlNode{Steps} block, the code is creating two database objects
called \texttt{Grid\_out} and \texttt{MC\_out}.
%
So the user needs to input the following:
\begin{lstlisting}[style=XML]
<Databases>
  <HDF5 name="Grid_out"/>
  <HDF5 name="MC_out"/>
</Databases>
\end{lstlisting}
As listed before, this will create two databases.
%
The files will have names corresponding to their \xmlAttr{name} appended with
the .h5 extension (i.e. \texttt{Grid\_out.h5} and \texttt{MC\_out.h5}).

\subsection{RELAP7 Interface}
\subsubsection{Files}
In the \xmlNode{Files} section, as specified before, all of the files needed for
the code to run should be specified.
%
In the case of RELAP7, the files typically needed are the following:
\begin{itemize}
  \item RELAP7 Input file
  \item Control Logic file
\end{itemize}
Example:
\begin{lstlisting}[style=XML]
<Files>nat_circ,control_logic.py</Files>
\end{lstlisting}
%%%%%%%%%%%%%%%%%%%%%%%%%%%%%%%%%%%%%%%%%%%%%%%%%%%%%%%%
\subsubsection{Models}
For the \xmlNode{Models} block RELAP7 uses the RAVEN executable.
%
Here is a standard example of what can be used to use RELAP7 as the model:
\begin{lstlisting}[style=XML]
<Models>
    <Code name='MyRAVEN' subType='RAVEN'><executable>~path/to/RAVEN-opt</executable></Code>
</Models>
\end{lstlisting}
%%%%%%%%%%%%%%%%%%%%%%%%%%%%%%%%%%%%%%%%%%%%%%%%%%%%%%%%
\subsubsection{Distributions}
The \xmlNode{Distributions} block, when using RELAP7, has to also be specified
through the control logic.
%\maljdan{Reword possibly, I am not sure what you are trying to say.}
%The \xmlNode{Distributions} block, when using RELAP7, has to be specified also
%through the control logic.
%
Given the names of the distributions and their parameters, a python file should
be used for the control logic.
%
For example, the sampling of a normal distribution for the primary pressure in
RELAP7 is required.
%
\begin{lstlisting}[style=XML]
<Distributions>
 <Normal name="Prim_Pres">
 <mean>1000000</mean>
 <sigma>100<sigma/>
 </Normal>
</Distributions>
\end{lstlisting}

The python file associated to it should look like this:
%\lstset{margin=1.5cm}
\begin{lstlisting}[language=Python]
def initial_function(monitored, controlled, auxiliary)
    print("monitored",monitored,"controlled",
    controlled,"auxiliary",auxiliary)

    controlled.pressure_in_pressurizer =
     distributions.Prim_Pres.getDistributionRandom()
    return
\end{lstlisting}
%%%%%%%%%%%%%%%%%%%%%%%%%%%%%%%%%%%%%%%%%%%%%%%
\subsubsection{Samplers}
In the \xmlNode{Samplers} block, which kind of sampling method used, the
variables not sampled inside the control logic, and the distributions for the
chosen sampling method will be defined.
%
For the example, we are performing Monte Carlo sampling with 500 runs.
%
The global initial pressure wasn't specified in the control logic so it is
sampled in this block.
%
In addition, the initial pressure is specified to follow the same distribution
as the primary pressure.
%
\begin{lstlisting}[style=XML]
<Samplers>
 <MonteCarlo name="MC_samp" limit="500">
  <variable name="GlobalParams|global_init_P">
   <distribution>Prim_Pres</distribution>
  </variable>
 </MonteCarlo>
</Samplers>
\end{lstlisting}
