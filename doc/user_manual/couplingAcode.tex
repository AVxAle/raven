\section{Advanced Users: How to couple a new code.  \\ \vspace{2 mm} {\small }}
\label{sec:newCodeCoupling}
In order to couple your Application to the RAVEN code, your code needs to satisfy some (FEW) pre-requisites.
CLASSIFICATION:
Input pre-requisites:
You need to know your code input syntax :D. Hence, you can write a Python-compatible parser for you input (a module that is able to read and modify your code input)
You need to decide the syntax your Code Interface will be able to interpret 
Output pre-requisites:
RAVEN is able to handle CSV files (as outputs)
If your code output is not in CSV format, your interface needs to be able to convert it in a CSV format





As already mentioned, RAVEN can load your Code Interface automatically, without being aware of it in the source code.
RAVEN expects, in your interface, to find few methods (that are going to be analyzed, in details, later on in this Presentation):
generateCommand(self,inputFiles,executable)
appendLoadFileExtension(self,fileRoot)
createNewInput(self,currentInputFiles,oriInputFiles,samplerType,**Kwargs)
actionsAtTheEnd(self, currentInputFiles,oriInputFiles,samplerType,**Kwargs)
RAVEN passes in these methods all  the information needed for perturbing and handling the input files of your code (for example, the variables that got sampled, etc.)

 