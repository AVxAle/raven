
\subsection{System Purpose}

The RAVEN code is a generic software framework to perform parametric
and probabilistic analysis based on the response of complex system
codes. RAVEN is capable of investigating the system response as well
as the input space using Monte Carlo, Grid, or Latin Hyper Cube
sampling schemes, but its strength is focused toward system feature
discovery, such as limit surfaces, separating regions of the input
space leading to system failure, using dynamic supervised learning
techniques.

The development of RAVEN started in 2012 to satisfy the need to
provide a modern risk evaluation framework. RAVEN's principal
assignment is to provide the necessary software and algorithms in
order to employ the concept developed by the Risk Informed Safety
Margin Characterization (RISMC) program. RISMC is one of the pathways
defined within the Light Water Reactor Sustainability (LWRS)
program. In the RISMC approach, the goal is not just specifically
identifying the frequency of an event potentially leading to a system
failure, but the closeness (or not) to key safety-related events. This
approach may be used in identifying and increasing the safety margins
related to those events. A safety margin is a numerical value
quantifying the probability that a safety metric (e.g. as peak
pressure in a pipe) is exceeded under certain conditions. The initial
development of RAVEN has been focused on providing dynamic risk
assessment capability to the MOOSE application RELAP-7, currently
under development at the INL and, the likely future replacement of the
RELAP5-3D code. Most of the capabilities implemented using RELAP-7 are
easily deployable for other system codes.

\subsection{System Scope}

The produced product is the RAVEN software.  It is a computer
code designed for probabilistic analysis.  RAVEN is a statistical
analysis tool that is used to estimate risk by computing real numbers
to determine what can go wrong, how likely is it, and what are its
consequences.  RAVEN takes in input (such as input files for
subprograms, or CSV files of data) and then can run subprograms with
perturbed input to calculate the result of physical simulations with
varying input parameters.  Then RAVEN takes the output of those
program or the data provided and performs statistical analysis on the
data.

\subsection{Other Design Documentation}

In addition to this document, with every merge request developer
documentation is automatically extracted from the source code using
Doxygen and is available to developers at
\url{https://hpcsc.inl.gov/ssl/RAVEN/docs/classes.html}

The software is under configuration management process identified in
`` Configuration Management Plan for Modeling and Simulation
Software'' PLN-4004 Revision 3.

\subsection{Definitions and Acronyms}

\begin{description}
\item[API] Application Programming Interfaces
\item[CDF] Cumulative Distribution Function
\item[DET] Dynamic Event Tree
\item[LWRS] Light Water Reactor Sustainability
\item[MC] Monte Carlo
\item[MOOSE] Multiphysics Object-Oriented Simulation Environment
\item[PDF] Probability Distribution Function
\item[RAVEN] Risk Analysis Virtual ENvironment
\item[RELAP-7] Reactor Excursion and Leak Analysis Program v.7
\item[RISMC] Risk Informed Safety Margin Characterization
\item[ROM] Reduced Order Model
\end{description}
