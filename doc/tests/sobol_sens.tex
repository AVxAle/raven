\section{Global Sobol Sensitivity: Sudret}
Associated external model: \texttt{sudret\_sobol\_poly.py}

This model provides analytic Sobol sensitivities for a flexible number of input parameters.  It is taken from
\cite{sudret2007} and has the following form:
\begin{equation}
  u(Y) = \frac{1}{2^N} \prod_{n=1}^N \left(3y_n^2 + 1\right).
\end{equation}
The variables $y_n$ are distributed uniformly on [0,1].  For three input variables ($N=3$), the Sobol sensitivities are
as follows, to 12 digits of accuracy:
\begin{align}
  S_1 = S_2 = S_3 &= \frac{25}{91}\hspace{10pt} (0.2747), \\
  S_{1,2} = S_{1,3} = S_{2,3} &= \frac{5}{91}\hspace{10pt} (0.0549), \\
  S_{1,2,3} &= \frac{1}{91}\hspace{10pt} (0.0110).
\end{align}
The mean is 1.0 and the variance is 0.728.

\subsection{Second-Order ANOVA of Second-Order Cut-HDMR Expansion of Sudret}
One particular analytic tests involves calculating Sobol sensitivities for the second-order cut-HDMR expansion
of the Sudret.  We use three variables and equate $u(Y) = f(x,y,z)$.  The reference cut point is $(\bar
x,\bar y, \bar z) = (\frac{1}{2},\frac{1}{2},\frac{1}{2})$.

The first step is to construct the second-order cut-HDMR expansion $T$,
\begin{equation}
  f(x,y,z) \approx T[f](x,y,z) = t_r + t_x + t_y + t_z + t_{xy} + t_{xz} + t_{yz},
\end{equation}
\begin{equation}
  t_r \equiv f(\bar x,\bar y,\bar z),
\end{equation}
\begin{equation}
  t_x \equiv f(x,\bar y,\bar z) -t_r,
\end{equation}
\begin{equation}
  t_{xy} \equiv f(x,y,\bar z) - t_x - t_y - t_r.
\end{equation}
Symmetry between $x,y,z$ provides similar expressions for the remaining terms.
The cut-plane evaluations are
\begin{equation}
  f_0 = t_r \equiv f(\bar x,\bar y,\bar z) = \frac{343}{512},
\end{equation}
\begin{equation}
  f_x \equiv f(x,\bar y, \bar z) = \frac{49}{128}(3x^2+1),
\end{equation}
\begin{equation}
  f_{xy} \equiv f(x,y, \bar z) = \frac{7}{32}(3x^2+1)(3y^2+1),
\end{equation}
and similar for the remaining terms.  Expanding the cut-HDMR expression,
\begin{align}
  T[f](x,y,z) = t_r &+ (f_x-t_r) + (f_y-t_r) + (f_y-t_r) +\nonumber\\
  &[f_{xy} - (f_x-t_r) - (f_y-t_r) - t_r]+ \nonumber\\
  &[f_{xz} - (f_x-t_r) - (f_z-t_r) - t_r]+ \nonumber\\
  &[f_{yz} - (f_y-t_r) - (f_z-t_r) - t_r],
\end{align}
and collecting terms
\begin{equation}
  T[f](x,y,z) = t_r - f_x - f_y - f_z + f_{xy} + f_{xz} + f_{yz}.
\end{equation}
The ANOVA terms are recovered by integration of the cut-HDMR expansion.  The ANOVA expansion $H$ is similar in
appearance to the cut-HDMR but uses different definitions; in fact, cut-HDMR is a coarse approximation of the
ANOVA expansion.  Here we use ANOVA to approximate the cut-HDMR expansion, instead of the original model.

\begin{equation}
  H[T](x,y,z) = h_0 + h_x + h_y + h_z + h_{xy} + h_{xz} + h_{yz},
\end{equation}
\begin{equation}
  h_0 \equiv \int_0^1\int_0^1\int_0^1 T[f](x,y,z)\ dx\ dy\ dz,
\end{equation}
\begin{equation}
  h_x \equiv \int_0^1\int_0^1 T[f](x,y,z)\ dy\ dz - h_0,
\end{equation}
\begin{equation}
  h_{xy} \equiv \int_0^1 T[f](x,y,z)\ dz - h_x - h_y - h_0,
\end{equation}
and similarly for the remaining terms.  We evaluate the necessary integrals in the expansion.
\begin{align}
  \int_0^1\int_0^1\int_0^1 T[f](x,y,z)\ dxdydz &= \int_0^1\int_0^1\int_0^1 t_r - f_x - f_y - f_z + f_{xy} +
    f_{xz} + f_{yz}\ dxdydz,\nonumber \\
    &= \frac{343}{512} - 3\left(\frac{49}{64}\right) + 3\left(\frac{7}{8}\right), \nonumber\\
    &= \frac{511}{512}.\nonumber
\end{align}
\begin{equation}
  h_0 = \int_0^1\int_0^1\int_0^1 T[f](x,y,z)\ dxdydz = \frac{511}{512}.
\end{equation}
\begin{align}
  \int_0^1\int_0^1 T[f](x,y,z)\ dxdy &= \int_0^1\int_0^1 t_r - f_x - f_y - f_z + f_{xy} +
    f_{xz} + f_{yz}\ dxdy, \nonumber\\
    &= \frac{259}{512} + \frac{189}{128}z^2. \nonumber
\end{align}
\begin{equation}
  h_z = \int_0^1\int_0^1 T[f](x,y,z)\ dxdy - h_0 = \frac{259}{512} + \frac{189}{128}z^2 - \frac{511}{512} =
  -\frac{63}{128} + \frac{189}{128}z^2.
\end{equation}
\begin{align}
  \int_0^1 T[f](x,y,z)\ dx &= \int_0^1 t_r - f_x - f_y - f_z + f_{xy} + f_{xz} + f_{yz}\ dx, \nonumber\\
  &= \frac{119}{512} + \frac{105}{128}(y^2+z^2)+ \frac{63}{32}y^2z^2,\nonumber
\end{align}
\begin{align}
  h_{yz} &=\int_0^1 T[f](x,y,z)\ dx - h_y - h_z - h_0,\nonumber\\
  &= \frac{119}{512} + \frac{105}{128}(y^2+z^2)+ \frac{63}{32}y^2z^2 - \nonumber\\
  &=\frac{7}{32} - \frac{21}{32}(y^2+z^2) + \frac{63}{32}y^2z^2.
\end{align}
The other terms are obtained similarly, and are symmetric.  In summary,
\begin{align}
  h_0 &= \frac{511}{512}, \\
  h_x &= -\frac{63}{128} + \frac{189}{128}x^2, \\
  h_y &= -\frac{63}{128} + \frac{189}{128}y^2, \\
  h_z &= -\frac{63}{128} + \frac{189}{128}z^2, \\
  h_{xy} &= \frac{7}{32} - \frac{21}{32}(x^2+y^2) + \frac{63}{32}x^2y^2, \\
  h_{xz} &= \frac{7}{32} - \frac{21}{32}(x^2+z^2) + \frac{63}{32}x^2z^2, \\
  h_{yz} &= \frac{7}{32} - \frac{21}{32}(y^2+z^2) + \frac{63}{32}y^2z^2.
\end{align}
It can be shown that the expectation value of any ANOVA expansion term is zero, with the exception of the
first term $h_0$.  Additionally, each term is orthogonal to each other term.  The second moment can thus be
calculated as
\begin{align}
  \langle H[T](x,y,z)^2\rangle &\equiv \int_0^1\int_0^1\int_0^1 (h_0+h_x+h_y+h_z+h_{xy}+h_{xz}+h_{yz})^2\
  dxdydz, \nonumber\\
  &= h_0^2+\int_0^1\int_0^1\int_0^1 h_x^2+h_y^2+h_z^2+h_{xy}^2+h_{xz}^2+h_{yz}^2\ dxdydz.
\end{align}
To obtain the variance $\sigma_\text{tot}^2$, we subtract the square of the mean,
\begin{equation}
  \sigma_\text{tot}^2 = \int_0^1\int_0^1\int_0^1 h_x^2+h_y^2+h_z^2+h_{xy}^2+h_{xz}^2+h_{yz}^2\ dxdydz.
\end{equation}
The partial variance $\sigma_k^2$ of any subset $k$ is the integral of the square of that ANOVA term.
\begin{equation}
  \sigma_x^2 = \sigma_y^2 = \sigma_z^2 = \frac{3969}{20480} \approx0.19379883,
\end{equation}
\begin{equation}
  \sigma_{xy}^2 = \sigma_{xz}^2 = \sigma_{yz}^2 = \frac{49}{1600} =0.030625.
\end{equation}
The total variance is a sum of the partial variances,
\begin{equation}
  \sigma_\text{tot}^2 = 3\left(\frac{3969}{20480}\right) + 3\left(\frac{49}{1600}\right) =
  \frac{68943}{102400} \approx 0.67327.
\end{equation}

The Sobol indices are the ratio of the partial variance to the total,

\begin{equation}
  \mathcal{S}_x =\mathcal{S}_y = \mathcal{S}_z = \frac{135}{469} \approx 0.287846482,
\end{equation}
\begin{equation}
  \mathcal{S}_{xy} =\mathcal{S}_{xz} = \mathcal{S}_{yz} = \frac{64}{1407} \approx 0.045486851.
\end{equation}
%
%
%
%
%
%
%
\section{Global Sobol Sensitivity: Ishigami}
Associated external model: \texttt{ishigami.py}

This model has interesting properties for its sensitivity indices, in that $y_3$ has zero impact alone but a
nonzero impact when coupled with $y_1$.  Additionally, the sinusoidal expression is not trivially represented
by polynomial expansion.  It is listed in \cite{saltelli2000} and has the following form:
\begin{equation}
  u(Y) = \sin(y_1) + a\sin^2(y_2) + b y_3^4\sin(y_1),
\end{equation}
where in this case $a=7$ and $b=0.1$, and all $y_n$ are uniformly distributed on $[-\pi,\pi]$.

The variance and partial variances are as follows:
\begin{align}
  D_\text{tot} &= \frac{a^2}{8} + \frac{b\pi^4}{5} + \frac{b^2\pi^8}{18} + \frac{1}{2}, \\
  D_1 &= \frac{b\pi^4}{5} + \frac{b^2\pi^8}{50} + \frac{1}{2} ,\\
  D_2 &= \frac{a^2}{8}, \\
  D_3 &= 0, \\
  D_{1,2} &= 0, \\
  D_{2,3} &= 0, \\
  D_{1,3} &= \frac{8b^2\pi^8}{225}, \\
  D_{1,2,3} &= 0.
\end{align}
The corresponding variance values and Sobol sensitivities are listed in Table \ref{tab:ishigami sens}.
\begin{table}[h]
  \centering
  \begin{tabular}{c|c|c}
    Variable & Partial Variance & Sobol Index \\ \hline
    (total) & 13.8446 & 1 \\
    $y_1$         & 4.34589 & 0.3138 \\
    $y_2$         & 6.125   & 0.4424 \\
    $y_3$         & 0       & 0      \\
    $y_1,y_2$     & 0       & 0      \\
    $y_1,y_3$     & 3.3737  & 0.2436 \\
    $y_2,y_3$     & 0       & 0      \\
    $y_1,y_2,y_3$ & 0       & 0      \\
  \end{tabular}
  \caption{Ishigami sensitivities and variances}
  \label{tab:ishigami sens}
\end{table}

%
%
%
%
%
%
\section{Sobol G-Function}
Associated external model: \texttt{gFunction.py}

This function developed by Sobol has the benefit of tuning factors $a_n$ that allow the importance of any
particular term to be increased or decreased.  Because of the absolute value, this function is quite
challenging for polynomial expansion.  Documentation can be found in \cite{sobol2003}.  The function is
represented by
\begin{equation}
  u(Y) = \prod_{n=1}^N \frac{|4y_n - 2|+a_n}{1+a_n},
\end{equation}
where $y_n$ are distributed uniformly on [0,1] and $a_n$ are non-negative.  $a_n$ are generally integers, and
smaller values lead to greater impact of corresponding $y_n$.  As in \cite{sudret2007} we use $N=8$ with
$a=[1,2,5,10,20,50,100,500]$.  The partial variances are given by
\begin{equation}
  D_n = \frac{1}{3(1+a_n)^2},
\end{equation}
\begin{equation}
  D_\text{tot} = \prod_{n=1}^N (D_n+1)-1.
\end{equation}
Analytic values for Sobol sensitivities are given in Table \ref{tab:gfunc sens}.
\begin{table}[h]
  \centering
  \begin{tabular}{c|c}
    Variable & Sobol sensitivity \\ \hline
    $y_1$ & 0.6037 \\
    $y_2$ & 0.2683 \\
    $y_3$ & 0.0671 \\
    $y_4$ & 0.0200 \\
    $y_5$ & 0.0055 \\
    $y_6$ & 0.0009 \\
    $y_7$ & 0.0002 \\
    $y_8$ & 0.0000 \\
  \end{tabular}
  \caption{G-Function sensitivities and variances}
  \label{tab:gfunc sens}
\end{table}
