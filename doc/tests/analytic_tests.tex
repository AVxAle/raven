%
% This is an example LaTeX file which uses the SANDreport class file.
% It shows how a SAND report should be formatted, what sections and
% elements it should contain, and how to use the SANDreport class.
% It uses the LaTeX article class, but not the strict option.
% ItINLreport uses .eps logos and files to show how pdflatex can be used
%
% Get the latest version of the class file and more at
%    http://www.cs.sandia.gov/~rolf/SANDreport
%
% This file and the SANDreport.cls file are based on information
% contained in "Guide to Preparing {SAND} Reports", Sand98-0730, edited
% by Tamara K. Locke, and the newer "Guide to Preparing SAND Reports and
% Other Communication Products", SAND2002-2068P.
% Please send corrections and suggestions for improvements to
% Rolf Riesen, Org. 9223, MS 1110, rolf@cs.sandia.gov
%
\documentclass[pdf,12pt]{INLreport}
% pslatex is really old (1994).  It attempts to merge the times and mathptm packages.
% My opinion is that it produces a really bad looking math font.  So why are we using it?
% If you just want to change the text font, you should just \usepackage{times}.
% \usepackage{pslatex}
\usepackage{times}
\usepackage[FIGBOTCAP,normal,bf,tight]{subfigure}
\usepackage{amsmath}
\usepackage{amssymb}
\usepackage{pifont}
\usepackage{enumerate}
\usepackage{listings}
\usepackage{fullpage}
\usepackage{xcolor}          % Using xcolor for more robust color specification
\usepackage{ifthen}          % For simple checking in newcommand blocks
\usepackage{textcomp}
\usepackage{graphicx}
\usepackage{float}
\usepackage[toc,page]{appendix}
%\usepackage{authblk}         % For making the author list look prettier
%\renewcommand\Authsep{,~\,}

% Custom colors
\definecolor{deepblue}{rgb}{0,0,0.5}
\definecolor{deepred}{rgb}{0.6,0,0}
\definecolor{deepgreen}{rgb}{0,0.5,0}
\definecolor{forestgreen}{RGB}{34,139,34}
\definecolor{orangered}{RGB}{239,134,64}
\definecolor{darkblue}{rgb}{0.0,0.0,0.6}
\definecolor{gray}{rgb}{0.4,0.4,0.4}

\lstset {
  basicstyle=\ttfamily,
  frame=single
}

\setcounter{secnumdepth}{5}
\lstdefinestyle{XML} {
    language=XML,
    extendedchars=true,
    breaklines=true,
    breakatwhitespace=true,
%    emph={name,dim,interactive,overwrite},
    emphstyle=\color{red},
    basicstyle=\ttfamily,
%    columns=fullflexible,
    commentstyle=\color{gray}\upshape,
    morestring=[b]",
    morecomment=[s]{<?}{?>},
    morecomment=[s][\color{forestgreen}]{<!--}{-->},
    keywordstyle=\color{cyan},
    stringstyle=\ttfamily\color{black},
    tagstyle=\color{darkblue}\bf\ttfamily,
    morekeywords={name,type},
%    morekeywords={name,attribute,source,variables,version,type,release,x,z,y,xlabel,ylabel,how,text,param1,param2,color,label},
}
\lstset{language=python,upquote=true}

\usepackage{titlesec}
\newcommand{\sectionbreak}{\clearpage}
\setcounter{secnumdepth}{4}

%\titleformat{\paragraph}
%{\normalfont\normalsize\bfseries}{\theparagraph}{1em}{}
%\titlespacing*{\paragraph}
%{0pt}{3.25ex plus 1ex minus .2ex}{1.5ex plus .2ex}

%%%%%%%% Begin comands definition to input python code into document
\usepackage[utf8]{inputenc}

% Default fixed font does not support bold face
\DeclareFixedFont{\ttb}{T1}{txtt}{bx}{n}{9} % for bold
\DeclareFixedFont{\ttm}{T1}{txtt}{m}{n}{9}  % for normal

\usepackage{listings}

% Python style for highlighting
\newcommand\pythonstyle{\lstset{
language=Python,
basicstyle=\ttm,
otherkeywords={self, none, return},             % Add keywords here
keywordstyle=\ttb\color{deepblue},
emph={MyClass,__init__},          % Custom highlighting
emphstyle=\ttb\color{deepred},    % Custom highlighting style
stringstyle=\color{deepgreen},
frame=tb,                         % Any extra options here
showstringspaces=false            %
}}


% Python environment
\lstnewenvironment{python}[1][]
{
\pythonstyle
\lstset{#1}
}
{}

% Python for external files
\newcommand\pythonexternal[2][]{{
\pythonstyle
\lstinputlisting[#1]{#2}}}

\lstnewenvironment{xml}
{}
{}

% Python for inline
\newcommand\pythoninline[1]{{\pythonstyle\lstinline!#1!}}

% Named Colors for the comments below (Attempted to match git symbol colors)
\definecolor{RScolor}{HTML}{8EB361}  % Sonat (adjusted for clarity)
\definecolor{DPMcolor}{HTML}{E28B8D} % Dan
\definecolor{JCcolor}{HTML}{82A8D9}  % Josh (adjusted for clarity)
\definecolor{AAcolor}{HTML}{8D7F44}  % Andrea
\definecolor{CRcolor}{HTML}{AC39CE}  % Cristian
\definecolor{RKcolor}{HTML}{3ECC8D}  % Bob (adjusted for clarity)
\definecolor{DMcolor}{HTML}{276605}  % Diego (adjusted for clarity)
\definecolor{PTcolor}{HTML}{990000}  % Paul

\def\DRAFT{} % Uncomment this if you want to see the notes people have been adding
% Comment command for developers (Should only be used under active development)
\ifdefined\DRAFT
  \newcommand{\nameLabeler}[3]{\textcolor{#2}{[[#1: #3]]}}
\else
  \newcommand{\nameLabeler}[3]{}
\fi
\newcommand{\alfoa}[1] {\nameLabeler{Andrea}{AAcolor}{#1}}
\newcommand{\cristr}[1] {\nameLabeler{Cristian}{CRcolor}{#1}}
\newcommand{\mandd}[1] {\nameLabeler{Diego}{DMcolor}{#1}}
\newcommand{\maljdan}[1] {\nameLabeler{Dan}{DPMcolor}{#1}}
\newcommand{\cogljj}[1] {\nameLabeler{Josh}{JCcolor}{#1}}
\newcommand{\bobk}[1] {\nameLabeler{Bob}{RKcolor}{#1}}
\newcommand{\senrs}[1] {\nameLabeler{Sonat}{RScolor}{#1}}
\newcommand{\talbpaul}[1] {\nameLabeler{Paul}{PTcolor}{#1}}
% Commands for making the LaTeX a bit more uniform and cleaner
\newcommand{\TODO}[1]    {\textcolor{red}{\textit{(#1)}}}
\newcommand{\xmlAttrRequired}[1] {\textcolor{red}{\textbf{\texttt{#1}}}}
\newcommand{\xmlAttr}[1] {\textcolor{cyan}{\textbf{\texttt{#1}}}}
\newcommand{\xmlNodeRequired}[1] {\textcolor{deepblue}{\textbf{\texttt{<#1>}}}}
\newcommand{\xmlNode}[1] {\textcolor{darkblue}{\textbf{\texttt{<#1>}}}}
\newcommand{\xmlString}[1] {\textcolor{black}{\textbf{\texttt{'#1'}}}}
\newcommand{\xmlDesc}[1] {\textbf{\textit{#1}}} % Maybe a misnomer, but I am
                                                % using this to detail the data
                                                % type and necessity of an XML
                                                % node or attribute,
                                                % xmlDesc = XML description
\newcommand{\default}[1]{~\\*\textit{Default: #1}}
\newcommand{\nb} {\textcolor{deepgreen}{\textbf{~Note:}}~}

%

%%%%%%%% End comands definition to input python code into document

%\usepackage[dvips,light,first,bottomafter]{draftcopy}
%\draftcopyName{Sample, contains no OUO}{70}
%\draftcopyName{Draft}{300}

% The bm package provides \bm for bold math fonts.  Apparently
% \boldsymbol, which I used to always use, is now considered
% obsolete.  Also, \boldsymbol doesn't even seem to work with
% the fonts used in this particular document...
\usepackage{bm}

% Define tensors to be in bold math font.
\newcommand{\tensor}[1]{{\bm{#1}}}

% Override the formatting used by \vec.  Instead of a little arrow
% over the letter, this creates a bold character.
\renewcommand{\vec}{\bm}

% Define unit vector notation.  If you don't override the
% behavior of \vec, you probably want to use the second one.
\newcommand{\unit}[1]{\hat{\bm{#1}}}
% \newcommand{\unit}[1]{\hat{#1}}

% Use this to refer to a single component of a unit vector.
\newcommand{\scalarunit}[1]{\hat{#1}}

% set method for expressing expected value as E[f]
\newcommand{\expv}[1]{\ensuremath{\mathbb{E}[ #1]}}

% \toprule, \midrule, \bottomrule for tables
\usepackage{booktabs}

% \llbracket, \rrbracket
\usepackage{stmaryrd}

\usepackage{hyperref}
\hypersetup{
    colorlinks,
    citecolor=black,
    filecolor=black,
    linkcolor=black,
    urlcolor=black
}

% Compress lists of citations like [33,34,35,36,37] to [33-37]
\usepackage{cite}

% If you want to relax some of the SAND98-0730 requirements, use the "relax"
% option. It adds spaces and boldface in the table of contents, and does not
% force the page layout sizes.
% e.g. \documentclass[relax,12pt]{SANDreport}
%
% You can also use the "strict" option, which applies even more of the
% SAND98-0730 guidelines. It gets rid of section numbers which are often
% useful; e.g. \documentclass[strict]{SANDreport}

% The INLreport class uses \flushbottom formatting by default (since
% it's intended to be two-sided document).  \flushbottom causes
% additional space to be inserted both before and after paragraphs so
% that no matter how much text is actually available, it fills up the
% page from top to bottom.  My feeling is that \raggedbottom looks much
% better, primarily because most people will view the report
% electronically and not in a two-sided printed format where some argue
% \raggedbottom looks worse.  If we really want to have the original
% behavior, we can comment out this line...
\raggedbottom
\setcounter{secnumdepth}{5} % show 5 levels of subsection
\setcounter{tocdepth}{5} % include 5 levels of subsection in table of contents

% ---------------------------------------------------------------------------- %
%
% Set the title, author, and date
%
\title{RAVEN Analytic Test Documentation}

\author{
\textbf{\textit{Principal Investigator (PI):}}
 \\Cristian Rabiti\\
\textbf{\textit{Main Developers:}}
\\Andrea Alfonsi
\\Joshua Cogliati
\\Diego Mandelli
\\Robert Kinoshita
\\Congjian Wang
\\Daniel P. Maljovec
\\Paul W. Talbot
}

% There is a "Printed" date on the title page of a SAND report, so
% the generic \date should [WorkingDir:]generally be empty.
\date{}


% ---------------------------------------------------------------------------- %
% Set some things we need for SAND reports. These are mandatory
%
%TODO someone help me know what goes here?  - Paul
\SANDnum{todo}
\SANDprintDate{todo}
\SANDauthor{todo}
\SANDreleaseType{todo}


% ---------------------------------------------------------------------------- %
% Include the markings required for your SAND report. The default is "Unlimited
% Release". You may have to edit the file included here, or create your own
% (see the examples provided).
%
% \include{MarkOUO} % Not needed for unlimted release reports

\def\component#1{\texttt{#1}}

% ---------------------------------------------------------------------------- %
\newcommand{\systemtau}{\tensor{\tau}_{\!\text{SUPG}}}

% ---------------------------------------------------------------------------- %
%
% Start the document
%

\begin{document}
    \maketitle

    \cleardoublepage		% TOC needs to start on an odd page
    \tableofcontents

    % ---------------------------------------------------------------------- %
    % This is where the body of the report begins; usually with an Introduction
    %
    \SANDmain		% Start the main part of the report

\section{Introduction}
In the interest of benchmarking and maintaining algorithms developed and used within \texttt{raven}, we
present here analytic benchmarks associated with specific models. The associated external models
referenced in each case can be found in

\texttt{raven/tests/framework/AnalyticModels/}

\documentclass[11pt]{article}
\usepackage[utf8]{inputenc}
\usepackage{amsmath}
\usepackage{amsfonts}
\usepackage{graphicx}
\usepackage{float}
\usepackage{fullpage}
\newcommand{\expv}[1]{\ensuremath{\mathbb{E}[ #1]}}


\textwidth6.6in
\textheight9in


\setlength{\topmargin}{0.3in} \addtolength{\topmargin}{-\headheight}
\addtolength{\topmargin}{-\headsep}

\setlength{\oddsidemargin}{0in}

\oddsidemargin  0.0in \evensidemargin 0.0in \parindent0em

%\pagestyle{fancy}\lhead{MATH 579 (UQ for PDEs)} \rhead{02/24/2014}
%\chead{Project Proposal} \lfoot{} \rfoot{\bf \thepage} \cfoot{}


\begin{document}

\title{1D Attenuation Case}

\author{Paul Talbot\thanks{talbotp@unm.edu}}
\date{}
\maketitle

Attenuation evaluation for quantity of interest $u$ with input parameters $Y=[y_1,\ldots,y_N]$:
\begin{equation}
u(Y) = \prod_{n=1}^N e^{-y_n/N}.
\end{equation}
This is the solution to the exit strength of a monodirectional, single-energy beam of neutral particles incident on a unit length material divided into $N$ sections with independently-varying absorption cross sections.  This test is useful for its analytic statistical moments as well as difficulty to represent accurately using polynomial representations.

\section{Uniform}
Let all $y_n$ be uniformly distributed between 0 and 1.  The first two statistical moments are:
\subsection{mean}
\begin{align}
\expv{u(Y)} &=\int_{0}^1 dY \rho(Y)u(Y),\\
  &=\int_{0}^1 dy_1\cdots\int_{0}^1 dy_N \prod_{n=1}^N \frac{e^{-y_n/N}}{2},\\
  &=\left[ \int_{0}^1 dy \frac{1}{2}e^{-y/N}\right]^N,\\
  &=\left[\frac{1}{2}\left(-Ne^{-y/N}\right)\bigg|_0^1\right]^N,\\
  &=\left[\frac{N}{2}\left(1-e^{-1/N}\right)\right]^N.
\end{align}
\subsection{variance}
\begin{align}
\expv{u(Y)^2} &= \int_{0}^1 dY \rho(Y)u(Y),\\
  &=\int_{0}^1 dy_1\cdots\int_{0}^1 dy_N \frac{1}{2^N} \left(\prod_{n=1}^N e^{-y_n/N}\right)^2,\\
  &=\left[\frac{1}{2}\left(\int_{0}^1 dy\ e^{-2y/N} \right)\right]^N,\\
  &=\left[\frac{1}{2}\left(\frac{N}{2}e^{-2y/N} \right)\bigg|_{0}^1 \right]^N,\\
  &=\left[\frac{N}{4}\left(1-e^{-2/N}\right)\right]^N.\\
\text{var}[u(Y)] &= \expv{u(Y)^2}-\expv{u(Y)}^2,\\
  &= \left[\frac{N}{4}\left(1-e^{-2/N}\right)\right]^N - \left[\frac{N}{2}\left(1-e^{-1/N}\right)\right]^{2N}.
\end{align}
\subsection{values}
\begin{table}[H]
\centering
\begin{tabular}{c|c|c}
$N$ & mean & variance \\ \hline
3  & 0.076875696691760276135 & 0.042692141294802908114\\
5  & 0.019112640711428062993 & 0.011520078695574176371\\
10 & 0.000594788019229837393 & 0.000364939262176318060\\
\end{tabular}
\end{table}

\section{Tests using this model}
\texttt{raven/tests/framework/ROM/Sobol:}
\begin{itemize}
  \item AdaptiveSobol
  \item AdaptiveSobolMaxRuns
  \item AdaptiveSobolParallel
\end{itemize}
\end{document}

\section{Tensor Polynomial (First-Order)}
Associated external model: \texttt{tensor\_poly.py}

Tensor polynomial evaluation for quantity of interest $u$ with input parameters $Y=[y_1,\ldots,y_N]$:
\begin{equation}
u(Y) = \prod_{n=1}^N y_n+1.
\end{equation}
This test is specifically useful for its analytic statistical moments.  It is used as a benchmark in
\cite{ayreseaton2015}.

\subsection{Uniform, (-1,1)}
Let all $y_n$ be uniformly distributed between -1 and 1.  The first two statistical moments are:

\subsubsection{mean}
\begin{align}
\expv{u(Y)} &=\int_{-1}^1 dY \rho(Y)u(Y),\\
  &=\int_{-1}^1 dy_1\cdots\int_{-1}^1 dy_N \prod_{n=1}^N \frac{y_n+1}{2},\\
  &=\left[ \int_{-1}^1 dy \frac{y+1}{2}\right]^N,\\
  &=\left[\frac{1}{2}\left(\frac{y^2}{2}+y\right)\bigg|_{-1}^1\right]^N,\\
  &=\left[\frac{2}{2}\right]^N,\\
  &=1.
\end{align}

\subsubsection{variance}
\begin{align}
\expv{u(Y)^2} &= \int_{-1}^1 dY \rho(Y)u(Y),\\
  &=\int_{-1}^1 dy_1\cdots\int_{-1}^1 dy_N \frac{1}{2^N} \left(\prod_{n=1}^N y_n+1\right)^2,\\
  &=\left[\frac{1}{2}\left(\int_{-1}^1 dy\ y^2+2y+1 \right)\right]^N,\\
  &=\left[\frac{1}{2}\left(\frac{y^3}{3}+y^2+y \right)\bigg|_{-1}^1 \right]^N,\\
  &=\left[\frac{1}{3}+1\right]^N, \\
  &=\left(\frac{4}{3}\right)^N.\\
\text{var}[u(Y)] &= \expv{u(Y)^2}-\expv{u(Y)}^2,\\
  &= \left(\frac{4}{3}\right)^N-1.
\end{align}

\subsubsection{numeric values}
Some numeric values for the mean and variance are listed below for several input cardinalities $N$.
\begin{table}[h!]
  \centering
  \begin{tabular}{c|c|c}
    $N$ & mean & variance \\ \hline
    2 & 1.0 & 0.77777777777 \\
    4 & 1.0 & 2.16049382716 \\
    6 & 1.0 & 4.61865569273
  \end{tabular}
\end{table}

\subsection{Uniform, (0,1)}
Let all $y_n$ be uniformly distributed between 0 and 1.  The first two statistical moments are:

\subsubsection{mean}
\begin{align}
\expv{u(Y)} &=\int_{0}^1 dY \rho(Y)u(Y),\\
  &=\int_{0}^1 dy_1\cdots\int_{0}^1 dy_N \prod_{n=1}^N y_n+1,\\
  &=\left[ \int_{0}^1 dy y+1\right]^N,\\
  &=\left[\left(\frac{y^2}{2}+y\right)\bigg|_{0}^1\right]^N,\\
  &=\left[\frac{3}{2}\right]^N.
\end{align}

\subsubsection{variance}
\begin{align}
\expv{u(Y)^2} &= \int_{0}^1 dY \rho(Y)u(Y),\\
  &=\int_{0}^1 dy_1\cdots\int_{0}^1 dy_N \left(\prod_{n=1}^N y_n+1\right)^2,\\
  &=\left[\left(\int_{0}^1 dy\ y^2+2y+1 \right)\right]^N,\\
  &=\left[\left(\frac{y^3}{3}+y^2+y \right)\bigg|_{0}^1 \right]^N,\\
  &=\left(\frac{7}{3}\right)^N.\\
\text{var}[u(Y)] &= \expv{u(Y)^2}-\expv{u(Y)}^2,\\
  &= \left(\frac{7}{3}\right)^N-\left(\frac{3}{2}\right)^N.
\end{align}

\subsubsection{numeric values}
Some numeric values for the mean and variance are listed below for several input cardinalities $N$.
\begin{table}[h!]
  \centering
  \begin{tabular}{c|c|c}
    $N$ & mean & variance \\ \hline
    2 & 2.25      & 0.38194444444 \\
    4 & 5.0625    & 4.01306905864 \\
    6 & 11.390625 & 31.6377499009
  \end{tabular}
\end{table}

\section{Global Sobol Sensitivity 1}
Associated external model: \texttt{sudret\_sobol\_poly.py}

This model provides analytic Sobol sensitivities for a flexible number of input parameters.  It is taken from
\ref{sudret} and has the following form:
\begin{equation}
  u(Y) = \frac{1}{2^N} \prod_{n=1}^N \left(3y_n^2 + 1).
\end{equation}
The variables $y_n$ are distributed uniformly on [0,1].  For five input variables ($N=5$), the Sobol sensitivities are
as follows, to 12 digits of accuracy:
\begin{align}
  S_1 = S_2 = S_3 &= \frac{25}{91} (0.2747), \\
  S_{1,2} = S_{1,3} = S_{2,3} &= \frac{5}{91} (0.0549), \\
  S_{1,2,3} &= \frac{1}{91} (0.0110).
\end{align}
The mean is 1.0 and the variance is 0.72*.

\begin{appendices}
  \section{Document Version Information}
  \input{../version.tex}
\end{appendices}


    % ---------------------------------------------------------------------- %
    % References
    %
    \clearpage
    % If hyperref is included, then \phantomsection is already defined.
    % If not, we need to define it.
    \providecommand*{\phantomsection}{}
    \phantomsection
    \addcontentsline{toc}{section}{References}
    \bibliographystyle{ieeetr}
    \bibliography{analytic_tests}

\end{document}
