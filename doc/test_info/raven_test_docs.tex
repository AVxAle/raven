%
% This is an example LaTeX file which uses the SANDreport class file.
% It shows how a SAND report should be formatted, what sections and
% elements it should contain, and how to use the SANDreport class.
% It uses the LaTeX article class, but not the strict option.
% ItINLreport uses .eps logos and files to show how pdflatex can be used
%
% Get the latest version of the class file and more at
%    http://www.cs.sandia.gov/~rolf/SANDreport
%
% This file and the SANDreport.cls file are based on information
% contained in "Guide to Preparing {SAND} Reports", Sand98-0730, edited
% by Tamara K. Locke, and the newer "Guide to Preparing SAND Reports and
% Other Communication Products", SAND2002-2068P.
% Please send corrections and suggestions for improvements to
% Rolf Riesen, Org. 9223, MS 1110, rolf@cs.sandia.gov
%
\documentclass[pdf,12pt]{INLreport}
% pslatex is really old (1994).  It attempts to merge the times and mathptm packages.
% My opinion is that it produces a really bad looking math font.  So why are we using it?
% If you just want to change the text font, you should just \usepackage{times}.
% \usepackage{pslatex}
\usepackage{times}
\usepackage[FIGBOTCAP,normal,bf,tight]{subfigure}
\usepackage{amsmath}
\usepackage{amssymb}
\usepackage{pifont}
\usepackage{enumerate}
\usepackage{listings}
\usepackage{fullpage}
\usepackage{xcolor}          % Using xcolor for more robust color specification
\usepackage{ifthen}          % For simple checking in newcommand blocks
\usepackage{textcomp}
\usepackage{graphicx}
\usepackage{float}
\usepackage[toc,page]{appendix}
%\usepackage{authblk}         % For making the author list look prettier
%\renewcommand\Authsep{,~\,}

% Custom colors
\definecolor{deepblue}{rgb}{0,0,0.5}
\definecolor{deepred}{rgb}{0.6,0,0}
\definecolor{deepgreen}{rgb}{0,0.5,0}
\definecolor{forestgreen}{RGB}{34,139,34}
\definecolor{orangered}{RGB}{239,134,64}
\definecolor{darkblue}{rgb}{0.0,0.0,0.6}
\definecolor{gray}{rgb}{0.4,0.4,0.4}

\lstset {
  basicstyle=\ttfamily,
  frame=single
}

\setcounter{secnumdepth}{5}
\lstdefinestyle{XML} {
    language=XML,
    extendedchars=true,
    breaklines=true,
    breakatwhitespace=true,
%    emph={name,dim,interactive,overwrite},
    emphstyle=\color{red},
    basicstyle=\ttfamily,
%    columns=fullflexible,
    commentstyle=\color{gray}\upshape,
    morestring=[b]",
    morecomment=[s]{<?}{?>},
    morecomment=[s][\color{forestgreen}]{<!--}{-->},
    keywordstyle=\color{cyan},
    stringstyle=\ttfamily\color{black},
    tagstyle=\color{darkblue}\bf\ttfamily,
    morekeywords={name,type},
%    morekeywords={name,attribute,source,variables,version,type,release,x,z,y,xlabel,ylabel,how,text,param1,param2,color,label},
}
\lstset{language=python,upquote=true}

\usepackage{titlesec}
\newcommand{\sectionbreak}{\clearpage}
\setcounter{secnumdepth}{4}

%\titleformat{\paragraph}
%{\normalfont\normalsize\bfseries}{\theparagraph}{1em}{}
%\titlespacing*{\paragraph}
%{0pt}{3.25ex plus 1ex minus .2ex}{1.5ex plus .2ex}

%%%%%%%% Begin comands definition to input python code into document
\usepackage[utf8]{inputenc}

% Default fixed font does not support bold face
\DeclareFixedFont{\ttb}{T1}{txtt}{bx}{n}{9} % for bold
\DeclareFixedFont{\ttm}{T1}{txtt}{m}{n}{9}  % for normal

\usepackage{listings}

% Python style for highlighting
\newcommand\pythonstyle{\lstset{
language=Python,
basicstyle=\ttm,
otherkeywords={self, none, return},             % Add keywords here
keywordstyle=\ttb\color{deepblue},
emph={MyClass,__init__},          % Custom highlighting
emphstyle=\ttb\color{deepred},    % Custom highlighting style
stringstyle=\color{deepgreen},
frame=tb,                         % Any extra options here
showstringspaces=false            %
}}


% Python environment
\lstnewenvironment{python}[1][]
{
\pythonstyle
\lstset{#1}
}
{}

% Python for external files
\newcommand\pythonexternal[2][]{{
\pythonstyle
\lstinputlisting[#1]{#2}}}

\lstnewenvironment{xml}
{}
{}

% Python for inline
\newcommand\pythoninline[1]{{\pythonstyle\lstinline!#1!}}

% Named Colors for the comments below (Attempted to match git symbol colors)
\definecolor{RScolor}{HTML}{8EB361}  % Sonat (adjusted for clarity)
\definecolor{DPMcolor}{HTML}{E28B8D} % Dan
\definecolor{JCcolor}{HTML}{82A8D9}  % Josh (adjusted for clarity)
\definecolor{AAcolor}{HTML}{8D7F44}  % Andrea
\definecolor{CRcolor}{HTML}{AC39CE}  % Cristian
\definecolor{RKcolor}{HTML}{3ECC8D}  % Bob (adjusted for clarity)
\definecolor{DMcolor}{HTML}{276605}  % Diego (adjusted for clarity)
\definecolor{PTcolor}{HTML}{990000}  % Paul

\def\DRAFT{} % Uncomment this if you want to see the notes people have been adding
% Comment command for developers (Should only be used under active development)
\ifdefined\DRAFT
  \newcommand{\nameLabeler}[3]{\textcolor{#2}{[[#1: #3]]}}
\else
  \newcommand{\nameLabeler}[3]{}
\fi
\newcommand{\alfoa}[1] {\nameLabeler{Andrea}{AAcolor}{#1}}
\newcommand{\cristr}[1] {\nameLabeler{Cristian}{CRcolor}{#1}}
\newcommand{\mandd}[1] {\nameLabeler{Diego}{DMcolor}{#1}}
\newcommand{\maljdan}[1] {\nameLabeler{Dan}{DPMcolor}{#1}}
\newcommand{\cogljj}[1] {\nameLabeler{Josh}{JCcolor}{#1}}
\newcommand{\bobk}[1] {\nameLabeler{Bob}{RKcolor}{#1}}
\newcommand{\senrs}[1] {\nameLabeler{Sonat}{RScolor}{#1}}
\newcommand{\talbpaul}[1] {\nameLabeler{Paul}{PTcolor}{#1}}
% Commands for making the LaTeX a bit more uniform and cleaner
\newcommand{\TODO}[1]    {\textcolor{red}{\textit{(#1)}}}
\newcommand{\xmlAttrRequired}[1] {\textcolor{red}{\textbf{\texttt{#1}}}}
\newcommand{\xmlAttr}[1] {\textcolor{cyan}{\textbf{\texttt{#1}}}}
\newcommand{\xmlNodeRequired}[1] {\textcolor{deepblue}{\textbf{\texttt{<#1>}}}}
\newcommand{\xmlNode}[1] {\textcolor{darkblue}{\textbf{\texttt{<#1>}}}}
\newcommand{\xmlString}[1] {\textcolor{black}{\textbf{\texttt{'#1'}}}}
\newcommand{\xmlDesc}[1] {\textbf{\textit{#1}}} % Maybe a misnomer, but I am
                                                % using this to detail the data
                                                % type and necessity of an XML
                                                % node or attribute,
                                                % xmlDesc = XML description
\newcommand{\default}[1]{~\\*\textit{Default: #1}}
\newcommand{\nb} {\textcolor{deepgreen}{\textbf{~Note:}}~}

%

%%%%%%%% End comands definition to input python code into document

%\usepackage[dvips,light,first,bottomafter]{draftcopy}
%\draftcopyName{Sample, contains no OUO}{70}
%\draftcopyName{Draft}{300}

% The bm package provides \bm for bold math fonts.  Apparently
% \boldsymbol, which I used to always use, is now considered
% obsolete.  Also, \boldsymbol doesn't even seem to work with
% the fonts used in this particular document...
\usepackage{bm}

% Define tensors to be in bold math font.
\newcommand{\tensor}[1]{{\bm{#1}}}

% Override the formatting used by \vec.  Instead of a little arrow
% over the letter, this creates a bold character.
\renewcommand{\vec}{\bm}

% Define unit vector notation.  If you don't override the
% behavior of \vec, you probably want to use the second one.
\newcommand{\unit}[1]{\hat{\bm{#1}}}
% \newcommand{\unit}[1]{\hat{#1}}

% Use this to refer to a single component of a unit vector.
\newcommand{\scalarunit}[1]{\hat{#1}}

% set method for expressing expected value as E[f]
\newcommand{\expv}[1]{\ensuremath{\mathbb{E}[ #1]}}

% \toprule, \midrule, \bottomrule for tables
\usepackage{booktabs}

% \llbracket, \rrbracket
\usepackage{stmaryrd}

\usepackage{hyperref}
\hypersetup{
    colorlinks,
    citecolor=black,
    filecolor=black,
    linkcolor=black,
    urlcolor=black
}

% Compress lists of citations like [33,34,35,36,37] to [33-37]
\usepackage{cite}

% If you want to relax some of the SAND98-0730 requirements, use the "relax"
% option. It adds spaces and boldface in the table of contents, and does not
% force the page layout sizes.
% e.g. \documentclass[relax,12pt]{SANDreport}
%
% You can also use the "strict" option, which applies even more of the
% SAND98-0730 guidelines. It gets rid of section numbers which are often
% useful; e.g. \documentclass[strict]{SANDreport}

% The INLreport class uses \flushbottom formatting by default (since
% it's intended to be two-sided document).  \flushbottom causes
% additional space to be inserted both before and after paragraphs so
% that no matter how much text is actually available, it fills up the
% page from top to bottom.  My feeling is that \raggedbottom looks much
% better, primarily because most people will view the report
% electronically and not in a two-sided printed format where some argue
% \raggedbottom looks worse.  If we really want to have the original
% behavior, we can comment out this line...
\raggedbottom
\setcounter{secnumdepth}{5} % show 5 levels of subsection
\setcounter{tocdepth}{5} % include 5 levels of subsection in table of contents

% ---------------------------------------------------------------------------- %
%
% Set the title, author, and date
%
\title{RAVEN Analytic Test Documentation}

\author{
\textbf{\textit{Principal Investigator (PI):}}
 \\Cristian Rabiti\\
\textbf{\textit{Main Developers:}}
\\Andrea Alfonsi
\\Joshua Cogliati
\\Diego Mandelli
\\Robert Kinoshita
\\Congjian Wang
\\Daniel P. Maljovec
\\Paul W. Talbot
}

% There is a "Printed" date on the title page of a SAND report, so
% the generic \date should [WorkingDir:]generally be empty.
\date{}


% ---------------------------------------------------------------------------- %
% Set some things we need for SAND reports. These are mandatory
%
%TODO someone help me know what goes here?  - Paul
\SANDnum{todo}
\SANDprintDate{todo}
\SANDauthor{todo}
\SANDreleaseType{todo}


% ---------------------------------------------------------------------------- %
% Include the markings required for your SAND report. The default is "Unlimited
% Release". You may have to edit the file included here, or create your own
% (see the examples provided).
%
% \include{MarkOUO} % Not needed for unlimted release reports

\def\component#1{\texttt{#1}}

% ---------------------------------------------------------------------------- %
\newcommand{\systemtau}{\tensor{\tau}_{\!\text{SUPG}}}

% ---------------------------------------------------------------------------- %
%
% Start the document
%

\begin{document}
    \maketitle

    \cleardoublepage		% TOC needs to start on an odd page
    \tableofcontents

    % ---------------------------------------------------------------------- %
    % This is where the body of the report begins; usually with an Introduction
    %
    \SANDmain		% Start the main part of the report

\section{Introduction}
This document is a collection of the information provided as documentation in each individual RAVEN
input in the \xmlNode{TestInfo} block.  The tests are grouped by directories.  Tests for which there is no
information are not included.

\section{Tests}
  \subsection{/Users/talbpw/projects/raven/tests/framework/PostProcessors/BasicStatistics}
    \subsubsection{framework/PostProcessors/BasicStatistics/grid\_analytic}
      
      This test checks all the basic statistics on analytic values using a grid sampling; however, the analytic models are not
      yet documented.
    
      \begin{itemize}
          \item filename: grid\_analytic.xml
          \item classes tested: PostProcessors.BasicStatistics
          \item created: 2015-11-21
          \item author: alfoa
      \end{itemize}
    \subsubsection{framework/PostProcessors/BasicStatistics/grid\_inValue}
      
      This test checks many basic statistics using a grid sampling.
    
      \begin{itemize}
          \item filename: grid\_inValue.xml
          \item classes tested: PostProcessors.BasicStatistics
          \item created: 2015-12-16
          \item author: alfoa
      \end{itemize}
    \subsubsection{framework/PostProcessors/BasicStatistics/mc}
      
      This test checks basic statistics on Monte Carlo samples.
    
      \begin{itemize}
          \item filename: mc.xml
          \item classes tested: PostProcessors.BasicStatistics
          \item created: 2015-11-18
          \item author: alfoa
      \end{itemize}
    \subsubsection{framework/PostProcessors/BasicStatistics/grid\_analytic}
      
      This test checks basic statistics on Stratified samples
    
      \begin{itemize}
          \item filename: stratified\_analytic.xml
          \item classes tested: PostProcessors.BasicStatistics
          \item created: 2015-11-22
          \item author: alfoa
      \end{itemize}
    \subsubsection{framework/PostProcessors/BasicStatistics/factorial\_analytic}
      
      This test checks basic statistics on Factorial samples.
    
      \begin{itemize}
          \item filename: factorial\_analytic.xml
          \item classes tested: PostProcessors.BasicStatistics
          \item created: 2015-11-22
          \item author: alfoa
      \end{itemize}
    \subsubsection{framework/PostProcessors/BasicStatistics/responseSurfaceDoE\_analytic}
      
      This test checks the basic statistics on Response Surface Design of Experiment samples
    
      \begin{itemize}
          \item filename: responseSurfaceDoE\_analytic.xml
          \item classes tested: PostProcessors.BasicStatistics
          \item created: 2015-11-22
          \item author: alfoa
      \end{itemize}
    \subsubsection{framework/PostProcessors/BasicStatistics/general}
      
      This requirements test checks operation of the basic statistics postprocessor.
    
      \begin{itemize}
          \item filename: test\_BasicStatistics.xml
          \item requirements: test\_BasicStatistics.xml
          \item classes tested: PostProcessors.BasicStatistics
          \item created: 2014-05-21
          \item author: alfoa
      \end{itemize}
    \subsubsection{framework/PostProcessors/BasicStatistics/printXml}
      
      This test checks the ability of the basic statistics PP to print RAVEN XML outputs.
    
      \begin{itemize}
          \item filename: print\_xml.xml
          \item classes tested: PostProcessors.BasicStatistics
          \item created: 2016-05-10
          \item author: talbpaul
      \end{itemize}
    \subsubsection{framework/PostProcessors/BasicStatistics/sensitivity}
      
      This test checks the sensitivities (and other metrics) calculated by basic statistics PP
    
      \begin{itemize}
          \item filename: sensitivity.xml
          \item classes tested: PostProcessors.BasicStatistics
          \item created: 2015-12-10
          \item author: wangc
      \end{itemize}
    \subsubsection{framework/PostProcessors/BasicStatistics/timeDependent}
      
      This tests time-dependent basic statistics.
    
      \begin{itemize}
          \item filename: time\_dep.xml
          \item classes tested: PostProcessors.BasicStatistics
          \item created: 2016-06-20
          \item author: alfoa
      \end{itemize}
    \subsubsection{framework/PostProcessors/BasicStatistics/timeDependentAsyncHistories}
      
      This test checks time-dependent basic statistics with histories that are not synchronized a priori.
    
      \begin{itemize}
          \item filename: time\_dep\_asyncHists.xml
          \item classes tested: PostProcessors.BasicStatistics
          \item created: 2016-06-21
          \item author: alfoa
      \end{itemize}
    \subsubsection{framework/PostProcessors/BasicStatistics/variationCoefficient}
      
			This test is used to test the defect mentioned in issue \#666
			This test will make sure the expectedValue and the variationCoefficient are correctly computed.
			If the expectedValues are zero, the variationCoefficient will be INF.
    
      \begin{itemize}
          \item filename: variationCoefficient.xml
          \item classes tested: PostProcessors.BasicStatistics
          \item created: 2015-11-21
          \item author: wangc
      \end{itemize}
    \subsubsection{framework/PostProcessors/BasicStatistics/mc\_uniform}
      
      Tests basic statistics on uniformly-sampled normally-distributed Monte Carlo sampling.
    
      \begin{itemize}
          \item filename: mcUnif.xml
          \item classes tested: PostProcessors.BasicStatistics
          \item created: 2016-11-28
          \item author: mandd
      \end{itemize}
  \subsection{/Users/talbpw/projects/raven/tests/framework/Samplers/Restart}
    \subsubsection{framework/Samplers/Restart/MC}
      
      Tests restarting a Monte Carlo sampling from restart.  \texttt{makeCoarse} samples initial data, then \texttt{makeRestart}
      makes additional samples, restarting from the first set of samples.  \texttt{makeFine} does all the samples without restart
      for comparison.  The model for "coarse" always returns a value of 1, while the model for "restart" returns a value of 2, so
      you can tell which samples came from which sampling strategy.
    
      \begin{itemize}
          \item filename: test\_restart\_MC.xml
          \item classes tested: Samplers.MonteCarlo
          \item created: 2015-07-07
          \item author: talbpaul
      \end{itemize}
    \subsubsection{framework/Samplers/Restart/StochPoly}
      
      The essence of this test is to demonstrate the input space when generated from restart or from scratch
      are identical.  The external models are organized so that outputs from the restart data have an
      output value of 1.0, and from the higher-fidelity sampler have an output value of 2.0.  Obviously using
      different models with restarts is a terrible idea in general, but makes this test work.
      In the end, "fine.csv" and "restart.csv" should have identical input space, but different output space;
      all the output of "fine.csv" should be 2.0, while the restarted points in "restart.csv" should be 1.0
    
      \begin{itemize}
          \item filename: test\_restart\_stochpoly.xml
          \item classes tested: Samplers.SparseGridCollocation
          \item created: 2015-07-07
          \item author: talbpaul
      \end{itemize}
    \subsubsection{framework/Samplers/Restart/Sobol}
      
      The essence of this test is to demonstrate the input space when generated from restart or from scratch
      are identical.  The external models are organized so that outputs from the restart data have an
      output value of 1.0, and from the higher-fidelity sampler have an output value of 2.0.  Obviously using
      different models with restarts is a terrible idea in general, but makes this test work.
      In the end, "fine.csv" and "restart.csv" should have identical input space, but different output space;
      all the output of "fine.csv" should be 2.0, while the restarted points in "restart.csv" should be 1.0
    
      \begin{itemize}
          \item filename: test\_restart\_sobol.xml
          \item classes tested: Samplers.Sobol
          \item created: 2015-07-07
          \item author: talbpaul
      \end{itemize}
    \subsubsection{framework/Samplers/Restart/Grid}
      
      The essence of this test is to demonstrate the input space when generated from restart or from scratch
      are identical.  The external models are organized so that outputs from the restart data have an
      output value of 1.0, and from the higher-fidelity sampler have an output value of 2.0.  Obviously using
      different models with restarts is a terrible idea in general, but makes this test work.
      In the end, "fine.csv" and "restart.csv" should have identical input space, but different output space;
      all the output of "fine.csv" should be 2.0, while the restarted points in "restart.csv" should be 1.0
    
      \begin{itemize}
          \item filename: test\_restart\_Grid.xml
          \item classes tested: Samplers.Grid
          \item created: 2015-07-07
          \item author: talbpaul
      \end{itemize}
    \subsubsection{framework/Samplers/Restart/CSV}
      
      This test demonstrates that a restart can be performed from a loaded CSV file, not just an internal data object
      from an earlier step.  As with the other restart tests, in output data objects samples from "course" have an
      output of 1, while samples from "fine" have an output of 2.
    
      \begin{itemize}
          \item filename: test\_restart\_csv.xml
          \item classes tested: Files
          \item created: 2015-07-27
          \item author: talbpaul
      \end{itemize}
    \subsubsection{framework/Samplers/Restart/LoadFromLargeCSV}
      
      At one point there was performance issues restarting from a large amount of data
      that is loaded from a CSV.  This test exists to ensure loading from a large CSV
      behaves consistently.
    
      \begin{itemize}
          \item filename: large\_load\_from\_csv.xml
          \item classes tested: Samplers.Grid
          \item created: 2016-06-30
          \item author: talbpaul
      \end{itemize}
  \subsection{/Users/talbpw/projects/raven/tests/framework/VariableGroups}
    \subsubsection{framework/VariableGroups/SetOperations}
      
      tests set operations and data objects for using variable groups
    
      \begin{itemize}
          \item filename: sets.xml
          \item classes tested: VariableGroups
          \item created: 2016-02-08
          \item author: talbpaul
      \end{itemize}
    \subsubsection{framework/VariableGroups/ROM}
      
      tests variable groups when used as part of a ROM or external model
    
      \begin{itemize}
          \item filename: rom.xml
          \item classes tested: VariableGroups
          \item created: 2016-02-08
          \item author: talbpaul
      \end{itemize}
    \subsubsection{framework/VariableGroups/ExternalNodes}
      
      tests variable groups used in external XML
    
      \begin{itemize}
          \item filename: extnodes.xml
          \item classes tested: VariableGroups
          \item created: 2016-02-08
          \item author: talbpaul
      \end{itemize}
    \subsubsection{framework/VariableGroups/OrderedVariables}
      
      tests order preservation of variables in variable groups
    
      \begin{itemize}
          \item filename: ordered.xml
          \item classes tested: VariableGroups
          \item created: 2016-02-08
          \item author: talbpaul
      \end{itemize}
  \subsection{/Users/talbpw/projects/raven/tests/framework/ROM/Sobol}
    \subsubsection{framework/Samplers/ROM/Sobol/staticSobolRomSmolyak}
      
      This test checks the operation of the sampler and ROM together.
    
      \begin{itemize}
          \item filename: test\_sobol\_rom.xml
          \item classes tested: Samplers.Sobol,SupervisedLearning.HDMRRom
          \item created: 2016-03-03
          \item author: talbpaul
      \end{itemize}
    \subsubsection{framework/Samplers/ROM/Sobol/staticSobolRomSmolyak}
      
      This test checks the operation of the sampler and model, using the tensor sparse grid.
    
      \begin{itemize}
          \item filename: test\_sobol\_tensor.xml
          \item classes tested: Samplers.Sobol,SupervisedLearning.HDMRRom
          \item created: 2015-07-07
          \item author: talbpaul
      \end{itemize}
    \subsubsection{framework/Samplers/ROM/Sobol/sobolSudretAnalytic}
      
      This analytic test checks the performance of HDMRRom against the simple Sudret polynomial.
    

      
      dumprom.xml has analytic mean and variance, and is documented in the analytic test documentation
      under "Global Sobol Sensitivity: Sudret"
    
      \begin{itemize}
          \item filename: test\_sobol\_sudret.xml
          \item classes tested: SupervisedLearning.HDMRRom
          \item created: 2016-03-09
          \item author: talbpaul
      \end{itemize}
    \subsubsection{framework/Samplers/ROM/Sobol/sobolIshigamiAnalytic}
      
      This analytic test checks the performance of HDMRRom against the sinusoidal Ishigami function.
    

      
      dumprom.xml has analytic mean and variance, and is documented in the analytic test document
      under "Global Sobol Sensitivity: Ishigami".
    
      \begin{itemize}
          \item filename: test\_sobol\_ishigami.xml
          \item classes tested: SupervisedLearning.HDMRRom
          \item created: 2016-03-08
          \item author: talbpaul
      \end{itemize}
    \subsubsection{framework/Samplers/ROM/Sobol/sobolGFunction}
      
      This analytic test checks the performance of HDMRRom against the discontinuous Sobol G-Function.
    

      
      dumprom.xml has analytic mean and variance, but is poorly converged for this model.  It is documented
      in the analytic test documentation under "Sobol G-Function"
    
      \begin{itemize}
          \item filename: test\_sobol\_gfunc.xml
          \item classes tested: SupervisedLearning.HDMRRom
          \item created: 2016-03-08
          \item author: talbpaul
      \end{itemize}
    \subsubsection{framework/Samplers/ROM/Sobol/AnovaOnCutHDMR}
      
      This analytically tests calculating variance using ANOVA on cut-HDMR.
    

      
      This test is analytic in the variance, Sobol sensitivities, and mean of the response.  These parameters
      are documented in the analytic tests documentation under "Second-Order ANOVA of Second-Order Cut-HDMR Expansion of Sudret".
    
      \begin{itemize}
          \item filename: test\_anova\_cut.xml
          \item classes tested: SupervisedLearning.HDMRRom
          \item created: 2016-03-11
          \item author: talbpaul
      \end{itemize}
    \subsubsection{framework/Samplers/ROM/Sobol/AdaptiveSobol}
      
      This tests using the AdaptiveSobol sampler to construct HDMRRom ROMs.
    

      
      dumprom.xml has analytic results for mean and variance in that are documented in the Attenuation
      section of the analytic tests manual.
    
      \begin{itemize}
          \item filename: test\_adapt\_sobol.xml
          \item classes tested: Samplers.AdaptiveSobol,SupervisedLearning.HDMRRom
          \item created: 2016-02-08
          \item author: talbpaul
      \end{itemize}
    \subsubsection{framework/Samplers/ROM/Sobol/AdaptiveSobolMaxRuns}
      
      This tests using the AdaptiveSobol sampler to construct HDMRRom ROMs, but restricted by
      a specified number of samples (maxRuns in the sampler).
    
      \begin{itemize}
          \item filename: test\_adapt\_sobol\_maxruns.xml
          \item classes tested: Samplers.AdaptiveSobol,SupervisedLearning.HDMRRom
          \item created: 2016-02-08
          \item author: talbpaul
      \end{itemize}
    \subsubsection{framework/Samplers/ROM/Sobol/AdaptiveSobolParallel}
      
      This tests using the AdaptiveSobol sampler to construct HDMRRom ROMs, in parallel.
    
      \begin{itemize}
          \item filename: test\_adapt\_sobol\_parallel.xml
          \item classes tested: Samplers.AdaptiveSobol,SupervisedLearning.HDMRRom
          \item created: 2016-02-08
          \item author: talbpaul
      \end{itemize}
    \subsubsection{framework/Samplers/ROM/Sobol/TimeDependent}
      
      This test checks the construction of a time-dependent HDMRRom.
    
      \begin{itemize}
          \item filename: test\_time\_dep\_sobol.xml
          \item classes tested: SupervisedLearning.HDMRRom
          \item created: 2016-03-09
          \item author: talbpaul
      \end{itemize}
    \subsubsection{framework/Samplers/ROM/Sobol/verifyHDMRRom}
      
      This analytic test checks that the time-dependent HDMRRom performs the same as the model it's representing.
    

      
      This test uses "projectile.py" as documented in the analytic tests document, and makes several evaluations
      of position based on time.
    
      \begin{itemize}
          \item filename: verify\_time\_dep\_sobol.xml
          \item classes tested: SupervisedLearning.HDMRRom
          \item created: 2016-11-08
          \item author: talbpaul
      \end{itemize}
  \subsection{/Users/talbpw/projects/raven/tests/framework/ROM}
    \subsubsection{framework/Samplers/ROM/timeDepRom}
      
      This tests using a general time-dependent reduced-order model.
    
      \begin{itemize}
          \item filename: test\_t\_rom.xml
          \item classes tested: Models.ROM
          \item created: 2015-11-30
          \item author: talbpaul
      \end{itemize}
    \subsubsection{framework/Samplers/ROM/timeDepGuassPoly}
      
      This tests using a time-dependent GaussPolynomialROM.
    
      \begin{itemize}
          \item filename: test\_time\_dep\_scgpc.xml
          \item classes tested: SupervisedLearning.GaussPolynomialROM
          \item created: 2016-03-09
          \item author: talbpaul
      \end{itemize}
    \subsubsection{framework/Samplers/ROM/verifyGaussPolyRom}
      
      This tests validates the time-dependent GaussPolynomialROM by sampling it and comparing to the original model.
    

      
      This test uses the "projectile.py" ballistic model and tracks position in time.  The evaluations of this model
      as well as the ROMs should match the results documented there.
    
      \begin{itemize}
          \item filename: verify\_time\_scgpc.xml
          \item classes tested: SupervisedLearning.GaussPolynomialROM
          \item created: 2016-03-09
          \item author: talbpaul
      \end{itemize}
  \subsection{/Users/talbpw/projects/raven/tests/framework/Samplers/Restart/Truncated}
    \subsubsection{framework/Samplers/Restart/Truncated/Grid}
      
      This is similar to the restart tests in the parent directory, but in this one we test the use of the
      restartTolerance to recover restart points from a code that produces finite precision when reporting input
      values.  As with the other restart tests, "coarse" returns a 1 and "fine" returns a 2.
    
      \begin{itemize}
          \item filename: grid.xml
          \item classes tested: Samplers.Grid
          \item created: 2016-04-05
          \item author: talbpaul
      \end{itemize}
    \subsubsection{framework/Samplers/Restart/Truncated/SparseGridCollocation}
      
      This is similar to the restart tests in the parent directory, but in this one we test the use of the
      restartTolerance to recover restart points from a code that produces finite precision when reporting input
      values.  As with the other restart tests, "coarse" returns a 1 and "fine" returns a 2.
    
      \begin{itemize}
          \item filename: sparse\_grid\_collocation.xml
          \item classes tested: Samplers.SparseGridCollocation
          \item created: 2016-04-05
          \item author: @talbpaul
      \end{itemize}
    \subsubsection{framework/Samplers/Restart/Truncated/Sobol}
      
      This is similar to the restart tests in the parent directory, but in this one we test the use of the
      restartTolerance to recover restart points from a code that produces finite precision when reporting input
      values.  As with the other restart tests, "coarse" returns a 1 and "fine" returns a 2.
    
      \begin{itemize}
          \item filename: sobol.xml
          \item classes tested: Samplers.Sobol
          \item created: 2016-04-05
          \item author: @talbpaul
      \end{itemize}
    \subsubsection{framework/Samplers/Restart/Truncated/Scaled}
      
      One challenge for the restartTolerance in restarting a sampler is finding a matching point in the restart data
      when the scale of the various inputs is wildly different.  This test assures such a restart can be performed.
      See parent directory for how restarts are tested.
    
      \begin{itemize}
          \item filename: scale\_test.xml
          \item classes tested: Samplers
          \item created: 2016-10-26
          \item author: @talbpaul
      \end{itemize}
  \subsection{/Users/talbpw/projects/raven/tests/framework}
    \subsubsection{framework/2steps\_same\_db}
      
        This  test is aimed to test the capability of the RAVEN database strucuture to use the same database (HDF5) for
        subsequential analyses in order to collect all the results in the same HDF5 container
    
      \begin{itemize}
          \item filename: test\_2steps\_same\_db.xml
          \item classes tested: Databases.HDF5
          \item created: 2015-05-01
          \item author: @alfoa
      \end{itemize}
    \subsubsection{framework/testAdaptiveDynamicEventTreeRAVEN}
      
        This  test is aimed to test the capability of RAVEN to employ the Adaptive Dynamic Event Tree Sampling strategy (using RELAP7 as system code)
    
      \begin{itemize}
          \item filename: test\_adaptive\_det\_simple.xml
          \item classes tested: Samplers.AdaptiveDET
          \item created: 2015-01-26
          \item author: @alfoa
      \end{itemize}
    \subsubsection{framework/testAdaptiveHybridDynamicEventTreeRAVEN}
      
        This  test is aimed to test the capability of RAVEN to employ the Adaptive Hybrid Dynamic Event Tree Sampling strategy (using RELAP7 as system code)
    
      \begin{itemize}
          \item filename: test\_adaptive\_hybrid\_det.xml
          \item classes tested: Samplers.AdaptiveDET
          \item created: 2015-01-26
          \item author: @alfoa
      \end{itemize}
    \subsubsection{framework/adaptive\_sampler\_ext\_model}
      
        This  test is aimed to test the capability of RAVEN to employ a goal oriented sampling. It tests the
        LimitSurfaceSearch algorithm using an external model as ``system code''
    
      \begin{itemize}
          \item filename: test\_adaptive\_sampler.xml
          \item classes tested: Samplers.LimitSurfaceSearch, Models.ExternalModel, Models.ROM
          \item created: 2015-04-09
          \item author: @alfoa
      \end{itemize}
    \subsubsection{framework/stochPolyPickleTest}
      
      This test checks the pickling and unpickling of the GaussPolynomialROM
    
      \begin{itemize}
          \item filename: test\_stochpoly\_pickle.xml
          \item classes tested: SupervisedLearning.GaussPolynomialROM
          \item created: 2015-07-07
          \item author: talbpaul
      \end{itemize}
    \subsubsection{framework/stochPolyInterpTest}
      
      This test checks the use of the "interpolation" blocks for the SCgPC methodology.
    
      \begin{itemize}
          \item filename: test\_stochpoly\_interp.xml
          \item classes tested: Samplers.SparseGridCollocation,SupervisedLearning.GaussPolynomialROM
          \item created: 2015-07-07
          \item author: talbpaul
      \end{itemize}
  \subsection{/Users/talbpw/projects/raven/tests/framework/Samplers/SparseGrid}
    \subsubsection{framework/Samplers/SparseGrid/SmolyakGridTest}
      
      This tests using Smolyak-style SparseGridCollocation.
    
      \begin{itemize}
          \item filename: test\_sparse\_grid.xml
          \item classes tested: Samplers.SparseGridCollocation
          \item created: 2015-07-07
          \item author: talbpaul
      \end{itemize}
    \subsubsection{framework/Samplers/SparseGrid/TensorGridTest}
      
      This tests using non-sparse tensor collocation.
    
      \begin{itemize}
          \item filename: test\_tensor\_grid.xml
          \item classes tested: Samplers.SparseGridCollocation
          \item created: 2015-07-07
          \item author: talbpaul
      \end{itemize}
    \subsubsection{framework/Samplers/SparseGrid/tensorPrductGrid}
      
      This tests the creation of a TensorProduct construction sparse grid.
    
      \begin{itemize}
          \item filename: test\_index\_TP.xml
          \item classes tested: IndexSets.TensorProduct
          \item created: 2015-09-11
          \item author: talbpaul
      \end{itemize}
    \subsubsection{framework/Samplers/SparseGrid/totalDegreeGrid}
      
      This tests the creation of a TotalDegree construction sparse grid.
    
      \begin{itemize}
          \item filename: test\_index\_TD.xml
          \item classes tested: IndexSets.TotalDegree
          \item created: 2015-09-11
          \item author: talbpaul
      \end{itemize}
    \subsubsection{framework/Samplers/SparseGrid/hyperbolicCrossGrid}
      
      This tests the creation of a HyperbolicCross construction sparse grid.
    
      \begin{itemize}
          \item filename: test\_index\_HC.xml
          \item classes tested: IndexSets.HyperbolicCross
          \item created: 2015-09-11
          \item author: talbpaul
      \end{itemize}
    \subsubsection{framework/Samplers/SparseGrid/customGrid}
      
      This tests the creation of a Custom sparse grid.
    
      \begin{itemize}
          \item filename: test\_index\_custom.xml
          \item classes tested: IndexSets.Custom
          \item created: 2015-09-11
          \item author: talbpaul
      \end{itemize}
    \subsubsection{framework/Samplers/SparseGrid/anisotropicGrid}
      
      This tests the creation of an anisotropic sparse grid (using the interpolation weights in the ROM).
    
      \begin{itemize}
          \item filename: test\_index\_anisotropic.xml
          \item classes tested: IndexSets
          \item created: 2015-09-11
          \item author: talbpaul
      \end{itemize}
    \subsubsection{framework/Samplers/SparseGrid/uniform}
      
      This tests using SparseGridCollocation with uniformly-distributed inputs.
    
      \begin{itemize}
          \item filename: test\_scgpc\_uniform.xml
          \item classes tested: Samplers.SparseGridCollocation
          \item created: 2015-09-11
          \item author: talbpaul
      \end{itemize}
    \subsubsection{framework/Samplers/SparseGrid/uniform\_cc}
      
      This tests using SparseGridCollocation with Clenshaw Curtis points and weights (on uniformly-distributed variables).
    
      \begin{itemize}
          \item filename: test\_scgpc\_uniform\_cc.xml
          \item classes tested: Samplers.SparseGridCollocation
          \item created: 2015-09-11
          \item author: talbpaul
      \end{itemize}
    \subsubsection{framework/Samplers/SparseGrid/normal}
      
      This tests using SparseGridCollocation with normally-distributed inputs.
    
      \begin{itemize}
          \item filename: test\_scgpc\_normal.xml
          \item classes tested: Samplers.SparseGridCollocation
          \item created: 2015-09-11
          \item author: talbpaul
      \end{itemize}
    \subsubsection{framework/Samplers/SparseGrid/gamma}
      
      This tests using SparseGridCollocation with gamma-distributed inputs.
    
      \begin{itemize}
          \item filename: test\_scgpc\_gamma.xml
          \item classes tested: Samplers.SparseGridCollocation
          \item created: 2015-09-11
          \item author: talbpaul
      \end{itemize}
    \subsubsection{framework/Samplers/SparseGrid/beta}
      
      This tests using SparseGridCollocation with beta-distributed inputs.
    
      \begin{itemize}
          \item filename: test\_scgpc\_beta.xml
          \item classes tested: Samplers.SparseGridCollocation
          \item created: 2015-09-11
          \item author: talbpaul
      \end{itemize}
    \subsubsection{framework/Samplers/SparseGrid/betanorm}
      
      This tests using SparseGridCollocation with truncated-normal-beta distributed inputs.
    
      \begin{itemize}
          \item filename: test\_scgpc\_betanorm.xml
          \item classes tested: Samplers.SparseGridCollocation
          \item created: 2015-09-11
          \item author: talbpaul
      \end{itemize}
    \subsubsection{framework/Samplers/SparseGrid/triang}
      
      This tests using SparseGridCollocation with triangular-distributed inputs.
    
      \begin{itemize}
          \item filename: test\_scgpc\_triang.xml
          \item classes tested: Samplers.SparseGridCollocation
          \item created: 2015-09-11
          \item author: talbpaul
      \end{itemize}
    \subsubsection{framework/Samplers/SparseGrid/exponential}
      
      This tests using SparseGridCollocation with exponential-distributed inputs.
    
      \begin{itemize}
          \item filename: test\_scgpc\_expon.xml
          \item classes tested: Samplers.SparseGridCollocation
          \item created: 2015-09-11
          \item author: talbpaul
      \end{itemize}
    \subsubsection{framework/Samplers/SparseGrid/AdaptiveOnVariance}
      
      This tests the adaptive sparse grid with adaptive samples chosen and converged according to variance.
    
      \begin{itemize}
          \item filename: test\_adaptive\_stochpoly\_var.xml
          \item classes tested: Samplers.AdaptiveSparseGrid
          \item created: 2015-07-07
          \item author: talbpaul
      \end{itemize}
  \subsection{/Users/talbpw/projects/raven/tests/framework/ROM/SparseGrid}
    \subsubsection{framework/Samplers/ROM/Sobol/SparseGrid/scgpcSudretAnalytic}
      
      This analytic test checks the performance of HDMRRom against the simple Sudret polynomial
    

      
      dumprom.xml has analytic mean and variance, documented in the analytic tests documentation under
      "Global Sobol Sensitivity: Sudret".
    
      \begin{itemize}
          \item filename: test\_scgpc\_sudret.xml
          \item classes tested: SupervisedLearning.GaussPolynomialROM
          \item created: 2016-03-18
          \item author: talbpaul
      \end{itemize}
  \subsection{/Users/talbpw/projects/raven/tests/framework/PostProcessors/RavenOutputPostProcessor}
    \subsubsection{framework/PostProcessors/RavenOutputPostProcessor/static}
      
      This test checks using the RavenOutput postprocessor to read multiple RAVEN XML output files.
      Creates both a BasicStatistics and collocation ROM dump XML for loading, then loads comparable values from them.
    
      \begin{itemize}
          \item filename: test\_notime.xml
          \item classes tested: PostProcessors.RavenOutput
          \item created: 2016-05-20
          \item author: talbpaul
      \end{itemize}
    \subsubsection{framework/PostProcessors/RavenOutputPostProcessor/dynamic}
      
      This test checks using the RavenOutput postprocessor to read a dynamic RAVEN XML output file.
    
      \begin{itemize}
          \item filename: test\_time.xml
          \item classes tested: PostProcessors.RavenOutput
          \item created: 2016-06-22
          \item author: talbpaul
      \end{itemize}
  \subsection{/Users/talbpw/projects/raven/tests/framework/PostProcessors/TopologicalPostProcessor}
    \subsubsection{framework/PostProcessors/TopologicalPostProcessor.topology\_simple}
      
       A simple example of the approximate Morse-Smale complex (AMSC) interface
       using a test function consisting of a 2D Gaussian with a single maximum
       and 4 local minimum occurring at the corners of the domain space. The
       hill in the middle is purposefully off-centered so as to create
       non-uniform cases at each corner allowing us to simplify each local
       minimum in turn to create a 4 partition case, a 3 partition case, a 2
       partition case, and a single partition case.
    
      \begin{itemize}
          \item filename: test\_topology\_simple.xml
          \item classes tested: PostProcessors.TopologicalDecomposition
          \item created: 2015-09-02
          \item author: maljdan
      \end{itemize}
    \subsubsection{framework/PostProcessors/TopologicalPostProcessor.persistence\_count}
      
       A simple example of the approximate Morse-Smale complex (AMSC) interface
       where select a persistence level using the count of points in each
       partition, that is if a partition has fewer than x points, it will be
       merged into a neighboring partition. Note, that all of the
       "persistence" tests will use the Schwefel function for testing.
    
      \begin{itemize}
          \item filename: test\_persistence\_count.xml
          \item classes tested: PostProcessors.TopologicalDecomposition
          \item created: 2015-09-21
          \item author: maljdan
      \end{itemize}
    \subsubsection{framework/PostProcessors/TopologicalPostProcessor.persistence\_prob}
      
       A simple example of the approximate Morse-Smale complex (AMSC) interface
       where we select a persistence level based on a minimum amount of
       probability. That is, if the total probability weight of a partition is
       too small, then that partition will be merged into a neighboring
       partition given the normal rules of persistence simplification. Note,
       that all of the "persistence" tests will use the Schwefel function for
       testing.
    
      \begin{itemize}
          \item filename: test\_persistence\_prob.xml
          \item classes tested: PostProcessors.TopologicalDecomposition
          \item created: 2015-09-21
          \item author: maljdan
      \end{itemize}
    \subsubsection{framework/PostProcessors/TopologicalPostProcessor.persistence\_value}
      
       A simple example of the approximate Morse-Smale complex (AMSC) interface
       where we select a persistence level based on the standard notion of the
       function value difference between the extrema and its nearest valued
       neighboring critical point. That is, larger peaks and valleys will
       persist longer than smaller "noisy" features. This is the standard
       definition of persistence used in Morse topology. Note, that all of the
       "persistence" tests will use the Schwefel function for testing.
    
      \begin{itemize}
          \item filename: test\_persistence\_value.xml
          \item classes tested: PostProcessors.TopologicalDecomposition
          \item created: 2015-09-21
          \item author: maljdan
      \end{itemize}
    \subsubsection{framework/PostProcessors/TopologicalPostProcessor.persistence\_full}
      
       A simple example of the approximate Morse-Smale complex (AMSC) interface
       where no persistence simplification will be done. Note, that all of the
       "persistence" tests will use the Schwefel function for testing.
    
      \begin{itemize}
          \item filename: test\_persistence\_full.xml
          \item classes tested: PostProcessors.TopologicalDecomposition
          \item created: 2015-09-21
          \item author: maljdan
      \end{itemize}
    \subsubsection{framework/PostProcessors/TopologicalPostProcessor.knn}
      
       A simple example of the approximate Morse-Smale complex (AMSC) interface
       that exercises the k-nearest neighbor graph structure as the underlying
       connectivity model for the point cloud. Note, each of the "graph" test
       cases uses the GaussianPeaks test case which consists of 4 distinctly
       shaped Gaussian peaks in a 2D input domain.
    
      \begin{itemize}
          \item filename: test\_graph\_knn.xml
          \item classes tested: PostProcessors.TopologicalDecomposition
          \item created: 2015-09-21
          \item author: maljdan
      \end{itemize}
    \subsubsection{framework/PostProcessors/TopologicalPostProcessor.relaxed\_beta\_skeleton}
      
       A simple example of the approximate Morse-Smale complex (AMSC) interface
       that exercises the beta skeleton graph structure as the underlying
       connectivity model for the point cloud. Note, each of the "graph" test
       cases uses the GaussianPeaks test case which consists of 4 distinctly
       shaped Gaussian peaks in a 2D input domain.
    
      \begin{itemize}
          \item filename: test\_graph\_bs.xml
          \item classes tested: PostProcessors.TopologicalDecomposition
          \item created: 2015-09-21
          \item author: maljdan
      \end{itemize}
    \subsubsection{framework/PostProcessors/TopologicalPostProcessor.relaxed\_beta\_skeleton}
      
       A simple example of the approximate Morse-Smale complex (AMSC) interface
       that exercises the relaxed beta skeleton graph structure as the
       underlying connectivity model for the point cloud. Note, each of the
       "graph" test cases uses the GaussianPeaks test case which consists of 4
       distinctly shaped Gaussian peaks in a 2D input domain.
    
      \begin{itemize}
          \item filename: test\_graph\_rbs.xml
          \item classes tested: PostProcessors.TopologicalDecomposition
          \item created: 2015-09-21
          \item author: maljdan
      \end{itemize}
  \subsection{/Users/talbpw/projects/raven/tests/framework/Samplers/Sobol}
    \subsubsection{framework/Samplers/Sobol/Sobol}
      
      This tests using the Sobol static sampler with the basic syntax.
    
      \begin{itemize}
          \item filename: test\_sobol\_sampler.xml
          \item classes tested: Samplers.Sobol
          \item created: 2015-09-14
          \item author: talbpaul
      \end{itemize}

  \section{Document Version Information}
  \input{../version.tex}
%\end{appendices}


    % ---------------------------------------------------------------------- %
    % References
    %
    %\clearpage
    % If hyperref is included, then \phantomsection is already defined.
    % If not, we need to define it.
    %\providecommand*{\phantomsection}{}
    %\phantomsection
    %\addcontentsline{toc}{section}{References}
    %\bibliographystyle{ieeetr}
    %\bibliography{analytic_tests}

\end{document}
