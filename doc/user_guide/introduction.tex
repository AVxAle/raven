\section{Introduction}

% High-level RAVEN description
RAVEN is a software framework able to perform parametric and stochastic
analysis based on the response of complex system codes.
%
The initial development was aimed at providing dynamic risk analysis
capabilities to the thermohydraulic code RELAP-7, currently under development
at Idaho National Laboratory (INL).
%
Although the initial goal has been fully accomplished, RAVEN is now a
multi-purpose stochastic and uncertainty quantification platform, capable of communicating with any system code.

% RAVEN's flexibility
In fact, the provided Application Programming
Interfaces (APIs) allow RAVEN to interact with any code as long as all the parameters
that need to be perturbed are accessible by input files or via python
interfaces.
%
%RAVEN is capable of investigating the system response, and investigating the
%input space using Monte Carlo, Grid, or Latin Hyper Cube sampling schemes, but
%its strength is focused towards system feature discovery, such as limit
%surfaces, separating regions of the input space leading to system failure, and
%using dynamic supervised learning techniques.
RAVEN is capable of investigating system response and explore input space using various
sampling schemes such as Monte Carlo, grid, or Latin Hypercube.
%
However, RAVEN strength lies in its system feature discovery capabilities such as: constructing
limit surfaces, separating regions of the input space leading to system failure,
and using dynamic supervised learning techniques.

% Historical Context
The development of RAVEN started in 2012 when, within the Nuclear Energy
Advanced Modeling and Simulation (NEAMS) program, the need to provide a modern
risk evaluation framework arose.
%
RAVEN's principal assignment is to provide the necessary software and algorithms
in order to employ the concepts developed by the Risk Informed Safety Margin
Characterization (RISMC) program.
%
RISMC is one of the pathways defined within the Light Water Reactor
Sustainability (LWRS) program.

% RISMC
In the RISMC approach, the goal is not just to identify the frequency of an
event potentially leading to a system failure, but the proximity (or lack
thereof) to key safety-related events.
%
Hence, the approach is interested in identifying and increasing the safety
margins related to those events.
%
A safety margin is a numerical value quantifying the probability that a safety
metric (e.g. peak pressure in a pipe) is exceeded under certain conditions.

% Conclusion

%The initial development of RAVEN has been focused on providing dynamic risk
%assessment capability to RELAP-7, currently under development at INL.
%
Most of the capabilities, implemented having RELAP-7 as a principal focus, are
easily deployable to other system codes.
%
For this reason, several side activates have been employed (e.g.  RELAP5-3D, any MOOSE-based App, etc.)
or are currently ongoing for coupling RAVEN with several different software.
%
The aim of this document is to provide a set of commented examples that can help the user to become familiar 
with the RAVEN code usage.
