\documentclass{article}
\begin{document}
\textbf{pbsdsh}: this ``mode'' uses the pbsdsh protocol to distribute
the program running; more information regarding this protocol can be found
in PBS documentation.
Mode ``pbsdsh'' automatically knows when it needs to generate a
      \texttt{qsub} command, by inquiring the machine enviroment:
         \begin{itemize}
           \item If RAVEN is executed in the HEAD node of an HPC system, RAVEN
             generates a \texttt{qsub} command, and instantiates and submits
             itself to the queue system.
           \item If the user decides to execute RAVEN from an ``interactive
             node'' (a certain number of nodes that have been reserved in
             interactive PBS mode), RAVEN, using the ``pbsdsh'' system, is going
             to utilize the reserved resources (CPUs and nodes) to distribute
             the jobs, but will not generate a \texttt{qsub} command.
         \end{itemize}
         In addition, this flag activates the remote (PBS) execution of internal Models (e.g. ROMs,
         ExternalModels, PostProcessors, etc.). If this node is not present, the internal Models
           are run using a multi-threading approach (i.e. master processor, multiple parallel threads)

\end{document}
