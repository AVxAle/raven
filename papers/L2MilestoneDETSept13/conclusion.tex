\label{sec:conclusions}
The milestone of developing the Dynamic Event Tree methodology has been fully achieved. The Dynamic Event Tree module implemented in RAVEN is fully functional and makes RAVEN a state-of-art PRA simulation tool. In addition a framework (part of a bigger general framework~\cite{RAVENFY13}) that makes this results replicable and stable has been generated. The framework is capable to deploy the DET and handle the data post processing. Data are currently stored in HDF5 compressed format, which is a general portable format supported in almost all computing architectures~\cite{HDF5}. 
The DET framework still needs to be integrated within the RAVEN/Peacock GUI, in order to show in real time some important information about the DET calculation (Branches' status, associated conditional probabilities, etc.).

As already mentioned, the DET module made RAVEN become a state-of-art PRA tool. In the near future, an adaptive DET methodology is going to be developed, in order to investigate the low risk zone in the phase space.
  
