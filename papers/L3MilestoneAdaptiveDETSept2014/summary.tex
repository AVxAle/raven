% Insert Text

Conventional \textbf{E}vent-\textbf{T}ree (\textbf{ET}) based methodologies are extensively used as tools to perform reliability and safety assessment of complex and critical engineering systems.
One of the disadvantages of these methods is that timing/sequencing of events and system dynamics are not explicitly accounted for in the analysis.
In order to overcome these limitations several techniques, also know as \textbf{D}ynamic \textbf{P}robabilistic \textbf{R}isk \textbf{A}ssessment (\textbf{DPRA}), have been developed. \textbf{M}onte-\textbf{C}arlo (MC) and \textbf{D}ynamic \textbf{E}vent \textbf{T}ree (\textbf{DET}) are two of the most widely used D-PRA methodologies to perform safety assessment of \textbf{N}uclear \textbf{P}ower \textbf{P}lants (NPP).
\\In the past two years, the Idaho National Laboratory (INL) has developed its own tool to perform Dynamic PRA: \textbf{RAVEN} (\textbf{R}eactor \textbf{A}nalysis and \textbf{V}irtual control \textbf{EN}vironment).
\\RAVEN has been designed to perform two main tasks: 1) control logic driver for the new Thermo-Hydraulic code RELAP-7 and 2) post-processing tool.
In the first task, RAVEN acts as a deterministic controller in which the set of control logic laws (user defined) monitors the RELAP-7 simulation and controls the activation of specific systems.
Moreover, the control logic infrastructure is used to model stochastic events, such as components failures, and perform uncertainty propagation. Such stochastic modeling is deployed using both MC and DET algorithms. In the second task, RAVEN processes the large amount of data generated by RELAP-7 using data-mining based algorithms. This report focuses on the analysis of dynamic stochastic systems using the newly developed RAVEN DET capability. As an example, a DPRA analysis, using DET, of a simplified pressurized water reactor for a Station Black-Out (SBO) scenario is presented.

