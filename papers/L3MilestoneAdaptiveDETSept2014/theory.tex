\label{sec:theory}
\subsection{Mathematical Framework}
\label{sec:mathFramework}
Let be $\bar{\theta}(t)$ a vector describing the plant status in the phase space; the dynamic of the plant, including the control system, can be summarized by the following equation~\cite{MathFrameworkMC2013}:
\begin{equation}
\frac{\partial \bar{\theta}}{\partial t} = \bar{H}(\theta(t),t)
\label{eq:SystemDynamics}
\end{equation}
In the above equation it is assumed the time differentiability in the phase space. Performing an arbitrary decomposition of the phase space, the following statement is obtained:
\begin{equation}
\bar{\theta}=\binom{\bar{x}}{\bar{v}}
\label{eq:firstDecomposition}
\end{equation}
The decomposition is made in such a way that $\bar{x}$ represents the unknowns solved by RELAP-7, while $\bar{v}$ are the variables directly controlled by the control system (i.e., RAVEN). Equation~\ref{eq:SystemDynamics} can now be rewritten as follows:
\begin{equation}
\begin{cases}
\dfrac{\partial \bar{x}}{\partial t} = \bar{F}(\bar{x},\bar{v},t) \\
\dfrac{\partial \bar{v}}{\partial t} = \bar{V}(\bar{x},\bar{v},t) \\
\end{cases}
\label{eq:generalSystemEquation}
\end{equation}
Note that the function $\bar{V}(\bar{x},\bar{v},t)$ representing the control system, does not depend on the knowledge of the complete status of the system but on a restricted subset (i.e. control variables) $\bar{C}$:
%Why is V still a function of x and v?
\begin{equation}
\begin{cases}
\dfrac{\partial \bar{x}}{\partial t} = \bar{F}(\bar{x},\bar{v},t) \\
\bar{C} = \bar{G}(\bar{x},t) \\
\dfrac{\partial \bar{v}}{\partial t} = \bar{V}(\bar{x},\bar{v},t)
\end{cases}
\label{eq:generalSystemEquationwithControl}
\end{equation}
The system of equations in Eq.~\ref{eq:generalSystemEquationwithControl} is fully coupled and has commonly been solved by an operator splitting approach. The reasons for this choice are several:
\begin{itemize}
\item Control system reacts with an intrinsic delay
\item The reaction of the control system might move the system between two different discrete states and
therefore numerical errors will be always of first order unless the discontinuity is treated explicitly.
\end{itemize}
RAVEN uses this approach to solve Eq.~\ref{eq:generalSystemEquationwithControl} which now becomes:
\begin{equation}
\begin{cases}
\dfrac{\partial \bar{x}}{\partial t} = \bar{F}(\bar{x},\bar{v}_{t_{i-1}},t) \\
\bar{C} = \bar{G}(\bar{x},t) & t_{i-1}\leq t\leq t_{i} = t_{i-1} + \Delta t_{i}\\
\dfrac{\partial \bar{v}}{\partial t} = \bar{V}(\bar{x},\bar{v}_{t_{i-1}},t) \\
\end{cases}
\label{eq:generalSystemEquationwithControlSplitting}
\end{equation}
Even if all information needed is contained in $\bar{x}$ and $\bar{v}$, it is not often practical and efficient to implement the control logic for complex system . Consequently, a system of auxiliary variables has been introduced.

The auxiliary variables are those that in statistical analysis are artificially added, when possible, to non-Markovian systems into the space phase to obtain a Markovian behavior back, so that only the information of the previous time step is needed to determine the future status of the system.
Thus, the introduction of the auxiliary variables into the mathematical framework leads to the following formulation of Eq.~\ref{eq:generalSystemEquationwithControlSplitting}:
\vspace{-2mm}
\begin{equation}
\begin{cases}
\dfrac{\partial \bar{x}}{\partial t} = \bar{F}(\bar{x},\bar{v}_{t_{i-1}},t) \\
\bar{C} = \bar{G}(\bar{x},t) & t_{i-1}\leq t\leq t_{i} = t_{i-1} + \Delta t_{i}\\
\dfrac{\partial \bar{a}}{\partial t} = \bar{A}(\bar{x},\bar{C},\bar{a}_{t_{i-1}},\bar{v}_{t_{i-1}},t) \\
\dfrac{\partial \bar{v}}{\partial t} = \bar{V}(\bar{x},\bar{C},\bar{v}_{t_{i-1}},\bar{a},t)
\end{cases}
\label{eq:generalSystemEquationwithControlSplittingAndAux}
\end{equation}

\subsection{Software Overview}
RAVEN~\cite{alfonsiMC}, is plugged into the software environment MOOSE~\cite{MOOSE}. MOOSE is a computer simulation framework,  developed at Idaho National Laboratory (INL),which simplifies the process for predicting the behavior of complex systems and developing non-linear, multi-physics simulation tools. MOOSE provides the algorithms for the solution of generic partial differential equations, and all the manipulation tools needed to extract information from the solution field.  This framework has been used to construct and develop the Thermal-Hydraulic code RELAP-7, giving an enormous flexibility in the coupling procedure with RAVEN.

RELAP-7 is the next generation nuclear reactor system safety analysis. It will become the main reactor systems simulation toolkit for the RISMC (\textbf{R}isk \textbf{I}nformed \textbf{S}afety \textbf{M}argin \textbf{C}haracterization)~\cite{mandelliANS_RISMC} project and the next generation tool in the RELAP reactor safety/systems analysis application series.

RAVEN has been developed in a highly modular and pluggable way in order to enable easy integration of different programming languages (i.e., \verb!C++!, \verb!Python!) and coupling with other MOOSE based applications and not only. The code consists of four main modules:
\vspace{-5mm}
\begin{itemize}
\itemsep0em
\item RAVEN/RELAP-7 interface
\item Python Control Logic
\item External Python Manager
\item Graphical User Interface
\end{itemize}
\vspace{-5mm}
The RAVEN/RELAP-7 interface, coded in \verb!C++!, is the container of all the tools needed to interact with RELAP-7/MOOSE. It has been designed in order to be general and pluggable with different solvers simultaneously in order to allow an easier and faster development of the control logic/PRA capabilities for multi physics applications.
The interface provides all the capabilities to extract the monitored quantities and accordingly modify the controlled parameters in the RELAP-7/MOOSE calculation.

The control logic module is used to drive a RAVEN/RELAP-7 simulation, as the real plant control system would do. It is implemented by the user via \verb!Python! scripting. The reason of this choice is to try to preserve generality of the approach in the initial phases of the project so that further specialization (pre-generated control logic blocks)  is possible and  inexpensive. The implementation of the control logic via \verb!Python! is rather convenient and flexible. The user only needs to know few \verb!Python! syntax rules in order to build an input. Though simple by itself, the GUI will provide tools to automate the construction of the control logic scripting in order to minimize user effort.

The core of PRA analysis is contained in an external driver/manager. It consists of a \verb!Python! framework that contains the capabilities and interfaces to drive a PRA analysis. Its basic infrastructure in connection with the DET module will be discussed in section~\ref{sec:CPUInfrastructure}

As previously mentioned, a GUI is not required to run RAVEN, but it represents an added value to the whole code. The GUI is compatible with all the capabilities actually available in RAVEN.  Its development is performed using \verb!PyQt4!, which is a \verb!Python! interface for \verb!Qt! (https://qt-project.org). \verb!Qt!  is a popular library used for cross-platform GUI development. RAVEN’s GUI is developed starting from Peacock, which is a GUI interface for generic MOOSE based applications. Because RELAP-7 is rather different from the other MOOSE based applications, much effort has been required to specialize Peacock for RAVEN and further addition will continue to follow RELAP-7 development.
\vspace{-5mm}
