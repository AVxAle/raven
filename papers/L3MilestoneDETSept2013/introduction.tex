\label{sec:introduction}
RAVEN (\textbf{R}eactor \textbf{A}nalysis and \textbf{V}irtual control \textbf{EN}viroment)~\cite{ravenFY12,mandelliANS2012} is a software tool that acts as the control logic driver for the newly developed Thermal-Hydraulic code RELAP-7  (\textbf{R}eactor \textbf{E}xcursion and \textbf{L}eak \textbf{A}nalysis \textbf{P}rogram).
RAVEN has been designed in a highly modular and pluggable way in order to enable easy integration of different programming languages (i.e., C++, Python) and coupling with other MOOSE based and external applications.
\\The goal of this report is to highlight the newly developed  Dynamic Event Tree (DET) module embedded in the code and its utilization in conjunction with RELAP-7.

As for all \textbf{P}robabilistic \textbf{R}isk \textbf{A}ssessment (PRA) software the capability to fully control the plant evolution during the simulation represents a highly desirable feature as it is for the analysis of the propagation of the uncertainty. . For these reasons, a strict interaction between RELAP-7 and RAVEN is a key of the long-term success of the overall project. In system safety analysis codes, a similar need is expressed by the implementation of the control logic of the plant. As a consequence the optimization of resources imposes the integration of this task under a common project that is naturally RAVEN.The final outcome is a very general and flexible implementation of the plant control logic that will easily allow the integration of proprietary information without any change in the RELAP-7 code. This is also regarded as a facilitating factor for the quick deployment of RELAP-7.

In summary, RAVEN is a multi-purpose PRA software framework that allows dispatching different functionalities.
It is designed to derive and actuate the control logic required to simulate the plant control system and operator actions (guided procedures) and to perform both Monte-Carlo sampling  and Event Tree based analysis of the probabilistic behavior of the NPP.
In order to facilitate the input/output handling, a \textbf{G}raphical \textbf{U}ser \textbf{I}nterface (GUI) and a post-processing data mining module (under development) are available.

This report provides an overview of the DET structure, highlighting the mathematical framework from which its structure is derived and its software implementation. In addition a \textbf{S}tation \textbf{B}lack \textbf{O}ut (SBO) DET based analysis of a simplified \textbf{P}ressurized \textbf{W}ater \textbf{R}eactor (PWR) model will be shown.
\vspace{-5mm}

%\subsection{subsection}
%text

%figure template

%\subsubsection{subsubsection}
%more text
%\paragraph{paragraph}
%lot of text
%\subparagraph{subparagraph}
%if you arrive at this point you have issues
